\author{Carsten Becker}
\title{Genusresolution bei mittelhochdeutsch \emph{beide}}
\subtitle{Eine Analyse im Rahmen der Lexical-Functional Grammar}
\renewcommand{\lsSeries}{ahl}
\renewcommand{\lsSeriesNumber}{1}%
\lsCoverTitleSizes{45pt}{15mm}% Font setting for the title page
\renewcommand{\lsID}{431}
\renewcommand{\lsISBNdigital}{978-3-96110-432-1}
\renewcommand{\lsISBNhardcover}{978-3-98554-090-7}
\BookDOI{10.5281/zenodo.10451456}
\BackBody{Bereits Jacob Grimm bemerkte in Band 2 seiner \emph{Geschichte der
  deutschen Sprache} (1848), dass adjektivische Genuskongruenz mit
  gemischtgeschlechtlichem Personenbezug in den älteren Sprachstufen des
  Deutschen häufig das Neutrum aufweist. Askedal (1973) widmete diesem Thema
  eine ausführliche Studie auf Basis kritischer Editionen einer Handvoll alt-
  und mittelhochdeutscher literarischen Werke. Die Standardwerke zur
  historischen Grammatik des Deutschen beschränken sich bislang darauf, diese
  Regel undifferenziert nach grammatischem Kontext lediglich zu konstatieren.

  Die vorliegende Arbeit zeichnet das Phänomen der Genusresolution
  (\emph{gender resolu\-tion}; Corbett 1983) im Mittelhochdeutschen nach. Dies
  geschieht auf handschriftennaher Grundlage anhand der Geschäftsprosa der
  Urkunden des \emph{Corpus der altdeutschen Originalurkunden bis zum Jahr
  1300} (Wilhelm u.\,a.~1932--2004) sowie der erst seit wenigen Jahren
  verfügbaren Transkriptionen der Haupthandschriften aller drei Rezensionen der
  \emph{Kaiserchronik} als literarischem Vergleichstext. Die Studie setzt sich
  zum Ziel, bestehendes Wissen über dieses grammatische Phänomen zu validieren
  und im Rahmen eines zeitgemäßen grammatiktheoretischen Modells zu
  reflektieren. Zu diesem Zweck wird systematisch Variation in der
  Kongruenzform des Quantors mittelhochdeutsch \emph{bėide} \wdef{beide}, der
  sich auf ein Referentenpaar bezieht, in den typischen Kontexten seines
  Auftretens hinsichtlich morphologischer, semantischer und syntaktischer
  Zusammenhänge detailliert untersucht.}

  \typesetter{Carsten Becker}
  \proofreader{Katja Politt,
Lisa Schäfer,
Jean Nitzke,
Lars Erik Zeige,
Tom Bossuyt,
Robin Lemke
}
