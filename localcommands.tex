% Legally required note on this being an accepted dissertation
\renewcommand{\lsImpressumExtra}{Submitted in partial fulfillment of the
requirements for the degree ``Dr.~phil.'' in German Language. Revised Ph.\,D.
thesis, University of Marburg, Department of German Studies and Arts, 2022.}

% Author's ORCID
\newcommand*{\orcid}{0000-0002-6023-551X}

% Blurb
\BackBody{}

% Check and cross mark
\newcommand{\chk}{✔} % Check mark
\newcommand{\crs}{✘} % Cross mark

% Foreign word in text
\newcommand{\fw}[1]{\emph{#1}}

% Gray text
\newcommand{\gr}[1]{\textcolor{gray}{#1}}

% Literally quoted MHG spelling
\newcommand{\lit}[1]{\emph{#1}}

% Normalized MHG spelling
\newcommand{\norm}[1]{\emph{#1}}

% Suppress shorthand field(s) in BibTeX citation on next citation
\newcommand{\nosh}{\AtNextCite{\AtEachCitekey{\clearfield{shorthand}}}}

% Orthography brackets
\newcommand{\orth}[1]{⟨#1⟩}

% Titles of monographies
% langsci-unified relies on defaults set by biblatex.def:
% \DeclareFieldFormat{citetitle}{\mkbibemph{#1}} for books
\newcommand{\tit}[1]{{\emph{#1}}}

% Text-style super- and subscript
\let\tsub\textsubscript
\let\tsup\textsuperscript

% Titles of abbreviated works
\newcommand{\CAO}{\tit{CAO}}
\newcommand{\KC}{\tit{KC}}
\newcommand{\REA}{\tit{ReA}}
\newcommand{\REM}{\tit{ReM}}
\newcommand{\WMU}{\tit{WMU}}

% % See Also for 2nd-level entries in index
% % Adapted from \langsciseealso in langscibook.cls
% \newcommand{\cbseealsollvl}{\hspace*{0.7cm}\hangindent=0.7cm\seealso}

% Shorthand for multicol, multirow
\let\mc\multicolumn
\let\mr\multirow

% Annotation macros
\newcommand{\SM}{♂}
\newcommand{\SF}{♀}
\newcommand{\SI}{⚬}
\newcommand{\SX}{?}
\newcommand{\SA}{∗}% {★}
\newcommand{\SMM}{⚣}
\newcommand{\SFF}{⚢}
\newcommand{\SMF}{⚤}
\newcommand{\SII}{⚭}
\newcommand{\subA}{\textsubscript{\SA}}
\newcommand{\subF}{\textsubscript{\SF}}
\newcommand{\subI}{\textsubscript{\SI}}
\newcommand{\subM}{\textsubscript{\SM}}
\newcommand{\subX}{\textsubscript{\SX}}
\newcommand{\subMM}{\textsubscript{\SMM}}
\newcommand{\subFF}{\textsubscript{\SFF}}
\newcommand{\subMF}{\textsubscript{\SMF}}
\newcommand{\subII}{\textsubscript{\SII}}
\newcommand{\MascA}{\textsc{m}\textsubscript{\SA}} % M∗
\newcommand{\MascF}{\textsc{m}\textsubscript{\SF}} % M♀
\newcommand{\MascI}{\textsc{m}\textsubscript{\SI}} % M⚬
\newcommand{\MascM}{\textsc{m}\textsubscript{\SM}} % M♂
\newcommand{\MascX}{\textsc{m}\textsubscript{\SX}} % M?
\newcommand{\FemA}{\textsc{f}\textsubscript{\SA}}
\newcommand{\FemF}{\textsc{f}\textsubscript{\SF}}
\newcommand{\FemI}{\textsc{f}\textsubscript{\SI}}
\newcommand{\FemM}{\textsc{f}\textsubscript{\SM}}
\newcommand{\FemX}{\textsc{f}\textsubscript{\SX}}
\newcommand{\NeutA}{\textsc{n}\textsubscript{\SA}}
\newcommand{\NeutF}{\textsc{n}\textsubscript{\SF}}
\newcommand{\NeutI}{\textsc{n}\textsubscript{\SI}}
\newcommand{\NeutM}{\textsc{n}\textsubscript{\SM}}
\newcommand{\NeutX}{\textsc{n}\textsubscript{\SX}}
\newcommand{\NeutMM}{\textsc{n}\textsubscript{\SMM}}
\newcommand{\NeutMF}{\textsc{n}\textsubscript{\SMF}}
\newcommand{\NeutII}{\textsc{n}\textsubscript{\SII}}
\newcommand{\NullA}{Ø\textsubscript{\SA}}
\newcommand{\NullF}{Ø\textsubscript{\SF}}
\newcommand{\NullI}{Ø\textsubscript{\SI}}
\newcommand{\NullM}{Ø\textsubscript{\SM}}
\newcommand{\NullX}{Ø\textsubscript{\SX}}
\newcommand{\UnknA}{?\textsubscript{\SA}}
\newcommand{\UnknF}{?\textsubscript{\SF}}
\newcommand{\UnknI}{?\textsubscript{\SI}}
\newcommand{\UnknM}{?\textsubscript{\SM}}
\newcommand{\UnknX}{?\textsubscript{\SX}}

% LFG macros
\newcommand{\downfeat}[2]{(↓~{\textsc{#1}})~=~{\textsc{#2}}} % (↓X) = Y
\newcommand{\downs}[1]{(↓~{\textsc{#1}})} % (↓X)
\newcommand{\elem}[1]{↓~$\in$~(↑~{\textsc{#1}})} % ↓ ∊ (↑X)
\newcommand{\pass}[1]{(↑~{\textsc{#1}})~=~↓} % (↑X) = ↓
\newcommand{\req}{=\textsubscript{c}} % =_c
\newcommand{\uncertain}[2]{((\textsc{#1}~↑)~{\textsc{#2}})} % ((X↑) Y)
\newcommand{\updown}{↑~=~↓} % ↑ = ↓
\newcommand{\updownelem}{↓~$\in$~↑} % ↓ ∊ ↑
\newcommand{\upfeat}[2]{(↑~{\textsc{#1}})~=~{\textsc{#2}}} % (↑X) = Y
\newcommand{\ups}[1]{(↑~{\textsc{#1}})} % (↑X)
\newcommand{\xbar}[1]{#1′} % X'
\newcommand{\xhead}[1]{#1⁰} % X⁰

% Secondary object and oblique argument
\newcommand{\SObj}[1]{%
  \notblank{#1}{%
    \textsc{obj}\textsubscript{\normalfont #1}%
  }{%
    \textsc{obj}\textsubscript{\normalfont θ}%
  }%
}
\newcommand{\Oblq}[1]{%
  \notblank{#1}{%
    \textsc{obl}\textsubscript{\normalfont #1}%
  }{%
    \textsc{obl}\textsubscript{\normalfont θ}%
  }%
}

% A-structure rules: \astruct[1]{2}{3} > '2 <3> 1'
\newcommand{\astruct}[3][]{`#2~⟨\,#3\,⟩%
	\notblank{#1}{%
		\enspace#1'%
	}{'}%
}

% Feature annotation above (for use in TREES); no value in [] prints ↑ = ↓
\newcommand{\anno}[2][\updown{}]{%
	$\overset{\text{#1}}{\text{#2}}$%
}

% tikz
% Alias for tikzmark's \subnode[]{}{} to spare lots of empty brackets for nodes
% not containing any text anyway
\newcommand{\mysn}[1]{\subnode{#1}{}}

% tikz
% Definition copied from pgfsys-common-pdf-via-dvi.def
% Compare http://tex.stackexchange.com/q/229500 and comments!
% This is currently needed to make tikzmark compile with xelatex
\makeatletter
\def\pgfsys@hboxsynced#1{%
  {%
    \pgfsys@beginscope%
    \setbox\pgf@hbox=\hbox{%
      \hskip\pgf@pt@x%
      \raise\pgf@pt@y\hbox{%
        \pgf@pt@x=0pt%
        \pgf@pt@y=0pt%
        \special{pdf: content q}%
        \pgflowlevelsynccm% 
        \pgfsys@invoke{q -1 0 0 -1 0 0 cm}%
        \special{pdf: content -1 0 0 -1 0 0 cm q}% translate to original coordinate system
        \pgfsys@invoke{0 J [] 0 d}% reset line cap and dash
        \wd#1=0pt%
        \ht#1=0pt%
        \dp#1=0pt%
        \box#1%
        \pgfsys@invoke{n Q Q Q}%
      }%
      \hss%
    }%
    \wd\pgf@hbox=0pt%
    \ht\pgf@hbox=0pt%
    \dp\pgf@hbox=0pt%
    \pgfsys@hbox\pgf@hbox%
    \pgfsys@endscope%
  }%
}
\makeatother
