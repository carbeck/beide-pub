% Author's ORCID
\newcommand*{\orcid}{0000-0002-6023-551X}

% Check and cross mark
\newcommand{\chk}{{\fontspec{FreeSerif}✓}} % Check mark
\newcommand{\crs}{{\fontspec{FreeSerif}✗}} % Cross mark

% Foreign word in text
\newcommand{\fw}[1]{\emph{#1}}

% Gray text
\newcommand{\gr}[1]{\textcolor{gray}{#1}}

% Literally quoted MHG spelling
\newcommand{\lit}[1]{\emph{#1}}

% Normalized MHG spelling
\newcommand{\norm}[1]{\emph{#1}}

% Suppress shorthand field(s) in BibTeX citation on next citation
\newcommand{\nosh}{\AtNextCite{\AtEachCitekey{\clearfield{shorthand}}}}

% Titles of monographies
\newcommand{\tit}[1]{{\emph{#1}}}

% Titles of abbreviated works
\newcommand{\CAO}{\tit{CAO}}
\newcommand{\KC}{\tit{KC}}
\newcommand{\REM}{\tit{ReM}}
\newcommand{\WMU}{\tit{WMU}}

% Shorthand for multicol, multirow
\let\mc\multicolumn
\let\mr\multirow

% biblatex
% Make divider for multicites and multipostnotes semicolon
\renewcommand{\multicitedelim}{\addsemicolon\addspace}
\renewcommand{\multipostnotedelim}{\addsemicolon\addspace}

% Annotation macros
\newcommand{\SM}{\fontspec{FreeSerif}♂}
\newcommand{\SF}{\fontspec{FreeSerif}♀}
\newcommand{\SI}{\fontspec{FreeSerif}⚪}
\newcommand{\SX}{\fontspec{FreeSerif}?}
\newcommand{\SA}{\fontspec{FreeSerif}∗}
\newcommand{\SMM}{\fontspec{FreeSerif}⚣}
\newcommand{\SFF}{\fontspec{FreeSerif}⚢}
\newcommand{\SMF}{\fontspec{FreeSerif}⚤}
\newcommand{\SII}{\fontspec{FreeSerif}⚭}
\newcommand{\subA}{\textsubscript{\SA}}
\newcommand{\subF}{\textsubscript{\SF}}
\newcommand{\subI}{\textsubscript{\SI}}
\newcommand{\subM}{\textsubscript{\SM}}
\newcommand{\subX}{\textsubscript{\SX}}
\newcommand{\subMM}{\textsubscript{\SMM}}
\newcommand{\subFF}{\textsubscript{\SFF}}
\newcommand{\subMF}{\textsubscript{\SMF}}
\newcommand{\subII}{\textsubscript{\SII}}
\newcommand{\MascA}{\textsc{m}\textsubscript{\SA}} % M∗
\newcommand{\MascF}{\textsc{m}\textsubscript{\SF}} % M♀
\newcommand{\MascI}{\textsc{m}\textsubscript{\SI}} % M⚪
\newcommand{\MascM}{\textsc{m}\textsubscript{\SM}} % M♂
\newcommand{\MascX}{\textsc{m}\textsubscript{\SX}} % M?
\newcommand{\FemA}{\textsc{f}\textsubscript{\SA}}
\newcommand{\FemF}{\textsc{f}\textsubscript{\SF}}
\newcommand{\FemI}{\textsc{f}\textsubscript{\SI}}
\newcommand{\FemM}{\textsc{f}\textsubscript{\SM}}
\newcommand{\FemX}{\textsc{f}\textsubscript{\SX}}
\newcommand{\NeutA}{\textsc{n}\textsubscript{\SA}}
\newcommand{\NeutF}{\textsc{n}\textsubscript{\SF}}
\newcommand{\NeutI}{\textsc{n}\textsubscript{\SI}}
\newcommand{\NeutM}{\textsc{n}\textsubscript{\SM}}
\newcommand{\NeutX}{\textsc{n}\textsubscript{\SX}}
\newcommand{\NeutMM}{\textsc{n}\textsubscript{\SMM}}
\newcommand{\NeutMF}{\textsc{n}\textsubscript{\SMF}}
\newcommand{\NeutII}{\textsc{n}\textsubscript{\SII}}
\newcommand{\NullA}{Ø\textsubscript{\SA}}
\newcommand{\NullF}{Ø\textsubscript{\SF}}
\newcommand{\NullI}{Ø\textsubscript{\SI}}
\newcommand{\NullM}{Ø\textsubscript{\SM}}
\newcommand{\NullX}{Ø\textsubscript{\SX}}
\newcommand{\UnknA}{?\textsubscript{\SA}}
\newcommand{\UnknF}{?\textsubscript{\SF}}
\newcommand{\UnknI}{?\textsubscript{\SI}}
\newcommand{\UnknM}{?\textsubscript{\SM}}
\newcommand{\UnknX}{?\textsubscript{\SX}}

% LFG macros
\newcommand{\downfeat}[2]{(↓~{#1})~=~{#2}} % (↓X) = Y
\newcommand{\downs}[1]{(↓~{#1})} % (↓X)
\newcommand{\elem}[1]{↓~$\in$~(↑~{#1})} % ↓ ∊ (↑X)
\newcommand{\pass}[1]{(↑~{#1})~=~↓} % (↑X) = ↓
\newcommand{\req}{=$_c$} % =_c
\newcommand{\uncertain}[2]{((#1~↑)~{#2})} % ((X↑) Y)
\newcommand{\updown}{↑~=~↓} % ↑ = ↓
\newcommand{\updownelem}{↓~$\in$~↑} % ↓ ∊ ↑
\newcommand{\upfeat}[2]{(↑~{#1})~=~{#2}} % (↑X) = Y
\newcommand{\ups}[1]{(↑~{#1})} % (↑X)
\newcommand{\xbar}[1]{#1′} % X'
\newcommand{\xhead}[1]{#1⁰} % X⁰

% \SObjT and \OblqT must be manually added to the leipzig glossary listing
\newcommand{\SObj}[1]{\textsc{obj}$_{#1}$}
\newcommand{\Oblq}[1]{\textsc{obl}$_{#1}$}

% A-structure rules: \astruct[1]{2}{3} > '2 <3> 1'
\newcommand{\astruct}[3][]{`#2~⟨\,#3\,⟩%
	\notblank{#1}{%
		\enspace#1'}{%
		}'%
}

% Feature annotation above (for use in TREES); no value in [] prints ↑ = ↓
\newcommand{\anno}[2][\updown{}]{%
	$\overset{\text{#1}}{\text{#2}}$%
}

% Alias for tikzmark's \subnode[]{}{} to spare lots of empty brackets for nodes
% not containing any text anyway
\newcommand{\mysn}[1]{\subnode{#1}{}}
