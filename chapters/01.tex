\chapter{Einleitung}
\label{ch:einleitung}

Im Unterschied zum modernen Standarddeutschen\il{Neuhochdeutsch} besitzen die
oberdeutschen\il{Oberdeutsch} Schreib\-dialekte des
Mittelhochdeutschen\il{Mittelhochdeutsch} im Nominativ und Akkusativ Plural der
starken \isi{Adjektivdeklination} noch zwei Formen: eine maskulin-feminine auf
\norm{-e} (\ref{ex:adjsnglmpl}--b) und eine neutrale auf
\norm{-iu} (\ref{ex:adjsnglnpl}). Die in \REF{ex:adjsngl} angeführten
Beispiele stellen den Normalfall der Kongruenz eines attributiven
Adjektivs\is{Adjektiv!attributiv} mit einem einzelnen \isi{Substantiv} im
Mittelhochdeutschen\il{Mittelhochdeutsch} oberdeutscher\il{Oberdeutsch} Prägung
dar \autocites[vgl.][181--184]{ksw2}[200--203]{paul2007}.

\begin{exe}
\ex \label{ex:adjsngl}
	\begin{xlist}
	\ex \label{ex:adjsnglmpl}
		% \gll alſ gvͦte riter ſolden \\
		\gll als guote rîter solden \\
			wie gut-\textsc{nom.pl.m.st} Ritter[\textsc{nom.pl.m}] sollten \\
	\trans `wie gute Ritter sollten'
		(\iai{Hartmann von Aue}, \tit{Iwein}: V.~5345;
			\cite[606]{mertens2004}%
			% und Gießen, Universitätsbibl., Hs.~97: 103\tsup{v}, 9
			% [\cite[1102]{hsc}]%
		)

	\ex \label{ex:adjsnglfpl}
		% \gll Do ſazen ſchoͤne froͮwen naht vnd tach \\
		\gll Dô sâzen scœne frouwen náht únde tac \\
			da saßen schön-\textsc{nom.pl.f.st} Frau-\textsc{nom.pl.f} Nacht 
			und	Tag \\
	\trans `da saßen schöne Frauen Nacht und Tag'
		(%
			\tit{Nibelungenlied}: 65,1;
			\cite[16]{deboor1988}%
			% und St.~Gallen, Stiftsbibl., Cod.~Sang.~857: 293\tsup{b}, 30--31
			% [\cite[1211]{hsc}]%
		)

	\ex \label{ex:adjsnglnpl}
		% \gll daz ſi richiv chleider tragn \\
		\gll daz si richiu kleider tragn \\
			dass sie vornehm-\textsc{acc.pl.n.st} Kleid-\textsc{acc.pl.n}
			tragen \\
		\trans `dass sie vornehme Kleider tragen'
			(%
				\iai{Wolfram von Eschenbach}, \tit{Parzival}: 22,20;
				\cite[24]{knechtschirok2003}%
				% und St.~Gallen, Stiftsbibl., Cod.~Sang.~857: 11\tsup{a}, 3
				% [\cite[1211]{hsc}]%
			)
\end{xlist}
\end{exe}

Angesichts dieser Rahmenbedingung\is{Bedingung} hat die vorliegende Studie zum
Ziel, exemplarisch die Setzung der Wortformen \norm{bėide} beziehungsweise
\norm{bėidiu} von mittelhochdeutsch\il{Mittelhochdeutsch} \norm{bėide} `beide'
zu untersuchen, insofern dieses ebenfalls adjektivisch
dekliniert\is{Adjektivdeklination} wird und sich typischerweise auf eine Menge
von \emph{zwei} Personen oder Dingen bezieht, die nicht notwendigerweise das
gleiche Genus besitzen. Im Vordergrund steht dabei die \isi{Quantifizierung} von
Variation zwischen den beiden Wortformen beim gleichzeitigen Bezug auf
unterschiedliche Kombinationen von formalem oder semantischem Geschlecht --
\isi{Genus} oder \isi{Sexus} -- ihrer Referenten in ansonsten gleichen
syntaktischen Kontexten (\sectref{sec:einlgendres}). Unter Berücksichtigung der
schreibdialektalen\is{Schreibdialekt} Gliederung des untersuchten
Sprachraums\is{Dialektgeografie} wird nach Bedingungen\is{Bedingung} geforscht,
die die beobachtete Variation motivieren oder begünstigen.

Die Studie verfolgt damit die Absicht, ein eher selten auftretendes Phänomen
innerhalb der \isi{Genuskongruenz} im Mittelhochdeutschen\il{Mittelhochdeutsch}
anhand historischer Originaltexte genau zu untersuchen, um bestehendes Wissen
darüber zu validieren und weiter zu differenzieren. Beim gleichzeitigen Bezug
einer kongruierenden Wortform auf zwei Kongruenz auslösende Größen handelt es
sich zudem um einen \q{nicht-kanonischen} Fall von
Kongruenz\is{Kongruenzrelation!nicht-kanonische} \autocite[8--27,
59--60]{corbett2006}, der durch den Abbau der Genuskategorie im Plural der
\isi{Adjektivdeklination} seit mittelhochdeutscher\il{Mittelhochdeutsch} Zeit
(ca.\ 1050--1350) verloren gegangen ist.

Neben quantifizierendem \norm{bėide} existiert im
Mittelhochdeutschen\il{Mittelhochdeutsch} die korrelative Konjunktion
\norm{bėide \dots\ unde} `sowohl \dots\ als auch', bei der \norm{bėide}
historisch\is{Diachronie} ebenfalls in pronominaler Funktion kongruiert
(\sectref{sec:einlbeidekonj}). Im Mittelhochdeutschen\il{Mittelhochdeutsch}
treten im Kontext dieser Konstruktion lexikalisiert sowohl
\norm{bėide} als auch \norm{bėidiu} auf. Auch hier besteht die Frage, welche
Faktoren das Auftreten der einen oder der anderen Form beeinflussen und
inwiefern die ausgewerteten Quellen bestehendes Wissen validieren.

Als theoretische Grundlage der Untersuchung dienen zum einen die typologischen
Studien von \citet{corbett1979,corbett1983,corbett1991,corbett2006} zum
Phänomen der Kongruenz an sich. Zum anderen bietet die \fw{Lexical-functional
Grammar}\is{Lexical-functional Grammar} (LFG;
\cites{kaplanbresnan1982}{bresnan2001}{bresnanetal2016}) einen angemessenen
grammatiktheoretischen Rahmen zur Diskussion der Beobachtungen. Grammatische
Kongruenz stellt ein genuin morphosyntaktisches Phänomen dar. Der Ansatz der
LFG\is{Lexical-functional Grammar} mit ihrer Modellierung von Kongruenz als
Vereinigung von lexikalisch teildefinierten, im Regelfall kompatiblen
grammatischen Merkmalen\is{Merkmale!grammatische} auf funktionaler Ebene ist
nützlich, um vor allem den Unterschied zwischen semantischer und formaler
Kongruenz über den Abgleich entsprechender grammatischer und semantischer
Merkmale\is{Merkmale!semantische} zu fassen (\sectref{sec:lfgkongr}).

Mit \citet[171--195]{wechslerzlatic2003} beziehungsweise \citet{wechsler2009}
liegt ein Erklärungsversuch zur Funktionsweise des Ausgleichs von verschiedenen
zu kombinierenden Genusmerkmalen\is{Genusmerkmal} vor dem Hintergrund dieses
grammatik\-theoretischen Rahmens vor. Doch auch bereits die Überlegungen von
\citet[237--264]{askedal1973} beziehen explizit den Ab- und Ausgleich von
semantischen\is{Merkmale!semantische} und formalen
Merkmalen\is{Merkmale!grammatische} als Grundlage für Kongruenz ein. Durch die
Wahl der LFG\is{Lexical-functional Grammar} als theoretischer Rahmen für diese
Arbeit lässt sich daher sehr gut an bestehende Forschung zum Thema anknüpfen.

Als Grundlage der Untersuchung dienen hauptsächlich zwei Materialsammlungen,
nämlich erstens das \citefield{cao1}[citetitle]{maintitle} (\CAO{};
\nosh\cites{cao1,cao2,cao3,cao4,caor,cao5}), das die umfangreichste Sammlung
deutschsprachiger Urkunden\is{Urkunde} hauptsächlich des späten
13.~Jahrhunderts in diplomatischer \isi{Transkription} darstellt. Das \CAO{}
ist in den 2000er Jahren digitalisiert worden \autocite{cao-online}.%
%
	\footnote{Siehe \citeurl{cao-online}. Die Digitalisierung der
		Schwarzweißfotos\is{Digitalisat} der Originalurkunden\is{Urkunde}, die
		für die Erarbeitung des \tit{Wörterbuchs der
		Mittelhochdeutschen\il{Mittelhochdeutsch} Urkundensprache} (\WMU{};
		\nosh\cites{wmu1,wmu2,wmu3}) angefertigt wurden, ist im Rahmen des
		Projekts \tit{Marburger Urkundenrepositorium} in Arbeit.}
%
Die Urkundentexte sind dadurch für Untersuchungen weitaus zugänglicher als die
gedruckten Bände, insbesondere nach der Überführung der Texte und Metadaten in
eine \citetitle{postgresql}-\isi{Datenbank}
\autocites[207]{beckerschallert2021}[155--156]{beckerschallert2022b}.

Das \CAO{} ist für sprachhistorische Untersuchungen reizvoll, da ein großer
Teil der über 4.500 enthaltenen Urkunden\is{Urkunde} datiert ist und diese sich
in vielen Fällen direkt oder indirekt dem näheren Umkreis eines
Ausstellungsorts zuordnen lassen. Durch die Menge der Urkunden und das dichte
Ortsnetz ist es möglich, sprachliche Merkmale des lebensnahen
Gebrauchsschrifttums der abgedeckten Regionen zu ihrer Zeit herauszuarbeiten.
Aufgrund ihres teils formularhaften Charakters sowie der großen Zahl an nicht
identifizierbaren Schreiberinnen und Schreibern stellten Urkunden in der
historischen Linguistik\is{Sprachgeschichte} des Deutschen lange Zeit eine
vernachlässigte, mit Misstrauen betrachtete Textgattung dar
(\sectref{sec:materialcao}). \citet[22]{schulze2011} betont dagegen den Nutzen
von Urkunden, insofern es \blockquote{\textins*{b}ei syntaktischen
Untersuchungen verschiedener Art \textelp{} viele übergreifende Fragen und
Beobachtungen \textins{gibt}, für die das besondere Quellenmaterial ergiebig
ist und die allenfalls zusätzlich sprachgeographisch\is{Dialektgeografie}
weiter differenziert werden können.}

Eine repräsentative Auswahl vollständiger Textzeugen aller drei Rezensionen der
\citetitle{kc} (\KC{}; vgl.~\cites{schroeder1895,nellmann1983}), der
bedeutendsten mittelhochdeutschen\il{Mittelhochdeutsch} \isi{Reimchronik},
dient als zweite Textquelle. Im Rahmen des Cambridge-Marburger Projekts zur
Neuedition\is{Editionsphilologie} der \KC{} \autocite[vgl.][]{chincaetal2019b}
wurden diplomatische Transkriptionen\is{Transkription} aller derzeit bekannten
Textzeugen dieses Werks erstmals digital verfügbar gemacht.%
%
	\footnote{Digitalisate\is{Digitalisat} und
	Transkriptionen\is{Transkription} sämtlicher Handschriften und Fragmente
	befinden sich in \citetitle{kcdigital}, siehe
	\citeurl{kcdigital}\nocite{kcdigital}.}
%
Die \KC{} ist in insgesamt fünfzig Handschriften vom 12.\ bis zum
16.~Jahrhundert überliefert und gilt als \blockcquote[93]{wolf2008}{\emph{das}
volkssprachig-deutsche Erfolgswerk des späten 12.\ und frühen 13.~Jh.s}. Der
Schwerpunkt der Editionstätigkeit\is{Editionsphilologie} seit dem
19.~Jahrhundert lag hauptsächlich auf ihrer ältesten Bearbeitungsschicht, der
Rezension A, mit ihrem Hauptvertreter, der \tit{Vorauer Handschrift 276} (A1
bzw.~V; \cite[vgl.][]{gaertner1999}). Da die Texte der
Überlieferungsträger\is{Überlieferung} der \KC{} nun digital zugänglich sind,
kann hier erstmals das Textkorpus\is{Korpus} aller drei Rezensionen im Rahmen
einer sprachhistorischen Analyse ausgewertet werden.

Mit der Beschränkung\is{Beschränkung} auf das \CAO{} und die \KC{} werden nicht
sämtliche zum Abfassungszeitpunkt elektronisch verfügbaren
mittelhochdeutschen\il{Mittelhochdeutsch} Texte oder Textsammlungen
ausgewertet. Allerdings liegen nach wie vor nur Texte dieser Periode digital in
diplomatischer Transkription\is{Transkription} vor. Als eine prominente
Textsammlung neben den hier ausgewerteten Quellen ist das \citetitle{rem}
(\REM; \cite{rem}) zu nennen, ein morphologisch annotiertes \isi{Korpus}, das
auszugsweise verschiedenste Reim- und Prosatexte\is{Prosa} aus dem 11.\ bis
14.~Jh.\ enthält \autocite[siehe auch][]{wegera2000}.

Auf Texte begrenzt, deren \isi{Schreibdialekt} als oberdeutsch\il{Oberdeutsch}
klassifiziert wurde (d.\,h.\ oberdeutsch\il{Oberdeutsch} ohne dialektale
Spezifikation, bairisch\il{Bairisch}, alemannisch\il{Alemannisch}, das
bai\-risch\il{Bairisch}-alemannische\il{Alemannisch}
Übergangsgebiet\il{Schwäbisch} und ostfränkisch\il{Ostfränkisch}), bietet es
für den Quantor \norm{bėide} `beide' im Nom./Akk.~Pl.~st.\ 220 Belege, für
\norm{bėidiu} mit der gleichen morphologischen Annotation\is{Annotation} 97
Belege, sowie in Bezug auf die korrelative Konjunktion \norm{bėide \dots\ unde}
`sowohl \dots\ als auch' 103 Belege für \norm{bėide} und 217 für \norm{bėidiu}.
Für die zwei Funktionen von \norm{bėide} sind also insgesamt 637 Belege
verfügbar -- wobei noch zu klären wäre, ob die darunter vertretenen Texte
tatsächlich beide Formen, \norm{bėide} und \norm{bėidiu}, enthalten. Das \REM{}
ist damit ähnlich umfangreich wie das \CAO{} und die \KC{} zusammen, aus denen
insgesamt 875 Belegstellen gesammelt wurden, von denen jedoch nicht alle in die
Auswertung eingeflossen sind. Auch wenn das \REM{} in dieser Studie nicht als
Quelle dient, wird es an mehreren Stellen zum Vergleich herangezogen.

Darüber hinaus befinden sich die Projekte \citetitle{iwdigital}
\autocite{iwdigital}, \citetitle{ldmdigital} \autocite{ldmdigital} und das
Berner Parzival-Projekt \autocite{parzivalprojekt} zu literarischen Verstexten
des deutschsprachigen Mittelalters im Aufbau.%
%
	\footnote{Die jeweiligen Projekte sind unter \citeurl{iwdigital},
	\citeurl{ldmdigital} und \citeurl{parzivalprojekt} zu finden.}
%
Wenn sich diese Studie exemplarisch auf die \KC{} als Beispiel für einen breit
überlieferten, literarisch gestalteten Text und ein zum Teil editorisches Novum
beschränkt, ist zu bedenken, dass auch der \tit{Iwein} Hartmanns von
Aue\ia{Hartmann von Aue} und der \tit{Parzival} Wolframs von
Eschenbach\ia{Wolfram von Eschenbach} eine vielfältige und komplexe
\isi{Überlieferung} besitzen (\cite[vgl.][s.\,v.~\textit{\iai{Hartmann von
Aue}: \tit{Iwein}}, \textit{\iai{Wolfram von Eschenbach}: \tit{Parzival}}]{hsc}
sowie die dort verzeichnete Literatur). Diese ebenfalls in Gänze aufzuarbeiten,
stellt ein \isi{Desiderat} dar, das mit einem Anschlussprojekt an die
vorliegende Arbeit verfolgt werden könnte.

\section{\norm{Bėide} und das Phänomen der \fw{Gender Resolution}}
\label{sec:einlgendres}
\is{Gender Resolution|(}

Der Ausgleich von unterschiedlichen Genera\is{Genus} beim gleichzeitigen Bezug
auf mehrere Referenten in der grammatischen Markierung von flektierenden
Wortarten, der in dieser Arbeit den ersten zu behandelnden Themenkomplex
darstellt, wird in der von \citet{corbett1983} geprägten Terminologie als
\fw{Gender Resolution} bezeichnet. Das Auftreten des Neutrums als
Resolutionsgenus oder \fw{\isi{Default}} \autocites{corbett1999}{wechsler2009}
in diesem Kontext kann heute noch im Isländischen\il{Isländisch}
\REF{ex:modgermbeide_1} und im Färöischen\il{Färöisch} \REF{ex:modgermbeide_2}
beobachtet werden, die die Dreiteilung des Genus im Plural beibehalten haben,
während sie im Deutschen\il{Neuhochdeutsch} komplett abgebaut wurde.

\begin{exe}
\ex \label{ex:modgermbeide}
\begin{xlist}
\ex \label{ex:modgermbeide_1}
	\langinfo%
		{Isländisch}%
		{}%
		{\cites[nach][283]{corbett1991}[569]{wechsler2009}}\\
	\gll Drengurinn og telpan eru þreytt. \\
		Junge[\textsc{m.sg}] und Mädchen[\textsc{f.sg}] sind
		müde[\textsc{n.pl}] \\
	\trans `Der Junge und das Mädchen sind müde.'

\ex \label{ex:modgermbeide_2}
	\langinfo%
		{Färöisch}%
		{}%
		{\cite[nach][225]{thrainsson2004}}\\
	\gll Garðarnir og gøturnar eru vøkur. \\
		Gärten[\textsc{m.pl}] und Straßen[\textsc{f.pl}] sind
		schön[\textsc{n.pl}] \\
	\trans `Die Gärten und die Straßen sind schön.'
\end{xlist}
\end{exe}

Die systematische Übereinstimmung der Form einer flektierenden Wortart in
Abhängigkeit von den grammatischen Merkmalen\is{Merkmale!grammatische} ihrer
Referenz wird als \emph{Kongruenz} bezeichnet. Das Interesse der historischen
Linguistik\is{Sprachgeschichte} an Kongruenz mit Bezug auf mehrere Referenten
wie bei den Adjektiven\is{Adjektiv!prädikativ} \fw{þreytt} `müde' auf
\fw{drengurinn} `der Junge' und \fw{telpan} `das Mädchen' in
\REF{ex:modgermbeide_1} sowie \fw{vøkur} `schön' auf \fw{garðarnir} `die
Gärten' und \fw{gøturnar} `die Straßen' in \REF{ex:modgermbeide_2} ist an sich
nicht neu. Bereits
\textcites[978]{grimm1848}[307--314]{grimm1890}[313--336]{grimm1898} sowie
\citet[244--247]{delbrueck1900} und \citet[33--41]{behaghel1928} reißen dieses
Thema an. Im Vordergrund steht dabei die Beobachtung, dass in den
historischen\is{Diachronie} Sprachstufen germanischer\il{Germanisch} Sprachen
beim gleichzeitigen Bezug auf mehrere belebte\is{Animata} Nomina mit
unterschiedlichem Genus sehr häufig -- aber nicht immer -- die neutrale Form
von Adjektiven\is{Adjektiv} oder Pronomina steht und nicht die maskuline, wie
es in vielen anderen indogermanischen\il{Indogermanisch} Sprachen üblich ist
\autocites[vgl.~z.\,B.][]%
	% [152--154]%
	{hock2008}%
	% [29--31]%
	{hock2009}%
	% [49--50]%
	{hock2012}{corbett1983}[206--207]{fritzmeierbruegger2021}. Die neutrale
Form erscheint darüber hinaus beim kombinierten Bezug auf
unbelebte\is{Inanimata} Referenten mit unterschiedlichem \isi{Genus}. Beides
wird in \REF{ex:mhgbeidiu} für das Mittelhochdeutsche\il{Mittelhochdeutsch}
illustriert.%
%
	\footnote{Die Grafie der mittelhochdeutschen\il{Mittelhochdeutsch}
		Beispiele folgt dem handschriftlichen Original weitest\-möglich.
		Normalisierte\is{Normalisierung} Schreibweisen alt-\il{Althochdeutsch}
		und mittelhochdeutscher\il{Mittelhochdeutsch} Wortformen im Text folgen
		den Konventionen von \textcites[579--581]{ksw3}[989--991]{ksw2}.
		Angaben zum Ausstellungsort und -jahr von Beispielen aus dem \CAO{}
		richten sich nach den Metadaten der Online\-edition von
		\citet{cao-online}.%
	}

\begin{exe}
\ex \label{ex:mhgbeidiu}
	\begin{xlist}
	\ex \label{ex:mhgbeidiu_1}
		\gll vlrich vnd frow Elzbet \textelp{} diſen prief / den ſi beidev
				habent gebeten ze ſchriben \\
			%
			Ulrich[\textsc{nom.sg.m}] und Frau Elsbeth[\textsc{nom.sg.f}] {}
				diesen Urkunde / den \textsc{3pl.nom}
				beide-\textsc{nom.pl.n.st} haben gebeten zu schreiben \\
		\trans `Ulrich und Frau Elsbeth \textelp{} diese Urkunde, die sie
			beide zu schreiben gebeten haben'
			\parencites%
				(Nr.~2843, Salzburg, 1297)%
				[176,26--27]{cao4}

	\ex \label{ex:mhgbeidiu_2}
		% \setlength{\glossglue}{5pt plus 2pt minus 1pt}
		\gll den Hof \textelp{} vnd di Mule / di er paidev \textelp{} hot \\
		%
			den Hof[\textsc{acc.sg.m}] {} und die Mühle[\textsc{acc.sg.f}] {}
				\textsc{rel.acc.pl} er beide-\textsc{nom.pl.n.st} {} hat \\
		\trans `den Hof und die Mühle, die er beide \textelp{} besitzt'
			\parencites%
				(Nr.~2011, Seitenstetten, Bz.~Amstetten, 1294)%
				[254,35--37]{cao3}
	\end{xlist}
\end{exe}

In beiden gezeigten Fällen steht bei gleichzeitigem Bezug auf Personen oder
Dinge mit verschiedenem Genus -- Maskulinum bei \lit{vlrich} `Ulrich' und
\lit{Hof} `Hof', Femininum bei \lit{Elzbet} `Elsbeth' und \lit{Mule}
`Mühle' -- eine Form von \norm{bėide}, die sich dem Plural Neutrum
\norm{-iu} zuordnen lässt.%
%
	\footnote{In bairischen\il{Bairisch} Schreibdialekten\is{Schreibdialekt}
		des 13.~Jahrhunderts macht sich die neuhochdeutsche
		Diphthongierung\is{Diphthongierung, neuhochdeutsche} bemerkbar durch
		die Realisation von mittelhochdeutsch\il{Mittelhochdeutsch} \norm{iu}
		/yː/ als \lit{eu}
		\autocites[74--77]{paul2007}.}
%
Daneben stehen allerdings auch Belege wie der in \REF{ex:mhgbeide} gezeigte.
Dort bezieht sich \lit{baide} `beide' auf \lit{Ernſt} `Ernst' und
\lit{Gerdroͤvt} `Gertraut' und damit ebenfalls auf Personen von
verschiedenem Geschlecht. Trotzdem tritt in diesem Beispiel nicht die neutrale
Form \norm{bėidiu} auf, sondern die maskulin-feminine Form \norm{bėide}.
% Sowohl die Beispiele in \REF{ex:mhgbeidiu} als auch das in
% \REF{ex:mhgbeide} sind im mittelbairischen\il{Bairisch}
% Sprachraum zu verorten.

\begin{exe}
\ex \label{ex:mhgbeide}
	\gll Her Ernst \textelp{} vnd \textelp{} ver Gerdroͤvt \textelp{} da ſi
			baide lebten \\
		%
		Herr Ernst[\textsc{nom.sg.m}] {} und {} Frau
			Gertraut[\textsc{nom.sg.f}] {} als \textsc{3pl.nom}
			beide-\textsc{nom.pl.m+f.st} lebten \\
		\trans `Herr Ernst \textelp{} und \textelp{} Frau Gertraut
			\textelp{} als sie beide am Leben waren'
			\parencites%
				(Nr.~1073, Wien, 1289)%
				[374,40--41]{cao2}
\end{exe}

\afterpage{%
\begin{table}
\centering
\caption%
	{Ausschnitt der althochdeutsch\il{Althochdeutsch} (pronominal-)starken
	Adjektivflexion\is{Adjektivdeklination} \autocites[vgl.][300]{braune2018}}
\begin{tabular}{l c c c c}
	\lsptoprule

	% \mr{2}{*}[-.5ex]{Kasus}
		& \textsc{sg}
		& \mc{3}{c}{\textsc{pl}}
	\\

	\cmidrule(lr){2-2}
	\cmidrule(l){3-5}

	%
		& \textsc{f}
		& \textsc{n}
		& \textsc{m}
		& \textsc{f}
		\\

	\midrule

	\textsc{nom}
		& \cellcolor{black!50}{-iu}
		& \cellcolor{black!50}{-iu}
		& \cellcolor{black!33}{-e}
		& \cellcolor{black!67}{\color{white}{-o}}
		\\

	\textsc{acc}
		& -a
		& \cellcolor{black!50}{-iu}
		& \cellcolor{black!33}{-e}
		& \cellcolor{black!67}{\color{white}{-o}}
		\\

	\lspbottomrule
\end{tabular}
\label{tab:ahd_stradj}
\end{table}

\begin{table}
\centering
\caption%
	{Ausschnitt der mittelhochdeutschen \il{Mittelhochdeutsch}
	(pronominal-)starken Adjektivflexion\is{Adjektivdeklination}
	\autocites[vgl.][260]{ksw2}}
\begin{tabular}{l c c c}
	\lsptoprule

	% \mr{2}{*}[-.5ex]{Kasus}
		& \textsc{sg}
		& \mc{2}{c}{\textsc{pl}}
	\\

	\cmidrule(lr){2-2}
	\cmidrule(l){3-4}

	%
		& \textsc{f}
		& \textsc{n}
		& \textsc{m+f}
		\\

	\midrule

	\textsc{nom}
		& \cellcolor{black!50}{-iu}
		& \cellcolor{black!50}{-iu}
		& \cellcolor{black!33}{-e}
		\\

	\textsc{acc}
		& \cellcolor{black!33}{-e}
		& \cellcolor{black!50}{-iu}
		& \cellcolor{black!33}{-e}
		\\

		\lspbottomrule
\end{tabular}
\label{tab:mhd_stradj}
\end{table}

\begin{table}
\centering
\caption%
	{Ausschnitt der neuhochdeutschen\il{Neuhochdeutsch} (pronominal-)starken
	Adjektivflexion\is{Adjektivdeklination}
	\autocites[vgl.][772]{woellstein2022}}
\begin{tabular}{l c c}
	\lsptoprule

	% \mr{2}{*}[-.5ex]{Kasus}
		& \textsc{sg}
		& \mr{2}{*}[-.5ex]{\textsc{pl}}
	\\

	\cmidrule(lr){2-2}

	%
		& \textsc{f}
		\\

	\midrule

	\textsc{nom}
		& \cellcolor{black!33}{-e}
		& \cellcolor{black!33}{-e}
		\\
	\textsc{acc}
		& \cellcolor{black!33}{-e}
		& \cellcolor{black!33}{-e}
		\\
	\lspbottomrule
\end{tabular}
\label{tab:nhd_stradj}
\end{table}}

Zur Übersicht wird in \tabref{tab:ahd_stradj}--\ref{tab:nhd_stradj} der
relevante Ausschnitt aus dem \isi{Paradigma} der oberdeutschen\il{Oberdeutsch}
starken \isi{Adjektivdeklination} des Mittelhochdeutschen\il{Mittelhochdeutsch}
im historischen\is{Diachronie} Kontext gezeigt. Da in der
mittelhochdeutschen\il{Mittelhochdeutsch} Periode nur das
Oberdeutsche\il{Oberdeutsch} eine Zweiteilung zwischen dem Maskulinum-Femininum
gegenüber dem Neutrum aufweist, findet die vorliegende Untersuchung darin sowie
durch die Ausweitung von \mbox{\norm{-e}} auf alle Genera im
Früh\-neu\-hoch\-deutschen\il{Frühneuhochdeutsch} ihre räumliche und zeitliche
Beschränkung\is{Beschränkung}
\autocites[vgl.][191--192]{reichmannwegera1993}[181--184]{ksw2}. Weil sich
sprachliche Veränderungen nicht plötzlich vollziehen, muss besondere
Aufmerksamkeit der Frage gewidmet werden, ob \norm{-e} im jeweils untersuchten
Text oder genereller am Ausstellungsort und seiner Umgebung schon regelmäßig
für alle Genera stehen kann oder noch auf das Maskulinum und das Femininum
beschränkt ist.

Insofern Diskrepanzen zwischen der Form einer kongruierenden Wortart und ihrem
doppelten Bezug durch diese \q{nicht-kanonische}
Kongruenzbeziehung\is{Kongruenzrelation!nicht-kanonische} entstehen, weist
\citet[144]{corbett2006} darauf hin, dass gerade diese einen Einblick in die
Funktionsweise und den Wandel von Kongruenz\-systemen erlauben. Der Reiz von
Gender Resolution als Untersuchungsgegenstand liegt vor allem darin, dass sie
zwar einen grundlegenden Mechanismus in Sprachen mit Genussystem darstellt. In
Hinsicht auf das Mittelhochdeutsche\il{Mittelhochdeutsch} hat sie aber bisher
eher wenig Beachtung gefunden.

Gerade die traditionsreichen Standard\-werke zur
mittelhochdeutschen\il{Mittelhochdeutsch} Grammatik widmen sich der Thematik
nur oberflächlich und sehr pauschal \autocites(dazu
\chapref{ch:forschungsueberblick})[siehe][384]{paul2007}[188]{dal2014}; der
Teilband der neuen \tit{Mittelhochdeutschen\il{Mittelhochdeutsch} Grammatik}
von \citeauthor*{ksw2} zur Syntax steht zum gegenwärtigen Zeitpunkt noch aus.
Darüber hinaus tritt der Quantor \norm{bėide} in \isi{Distanzstellung} zu
seinem Bezug auf, wie in \REF{ex:mhgbeidiu_2} dargestellt. Solche Floating
Quantifiers\is{Floating Quantifier} \autocite{sportiche1988} werden in der von
\citet[204]{corbett1979} formulierten \textit{Agreement Hierarchy} nicht
berücksichtigt (\sectref{sec:floatquant}, \sectref{sec:kongrhier}). Ihr
Kongruenzverhalten in Hinsicht auf Gender Resolution zu untersuchen, ist daher
ebenfalls von Interesse.

Eine ausführliche Beschäftigung mit Gender Resolution in den
historischen\is{Diachronie} Sprachstufen des Deutschen sowie ihrer Einbettung
in den historischen\is{Diachronie} Kontext bietet \citet{askedal1973}, der das
Auftreten des Neutrums vor allem beim Bezug auf Personen in
althochdeutschen\il{Althochdeutsch} und
mittelhochdeutschen\il{Mittelhochdeutsch} Quellen untersucht. Als Material für
den letzteren Zeitabschnitt dienen ihm kritische
Editionen\is{Editionsphilologie} des \tit{Parzival} Wolframs von
Eschenbach\ia{Wolfram von Eschenbach} und des \tit{Tristan} Gottfrieds von
Straßburg\ia{Gottfried von Straßburg} anhand der Edition\is{Editionsphilologie} von
\citet{lachmannhartl1952} sowie \citet{maroldschroeder1969}.%
%
	\footnote{Zur methodischen Problematik der Auswertung von kritischen
		Editionen\is{Editionsphilologie} als Quelle für linguistische
		Untersuchungen siehe \sectref{sec:materialcao}.}

\posscite{askedal1973} Belegsammlung ist mit knapp fünfzig Stellen zum hier
untersuchten \norm{bėide} allerdings nicht sehr groß. Ein Anliegen der
Belegauswertung in dieser Arbeit ist daher, unter Verwendung moderner Methoden
digital verfügbare, diplomatische Transkriptionen\is{Transkription} von
Originaltexten auszuwerten, um im Gegensatz zu \citeauthor{askedal1973}s Arbeit
eine möglichst große Nähe zur handschriftlichen Realität der
Textüberlieferung\is{Überlieferung} zu gewährleisten. Die digitale
Durchsuchbarkeit der hier verwendeten Quellen ermöglicht es außerdem, eine
größere Textmenge als \citeauthor{askedal1973} auszuwerten. Dies kommt der von
\citet[118]{fleischerschallert2011} geäußerten Frage entgegen, ob basierend auf
\citeauthor{askedal1973}s kleiner
\isi{Stichprobe} überhaupt Generalisierungen möglich seien. Darüber hinaus wird
aktu\-elle Forschung zu Gender Resolution auf das
Mittelhochdeutsche\il{Mittelhochdeutsch} angewendet, um auch die theoretische
Diskussion gegenüber \citet{askedal1973} zu modernisieren und Ergebnisse damit
weiter zu präzisieren.

Aufgrund der Beschränkungen\is{Beschränkung} der Materialauswahl ist es nicht
möglich, allen Kritikpunkten \posscite{askedal1973} an Skizzen des 19.\
Jahrhunderts und der ersten Hälfte des 20.\ Jahrhunderts zum hier behandelten
Thema gerecht zu werden. Seine Kritik besteht hauptsächlich darin, dass
\isi{Belebtheit} und die historische\is{Diachronie} Einbettung in
Sprachwandelprozesse häufig außer Acht gelassen worden seien. Während sich die
vorliegende Untersuchung konsequent dem ersten Kritikpunkt verschreibt, kommt
auch hier letzterer tendenziell zu kurz. Hervorzuheben ist aber, dass
komplementär zu \citeauthor{askedal1973}s Unter\-suchung anhand von
Vers\-texten durch die Herannahme von Urkundentexten\is{Urkunde} an dieser
Stelle eine größere Menge gebrauchsschriftlicher Prosatexte\is{Prosa} zur
Kongruenz von \norm{bėide} detailliert untersucht wird.

Darüber hinaus beschränkt sich diese Arbeit exemplarisch auf \norm{bėide},
während beispielsweise auch \norm{ƶwēne/ƶwō/ƶwėi} `zwei (\textsc{m/f/n})' und
\norm{drī/driu} `drei (\textsc{m+f/n})' eine Anzahl von Menschen oder Dingen
spezifizieren können. Bezug auf Gruppen und Mengen haben auch die adjektivisch
deklinierenden\is{Adjektivdeklination} Lemmata \norm{al} `alle' sowie die
Pluralformen von \norm{manic} `viel', \norm{ętelich} `(irgend)ein, irgendwelch'
oder \norm{sumelich} `irgendein, manch, mancherlei'. Gerade bei \norm{bėide}
erfolgt aber häufig die Nennung seiner zwei Referenten, während im untersuchten
Material \norm{al} oft auf einzelne Substantive\is{Substantiv} im Plural
bezogen ist, die im Bezug auf Kongruenz tendenziell unproblematisch sind
(\sectref{sec:beidequant}). Zu beachten ist auch die Häufigkeit\is{Frequenz}
der Lemmata selbst, die in \tabref{tab:freqquantcao} für das \CAO{} basierend
auf den Angaben des \WMU{} angegeben ist. Davon flektiert jeweils nur ein Teil
stark für den hier relevanten Nom./Akk.~Pl.

\begin{table}
\centering
\caption{Häufigkeit\is{Frequenz} verschiedener Lemmata mit Bezug auf Gruppen
	oder Mengen im \tit{Corpus der altdeutschen Originalurkunden}}
\begin{tabularx}{\linewidth}{l X r l @{\citereset}}
\lsptoprule

Lemma
	& Übersetzung
	& Häufigkeit
	& Quelle
	\\

\midrule

\norm{al}
	& `jeder(art); alle; ganz, komplett'
	& 17.500
	& \cite[46--49]{wmu1}
	\\

\norm{bėide}
	& `beide, jeder von zweien'
	& 2.050
	& \cite[166--168]{wmu1}
	\\

\norm{drī}
	& `drei'
	& 2.280
	& \cite[398--399]{wmu1}
	\\

\norm{ętelich}
	& `manche, einige, etliche (Pl.)'
	& 121
	& \cite[536]{wmu1}
	\\

\norm{manic}
	& `viele, mehrere (Pl.)'
	& 205
	& \cite[1180]{wmu2}
	\\

\norm{sumelich}
	& `einige, manche (Pl.)'
	& 25
	& \cite[1689--1690]{wmu2}
	\\

\norm{ƶwēne}
	& `zwei'
	& ca. 5.750
	& \cite[2543--2545]{wmu3}
	\\

\lspbottomrule
\end{tabularx}
\label{tab:freqquantcao}
\end{table}

\is{Gender Resolution|)}

\section{\norm{Bėide} auch konjunktional}
\label{sec:einlbeidekonj}
\is{Konjunktion|(}

Neben der Verwendung von \norm{bėide} als Quantor -- damit ist hier ein
\isi{Determinierer} ähnlich wie \fw{das} oder \fw{mein} gemeint, der die Anzahl
oder Menge des von ihm bestimmten Substantivs\is{Substantiv} spezifiziert --
kommt im Mittelhochdeutschen\il{Mittelhochdeutsch} \norm{bėide} auch noch im
Rahmen der Konstruktion \norm{bėide \dots\ unde} vor, die dem modernen
\norm{sowohl \dots\ als auch} entspricht. Die Konstruktion dient dazu, die
Existenz von zwei neben\-einander\-stehenden Optionen oder Sachverhalten zu
betonen. Sie ist im Neuhochdeutschen nach \citet[530, Fußnote
2]{walchhaeckel1988} vereinzelt noch bis ins 17., nach \citet[222]{dal2014} bis
ins 18.~Jahrhundert belegt. Die Variation zwischen \norm{bėide} und
\norm{bėidiu} auch in diesem syntaktischen Kontext zu analysieren, bildet den
zweiten Themenkomplex, der in dieser Arbeit behandelt wird.

Zwei Beispiele dazu aus dem Mittelhochdeutschen\il{Mittelhochdeutsch} werden in
\REF{ex:mhgbeideunde} gegeben, wo \norm{wīp} `Frau' mit \norm{man}
`Mann' kombiniert wird.%
%
	\footnote{Belegstellen in \KC{}-Handschriften werden angegeben als:
		\textit{<Sigle>: <Blattnummer><Blattseite recto/verso>(<Spalte a--c>),%
		<Zeile(n)>}. Falls sich die zitierten Verse einer Textstelle in der
		Edition von \citet{schroeder1895} zuordnen lassen, wird sie zusätzlich
		angegeben: \textit{V.~<Vers>; \cite[<Seite>]{schroeder1895}}.
		Parallelstellen werden mit \textit{vgl. (abweichend)} referenziert.
		Zitate, die nur ihrem Kontext nach Parallelstellen oder der Edition
		zugeordnet werden können, werden mit \textit{zu} gekennzeichnet. Die
		Siglen der Handschriften richten sich nach \citetitle{kcdigital}, siehe
		unter \citeurl{kcdigital}\nocite{kcdigital}.%
	}
%
Neben dem Wechsel zwischen je einer Form vom Typ \norm{-e} in
\REF{ex:mhgbeideunde_1} und \norm{-iu} in \REF{ex:mhgbeideunde_2} bei
ansonsten gleichem Wortlaut der \isi{Nominalgruppe} ist zu beachten, dass
\norm{wīp} ein sogenanntes \fw{\isi{Lexical Hybrid}} darstellt. Dies bedeutet,
dass das grammatische Genus (Neutrum) und das Geschlecht der bezeichneten
Person (weiblich) bei diesem Lexem auseinanderfallen (siehe
\sectref{sec:gendsex} sowie \cite[183--184]{corbett1991}).

\begin{exe}
\ex \label{ex:mhgbeideunde}
	\begin{xlist}
	\ex \label{ex:mhgbeideunde_1}
		\gll baide wip und man. \\
			beide Frau[\textsc{nom.pl.n}] und Mann[\textsc{nom.pl.m}] \\
	\sn \gll ſciden alle weíninde dann. \\
			schieden alle weinend dannen \\

	\ex \label{ex:mhgbeideunde_2}
		\gll Beidiv wip vnd man \\
			beide Frau[\textsc{nom.pl.n}] und Mann[\textsc{nom.pl.m}] \\
	\sn \gll Schieden weínende dan \\
			schieden weinend dannen \\
		\trans `Sowohl Frauen als auch Männer verabschiedeten sich (alle)
			weinend.'
			(%
				A1: 7ra,38--39;
				VB: 8va,5--6;
				vgl.~\KC: V.~1564--1565; \cite[110]{schroeder1895}%
			)
	\end{xlist}
\end{exe}

Unter Wiederverwendung des Quellenmaterials zu seiner Abhandlung zum Plural
Neutrum mit persönlichem Bezug \autocite{askedal1973} versucht
\citet{askedal1974} nachzuweisen, dass entgegen dem Konsens der Forschung
\autocites[vgl.~z.\,B.][433]{behaghel1923}[133]{dal2014}[626]{ksw2} noch im
Mittelhochdeutschen\il{Mittelhochdeutsch} der Zeit um 1200 das einleitende
\norm{bėide} regelmäßig flektiert wird. \citeauthor{askedal1974}s Behauptung
wurde bereits von \citet{gjelsten1980} widerlegt. Nichtsdestoweniger wird auch
dieser Aspekt angesprochen, da gerade die \KC{} vergleichsweise viele Beispiele
für diese Konstruktion enthält und besonders bei nominalen Konjunkten nicht
immer eindeutig\is{Ambiguität} ist, ob statt der korrelativen Konstruktion die
Appositionskonstruktion\is{Apposition} \norm{bėide, \emph{X} unde \emph{Y}}
vorliegt.

Daneben diskutieren \citet[627--628]{ksw2} die räumliche
Verteilung\is{Distribution!geografische} von \norm{bėide \dots\ unde},
inklusive des Vorkommens von \norm{bėide} in der \KC{}-Handschrift A1 (dort mit
V bezeichnet), insofern Auszüge daraus Teil ihres \isi{Korpus} sind. Die hier
zusätzlich ausgewerteten Materialien eignen sich als Vergleichsobjekt, zumal
nur Teile dieser einen \KC{}-Handschrift im Grammatikkorpus\is{Korpus} von
\citet{ksw3,ksw2} enthalten und Urkunden\is{Urkunde} des 13.~Jahrhunderts
aufgrund ihrer strengen Text\-auswahl\-kriterien unterrepräsentiert sind
\autocite[1309, 1311]{wegera2000}.

\is{Konjunktion|)}

\section{Forschungsfragen und Aufbau der Arbeit}

In \sectref{sec:einlgendres} und \sectref{sec:einlbeidekonj} wurden einige
Aspekte des in dieser Arbeit behandelten Themenkomplexes skizziert, die sich
auf drei Fragestellungen herunterbrechen lassen. Diese bilden als
Forschungsfragen die Grundlage für die weitere Diskussion:

\begin{enumerate}
	\item In welchen morphosyntaktischen Kontexten treten die Formen
		\norm{bėide} und \norm{bėidiu} auf?
	\item Wie häufig treten die einzelnen Formen in ihren verschiedenen
		Kontexten in den ausgewerteten Quellen auf und wo besteht Variation
		zwischen den Formen in ansonsten gleichen Kontexten vor allem beim
		kombinierten Bezug auf zwei Personen oder Dinge?
	\item Was sind die Bedingungen\is{Bedingung}, die das Auftreten der einen
		oder der anderen Form in diesem Fall motivieren oder begünstigen?
\end{enumerate}

Um diese Fragen zu erörtern, ist die vorliegende Untersuchung im weiteren
Verlauf in drei Teile gegliedert, deren Inhalt im Folgenden kurz umrissen wird.

Der erste Teil beschäftigt sich mit Grundlagen. Er enthält zunächst eine
Einführung in die für diese Studie relevanten theoretischen Grundlagen und die
verwendete Terminologie (\chapref{ch:theorie}). Zur Einordnung in den
bestehenden Forschungsdiskurs folgt ein kurzer Überblick zu bestehender
Forschung zum hier behandelten Themenkomplex
(\chapref{ch:forschungsueberblick}). Anschließend werden die zur Auswertung
verwendeten Materialien mit ihren geschichtlichen Hintergründen, Besonderheiten
und ihrem Aufbau kurz vorgestellt (\chapref{ch:materialien}). Der erste Teil
endet mit der Beschreibung der Vorgehensweise bei der Datengewinnung
(\chapref{ch:methoden}).

Der zweite Teil der Arbeit dreht sich um die Auswertung der gesammelten Belege.
Wie oben angesprochen, steht im Hintergrund die Frage, in welchen Texten und an
welchen Ausstellungs\-orten überhaupt ein grammatischer Unterschied zwischen
\norm{e}- und \norm{iu}-Formen der Flexion von \norm{bėide} vorliegt. Um eine
Grundlage für die weitere Diskussion zu schaffen, wird also zunächst dieser
Frage als Bindeglied zwischen der Methodik der Belegsammlung und der
Belegauswertung selbst nachgegangen (\chapref{ch:adjflex}). Für die beiden
Materialsammlungen, \CAO{} und \KC{}, ist je ein Kapitel vorgesehen
(\chapref{ch:caoanalyse} und \ref{ch:kcanalyse}), in dem die geografische
Verteilung\is{Distribution!geografische} der jeweils gesammelten Belege
umrissen und dann das Vorkommen von \norm{bėide} und\norm{bėidiu} in
Abhängigkeit von verschiedenen Faktoren untersucht wird: zunächst für
quantifizierendes \norm{bėide} nach den referenzierten
Personenmerkmalen\is{Personenmerkmal}, nach der Distanz in
Wortformen\is{Distanz!lineare} zwischen \norm{bėide} und seinem mittelbaren und
unmittelbaren Bezug sowie nach der Spanne\is{Distanz!syntaktische} der
syntaktischen \isi{Domäne} der Bezugspaare im Sinne der Konstituenz von Phrasen
und Sätzen. Im Anschluss daran steht jeweils die Diskussion der auftretenden
Formen von \norm{bėide} im Rahmen der Konstruktion \norm{bėide \dots\ unde}
`sowohl \dots\ als auch'. Neben der Diskussion der quantitativen Verteilung der
Belege werden Ausnahmen\is{Ausnahme} und sonstige Auffälligkeiten auch
qualitativ eingeordnet, indem erörtert wird, wie diese zustande kommen können.

Im dritten Teil werden die bei der Belegsammlung und -analyse beobachteten
Kongruenzphänomene diskutiert (\chapref{ch:diskussion}). Dabei kommt vor allem
der Erklärungsansatz für \isi{Gender Resolution} von
\citet[171--195]{wechslerzlatic2003} beziehungsweise \citet{wechsler2009} zum
Tragen. Überlegungen werden angestellt, warum insbesondere beim unmittelbaren
Bezug von \norm{bėide} auf ein Pronomen die größte Variation zwischen
\norm{bėide}- und \norm{bėidiu}-\allowbreak{}Formen beobachtet werden kann.
Dagegen scheint bei \norm{bėide} in Dis\-tanz\-stellung, wie zum Beispiel in
\fw{Sie promovieren \emph{beide} in Sprachgeschichte}, auffällig regelmäßig
semantische Kongruenz zu operieren. Auch die Genese von \norm{bėide} als
Konjunktion wird kurz im Rekurs auf die Theorie zur Grammatikalisierung nach
\citet{lehmann2015} diskutiert.

Abschließend werden die Untersuchungsergebnisse noch einmal zusammengefasst.
Darüber hinaus werden mögliche Anknüpfungsmöglichkeiten für eine zukünftige
Forschung im Rahmen der Kongruenzmorphologie des
Mittel\-hoch\-deutschen\il{Mittelhochdeutsch} in einem kurzen Ausblick
aufgezeigt (\chapref{ch:zusammenfassung}).
