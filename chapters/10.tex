\chapter{Zusammenfassung und Ausblick}
\label{ch:zusammenfassung}

Die starke \isi{Adjektivdeklination} im oberdeutschen\il{Oberdeutsch}
Sprachraum\is{Dialektgeografie} der mittelhochdeutschen\il{Mittelhochdeutsch}
Periode macht im Nom.\ und Akk.\ Pl.\ noch einen Unterschied zwischen
Maskulinum und Femininum gegenüber dem Neutrum \autocite[182]{ksw2}. Die
Wortform des hier exemplarisch untersuchten determinierenden\is{Determinierer}
Quantors \norm{bėide} `beide' lautet daher je nach Kongruenzkontext
\norm{bėide} (Nom./Akk.\ Pl.\ M./F.\ stark) oder \norm{bėidiu} (Nom./Akk.\ Pl.\
N.\ stark).

Gerade beim kombinierten belebten\is{Animata} Bezug auf Referenten mit
unterschiedlichem Genus beziehungsweise Sexus -- männlich und weiblich als
grammatikalisierte semantische Geschlechterkategorien der Animata\is{Animata}
-- ist aber in der Regel die formal neutrale Form zu finden. Diese Art von
\q{Auflösungsphänomen}, das bei konfligierenden Genus- oder Sexusmerkmalen von
kombinierten Personenmerkmalen auftritt, wird als \textit{Gender Resolution}
bezeichnet
\autocites{corbett1983}[264--306]{corbett1991}. Da jedoch auch Fälle zu
beobachten sind, in denen in diesem Kontext die formal maskulin-feminine Form
auftritt, wurde in dieser Arbeit der Frage nachgegangen, wie häufig die
jeweiligen Formen in verschiedenen syntaktischen Kontexten auftreten und welche
Parameter einen Einfluss auf das Erscheinen der einen oder anderen Form haben.
Darüber hinaus bestand die Frage, wie sich \norm{bėide} in der koordinierenden
Konstruktion \norm{bėide \dots\ unde} `sowohl \dots\ als auch' im Vergleich zu
quantifizierendem \norm{bėide} verhält.

Die älteren Grammatikwerke zum Mittelhochdeutschen\il{Mittelhochdeutsch}
\autocites[384]{paul2007}[187--189, 222]{dal2014}[258]{michels1979}%
[221]{mettke2000} machen zu beiden Aspekten nur sehr pauschale Aussagen.
\citet[621--628]{ksw2} gehen zwar auf die möglichen syntaktischen Kontexte des
Quantors \norm{bėide} und deren Häufigkeit in ihrem Korpus sowie auf die
geografische\is{Dialektgeografie} Distribution der Formen der Konjunktion
\norm{bėide} ein. Sie machen aber keine Angaben zur Variationsbreite zwischen
den Formen in den verschiedenen Kontexten sowie deren steuernde Faktoren;
letzterer Aspekt wäre für den derzeit ausstehenden Band zur Syntax zu erwarten.

Darüber hinaus stehen zwar zwei Studien zum hier untersuchten Themenkomplex von
\citet{askedal1973,askedal1974} zur Verfügung, doch dienen ihm für die
mittelhochdeutsche\il{Mittelhochdeutsch} Periode (ca.~1050--1350) textkritische
Editionen lediglich zweier prominenter Verstexte der Artusepik als Quelle.
Seine Schlussfolgerungen zum konjunktionalen Gebrauch von \norm{bėide}
\autocite{askedal1974} sind zudem fragwürdig und beschränken sich auf nominale
Konjunkte \autocite{gjelsten1980}.

Die Idee dieser Arbeit war daher, sowohl den Quantor als auch die Konjunktion
\norm{bėide} auf einer handschriftennahen und zumindest synchron breiteren
Basis als die Arbeiten von \citet{askedal1973,askedal1974} zu untersuchen. Als
Textgrundlage dienten zum einen das \citefield{cao1}[citetitle]{maintitle}
(\CAO{}; \nosh\cites{cao1,cao2,cao3,cao4,caor,cao5}) als Sammlung von
alltagsnahen Gebrauchstexten in Prosa vor allem des späten 13.~Jahrhunderts und
zum anderen die Haupthandschriften der \citetitle{kc} (\KC{};
vgl.~\cites{schroeder1895,nellmann1983}) als erzählerisch ausgestalteter
Sachtext in Versform, der hauptsächlich im 13.\ und 14.\ Jahrhundert breit
über\-liefert ist. Diese Texte haben den Vorteil, dass sie digital in
diplomatischer Transkription vorliegen. Dies vereinfacht die Belegsammlung
enorm, gerade auch, weil das hauptsächlich untersuchte Phänomen,
mittelhochdeutsch\il{Mittelhochdeutsch} \norm{bėide} `beide' in Bezug auf zwei
Referenten, nicht ausgesprochen häufig auftritt.

Für das \CAO{} beträgt die Frequenz von \norm{bėide} und \norm{bėidiu} zusammen
ca.~282 Vorkommen pro 1~Mio. Wortformen, für die \KC{} ca.~510 Vorkommen pro
1~Mio. Die Texte des \CAO{} und der \KC{} wurden mit Hilfe von regulären
Ausdrücken durchsucht und anschließend manuell annotiert \autocites[vgl.\
z.\,B.][33--37]{perkuhnetal2012}[zur Methode
vgl.][207--209]{beckerschallert2021}[155--158]{beckerschallert2022b}. Der
Umstand, dass die Urkunden häufig dem engeren Umkreis eines bestimmten
Ausstellungs\-orts zugeordnet werden können, erlaubt einen zusätzlichen Blick
auf die geografische\is{Dialektgeografie} Dimension der beobachteten Variation.% Dies ermöglicht prinzipiell einen schreibdialektalen Vergleich mit
% literarischen Handschriften, deren regionale Herkunft in der Regel nur
% indirekt erschlossen werden kann.

\section{Ergebnisse}

In einer vorbereitenden Untersuchung der zahlenmäßig größten Ausstellungs\-orte
und ihrer näheren Umgebung sowie zu den auszuwertenden Handschriften der
\KC{} anhand einiger im Material häufig belegter Adjektive\is{Adjektiv} wurde
sichergestellt, dass am jeweiligen Ort oder in der jeweiligen Handschrift
grundsätzlich ein Unterschied in der Flexion zwischen \norm{\mbox{-e}} und
\norm{-iu} vorhanden ist. Dies steht vor dem Hintergrund des Abbaus der
Genusdistinktion im Plural der \isi{Adjektivdeklination} zum
Frühneuhochdeutschen\il{Frühneuhochdeutsch} hin
\autocite[191--192]{reichmannwegera1993}, deren Wirken sich jedoch weder im
\CAO{} noch in den meisten \KC-Handschriften stark bemerkbar gemacht hat.
Ausnahmen\is{Ausnahme} bilden Straßburger Urkunden und die \KC{}-Handschrift Z.%
%
	\footnote{Die Siglen der \KC-Handschriften richten sich nach
		\citetitle{kcdigital}, siehe \citeurl{kcdigital}.}

Die \KC{}-Handschriften wurden jeweils nach Eindeutigkeit und Vorkommen der
Opposition im jeweiligen syntaktischen Kontext in verschiedene Gruppen
eingeteilt, um sie in ihrer Relevanz für die Untersuchung zu priorisieren. Zur
Auswertung des Quantors \norm{bėide} dienten hauptsächlich die Handschriften
B1, C1, K und VB; bei der Konjunktion \norm{bėide} kamen vor allem A1, B1 und
VB zum Zug. Gerade die Vorauer Handschrift (A1) als bedeutendste
\KC{}-Handschrift fand bei der Auswertung zum Quantor \norm{bėide} also keine
Verwendung, da dort für diesen Kontext nahezu ausschließlich maskuline
\isi{Controller} belegt sind. Dies stellte generell bei der Auswertung der
\KC{} eine Herausforderung dar, die sich auf die Thematik des Texts
zurückführen lässt.

Die beiden Auswertungsserien sind prinzipiell miteinander vergleichbar. Die aus
der \KC{} exzerpierten Belege, wenn auch geringer in der Anzahl, verhalten sich
nicht grundlegend anders als die aus dem \CAO{} gewonnenen. Allerdings wurden
für die \KC{} aufgrund der von \citet[89--90]{askedal1973} und
\citet[191]{gjelsten1980} geäußerten Vorbehalte solche Kontexte nicht gewertet,
in denen \norm{bėide} am Versende oder vor einem Wort steht, das mit einem
Vokal beginnt. Sie sprechen diesbezüglich von Neutralisierungs\-positionen. Für
das \CAO{} wurden Belege im letzteren Kontext ebenfalls zunächst unter
Vorbehalt betrachtet für den Fall, dass dieser auch auf Prosatexte zutrifft,
wie \citet[92]{askedal1973} zu bedenken gibt. Jedoch ließ sich für die
Prosatexte kein Unterschied zu Nicht-\allowbreak{}Hiatus-\allowbreak{}Belegen
feststellen. \citeauthor{askedal1973}s Einwand ist also zumindest für das
\CAO{} zurückzuweisen.

\subsection{\norm{Bėide} als Quantor: Animata}
\is{Animata|(}

Die Funktion von \norm{bėide} als Quantor betreffend stimmt die Mehrzahl der
gesammelten Belege zu belebten Referenten mit der bereits von
\citet[312]{grimm1890} und \citet[39--41]{behaghel1928} formulierten
Beobachtung überein, dass beim Bezug auf zwei Referenten mit gleichem Genus die
maskulin-feminine Form steht, während in Bezug auf Referenten mit
unterschiedlichem Genus das Neutrum auftritt. Dies gilt allerdings auch zum
Beispiel für den Bezug auf Pronomen der ersten und zweiten Person, die formal
keine overte Genuskategorie besitzen, diese Informationen aber in ihrer
referentiellen Semantik beinhalten. Generell gilt, dass beim kombinierten Bezug
auf belebte Referenten deren semantisches Geschlecht (Sexus) beziehungsweise
die damit assoziierten Resolutionsregeln im Großteil der Fälle die
Kongruenzform bestimmen.

Betreffend der Distribution der syntaktischen Kontexte ist mit \citet[624,
Abbildung P~179]{ksw2} übereinstimmend anzumerken, dass der unmittelbare Bezug
von \norm{bėide} auf zwei \isi{Controller} in beiden Auswertungsserien sehr
selten auftritt. In nahezu allen der wenigen belegten Fälle liegt in diesen
Kontexten Distanzstellung vor (\norm{\emph{N\tsub{i}} unde \emph{N\tsub{j}}
\dots\ bėide}), was die Wortabfolge betrifft. Der mittelbare Bezug auf zwei
\isi{Controller} über mindestens ein Pronomen macht den Großteil der Belege aus
(Kontaktstellung: \norm{\emph{PRO\tsub{i+j}} bėide}; Distanzstellung:
\norm{\emph{PRO\tsub{i+j}} \dots\ bėide}). Genus\-indifferentes \norm{si} `sie'
stellt gefolgt von \norm{wir} `wir' den häufigsten unmittelbaren
\isi{Controller} von \norm{bėide} dar.

Historisch\is{Diachronie} ist zwar im Nom./Akk.\ bei den Personalpronomen der
3.~Pers.\ Pl.\ mit der Genus\-opposition \norm{sie} `sie (\textsc{m+f})'
gegenüber \norm{siu} `sie (\textsc{n})' zu rechnen. Eine stichprobenhafte
Teilauswertung hat ergeben, dass in den hier untersuchten Materialien dieser
Unterschied in aller Regel aber schon nicht mehr nachvollziehbar ist, also
generell \norm{si} steht, landschaftlich auch eine Form vom Typ
\norm{sei},
\norm{seu},
\norm{sie} oder
\norm{siu}
\autocites[vgl.][213--214]{paul2007}[369, 390--397]{ksw2}[482--483]{wmu1}.
\posscite[99]{askedal1973} Behauptung, dass \norm{si bėide} eine
Konstruktion bildet, bei der das Neutrum immer nur an einem der beiden Glieder
markiert wird (\q{Monoflexion}), kann für das ausgewertete Material daher
nicht nachvollzogen werden.

Gerade in Kontaktstellung zu Personalpronomen wurde bei belebtem Bezug die
höchste Variation zwischen \norm{bėide} und \norm{bėidiu} beobachtet. Die
Begründung dafür wurde in der Annahme gesucht, dass für die Auflösung des
Bezugs auf \isi{Controller} mit gemischtem Geschlecht entweder formale
(\textsc{m}~∨~\textsc{f} $\Rightarrow$~\norm{-e}) oder semantische Kongruenz
(\SM{}~∩~\SF{} $\Rightarrow$~\norm{-iu}) in Frage kommt, falls \norm{si} und
\norm{bėide} eine syntaktische Phrase bilden. Das Pronomen gibt nämlich in
diesen Fällen selbst keine formalen Genus\-merk\-male (mehr) vor, die per
\isi{Concord} verfügbar wären. Etwa drei Viertel der Targets mit
gemischtgeschlechtlichem Bezug (belebt und unbelebt) im \CAO{} weisen in diesem
Kontext Gender Resolution auf. Die insgesamt bedeutend geringere Belegzahl in
der \KC{} ist für sich genommen nicht aussagekräftig, verhält sich aber konform
zu den \CAO-Belegstellen.

Bei eindeutiger Distanzstellung des Targets wurde für das \CAO{} beobachtet,
dass regel\-mäßig Gender Re\-solu\-tion vorliegt. Daraus kann geschlossen
werden, dass die Kongruenz im Sinne der syntaktischen Konstituentenstruktur in
diskontinuierlichen Kontexten über Koindizierung funktioniert. Unter der
Hypothese, dass das Neutrum generell mit Inanimata und, im Umkehrschluss, die
maskulin-feminine Form mit Animata assoziiert wird
\autocite[243--245]{askedal1973}, ist eine auffällige Zunahme von \norm{bėide}
mit belebter kombinierter Referenz bei steigendem
Wortformenabstand\is{Distanz!absolute} nicht zu beobachten. Ohnehin stellte
sich lineare Distanz\is{Distanz!absolute} nicht als ausschlaggebender Faktor
für die beob\-ach\-tete Variation heraus.

In wenigen Fällen tritt in beiden Auswertungs\-serien beim belebten, männlichen
Bezug die neutrale Form auf, ohne dass dafür eine systematische Motivation
gefunden werden konnte. \citet{wechslerzlatic2003} und \citet{wechsler2009}
modellieren Gender Resolution bei Animata mit einer Schnittmengenoperation
zwischen den Sexusmerkmalen der involvierten Referenten. Zwei männliche
Referenten bilden eine Schnittmenge, die regelmäßig auf formaler Ebene das
Maskulinum auslöst -- bis auf die genannten Ausnahmefälle\is{Ausnahme}.

\is{Animata|)}

\subsection{\norm{Bėide} als Quantor: Inanimata}

Bei unbelebtem gemischten Bezug erfolgt noch regelmäßiger als bei den
Animata\is{Animata} die Auflösung zum Neutrum. Allerdings tritt auch hier die
Form \norm{bėidiu} im \CAO{} mehrfach mit Bezug auf kombinierte maskuline
\isi{Controller} auf, und zwar etwas häufiger als die erwartete Form
\norm{bėide}. Für diese Kombination liegen aber insgesamt nur sechs Belege vor;
kombinierte feminine \isi{Controller} sind im Belegmaterial keine vorhanden.

In allen anderen Fällen verhalten sich die
mittelhochdeutschen\il{Mittelhochdeutsch} Belege wie von
\citet{wechslerzlatic2003} und \citet{wechsler2009} für das
Isländische\il{Isländisch} beschrieben. Das Neutrum stellt in den untersuchten
mittelhochdeutschen\il{Mittelhochdeutsch} Texten das \isi{Default} dar. Es tritt
immer dann ein, wenn keine Schnittmenge zwischen den jeweiligen Genusmerkmalen
der zu kombinierenden \isi{Controller} und der Menge der grammatikalisierten
semantischen Genera (G\tsub{s}) gebildet werden kann.

Der Ausnahmefall\is{Ausnahme} mit dem Neutrum als Kongruenzform bei
kombinierten Maskulina (sowohl belebten\is{Animata} als auch unbelebten) kann
mit diesem Ansatz nicht erklärt werden. Anzunehmen ist, dass die
Defaultform\is{Default} als solche übergeneralisiert auch auf ansonsten
unproblematische Kontexte mit kombinierter Referenz angewendet wird. Diese
Möglichkeit wird von \citet[302]{corbett1991} angeführt, allerdings im Rahmen
seiner Untersuchungen zum
Bosnisch-\allowbreak{}Kroatisch-\allowbreak{}Mazedonisch-\allowbreak{}Serbischen
(\ili{BKMS}) nur auf Inanimata bezogen
\autocites[vgl.~auch][190]{wechslerzlatic2003}[581]{wechsler2009}.

\subsection{\norm{Bėide} als Konjunktion}

Neben seiner Funktion als Quantor tritt `beide' im
Mittelhochdeutschen\il{Mittelhochdeutsch} auch als Teil der korrelativen
Konjunktion \norm{bėide \dots\ unde} `sowohl \dots\ als auch' auf. In beiden
Auswertungs\-serien, \CAO{} und \KC{}, treten als Konjunkte in dieser
Konstruktion verschiedene Phrasen\-typen und Wort\-arten auf: solche, die
selbst \isi{Controller} sind; solche, die Kongruenztargets darstellen; und
schließlich solche, die Personenmerkmale weder definieren noch widerspiegeln.

Die Tatsache, dass \norm{bėide} in den
mittelhochdeutschen\il{Mittelhochdeutsch} Belegen auch mit letzteren vorkommt,
wurde als Evidenz für die weit fortgeschrittene Grammatikalisierung der
Konstruktion angesehen. Der Verwendungskontext von \norm{bėide} `beide' ist in
diesem Zusammenhang nicht auf seine Bedeutung als Quantor von Nominalen
beschränkt, sondern wurde auf andere syntaktische Zusammenhänge ausgeweitet.
\norm{Bėide} dient hier als Fokuspartikel, die die parallele Gültigkeit der
zwei genannten Optionen betont, nicht mehr als pronominal verwendeter Quantor,
der sich kataphorisch auf seine Referenten bezieht und deren Zweizahl betont.

Bereits für das Spätalthochdeutsche\il{Althochdeutsch} lassen sich Belege
finden, die einen Übergang\is{Ambiguität} zwischen beiden Verwendungsarten
andeuten \autocite[vgl.\ die Beispiele in][627]{ksw2}. Entsprechend der
Grammatikalisierungstheorie nach \citet[146--150]{lehmann2015} geht der Wandel
von einem freien Morphem zu einem funktionalen typischerweise mit dem Verlust
von paradigmatischer Variabilität einher. Es überrascht daher nicht, dass
\norm{bėide} in diesem Kontext erstarrt erscheint, also keine klare
Abhängigkeit von den Personenmerkmalen der Konjunkte ausgemacht werden konnte,
falls solche vorhanden sind. Stattdessen deutete sich an, dass die
geografische\is{Dialektgeografie} Verteilung einen wichtigeren Faktor darstellt.
Während \norm{bėidiu} im ganzen oberdeutschen\il{Oberdeutsch} Sprachgebiet
auftritt, liegen Belege für \norm{bėide} hauptsächlich an dessen Rändern vor.
Die Ergebnisse stimmen mit den Angaben von \citet[627--628]{ksw2} überein.

\section{Ausblick}
\is{Desiderat|(}

Generell wäre in dieser Arbeit wünschenswert gewesen, für alle möglichen
Kombinationen von Genera in allen syntaktischen Kontexten mehrere Belege zur
Verfügung zu haben. Dass sich dieser Wunsch nicht erfüllt hat, ist eine
Konsequenz der grundlegenden Herausforderung im Umgang mit historischen Texten
entsprechend dem \citeauthor{labov1994}'schen Diktum, dass die historische
Sprachwissenschaft die Kunst sei, den besten Nutzen aus schlechten Daten zu
ziehen.%
%
	\footnote{\foreignblockcquote{english}[11]{labov1994}{Historical
		linguistics can \textelp{} be thought of as the art of making the best
		use of bad data}.%
	}
%
Die Überlieferung ist letztlich dem Zufall unterworfen und fehlerbehaftet.
Evidenz kann nur indirekt erhoben werden, da es keine Muttersprachlerinnen und
Muttersprachler gibt, die gezielt befragt werden können
\autocite[11]{labov1994}. Gerade digitale Transkriptionen erleichtern es aber
enorm, an Belege und Beispiele zu kommen, selbst wenn keine morphologische oder
sogar syntaktische Annotation\is{Annotation} vorliegt.
% %
% 	\footnote{Darüber hinaus zeigt zum Beispiel die Arbeit von
% 		\citet{farrisarora2021}, dass schon einfache dependenzgrammatische
% 		Annotationen hilfreich sein können, um gezielt Belege für bestimmte
% 		syntaktische Kontexte zu finden, besonders auch für anscheinend
% 		seltenere syntaktische Kontexte wie \norm{bėide} in Bezug auf zwei
% 		Feminina.%
% 	}
% %
Für sprachgeschichtliche Untersuchungen ist daher wünschenswert,
Transkriptionen digital verfügbar zu machen, wenn sie im Rahmen von
Editionsvorhaben ohnehin anfallen.

Bei der Arbeit an dieser Untersuchung wurde deutlich, wie wichtig eine
sorgfältige Material\-auswahl ist. Es gilt zu bedenken, welche möglichen
morphologischen Kontexte von dem (oder den) untersuchten Text(en) ihrem Inhalt
gemäß abgedeckt werden können. Die Urkunden des \CAO{} haben sich an dieser
Stelle als sehr geeignet erwiesen, da der Genusunterschied in der Pluralflexion
auf das Oberdeutsche\il{Oberdeutsch} begrenzt ist und das Urkundenkorpus gerade
diesen Raum\is{Dialektgeografie} dicht abdeckt. Darüber hinaus fungieren nicht
selten Paare aus Mann und Frau als Aussteller. Die Wortformen \norm{bėide} und
\norm{bėidiu} kommen häufig genug vor, dass eine größere Belegmenge mit Hilfe
von regulären Ausdrücken manuell durchsucht und annotiert werden konnte,
andererseits aber selten genug, dass eine möglichst exhaustive Auswertung in
dieser Form möglich war.

Die \KC{} stellte dagegen für die Belegsammlung eine Herausforderung
dar, weil aufgrund ihres Inhalts kaum Paare aus Mann und Frau gemeinsam
auftreten, obwohl einzelne Episoden explizit eine Protagonistin besitzen.
Möglicherweise wäre es aufschlussreicher gewesen, die verschiedenen Textzeugen
eines ähnlich breit überlieferten, erzählenden Texts zu untersuchen -- insofern
sie als digitale Transkriptionen verfügbar sind. Zum Beispiel bietet sich neben
einer Erweiterung von \posscite{askedal1973,askedal1974} Auswertung des
\tit{Tristan} und des \tit{Parzival} auf ihre einzelnen Textzeugen prinzipiell
auch der \tit{Iwein} Hartmanns von Aue an, dessen Überlieferung 33 derzeit
bekannte Textzeugen umfasst \autocites[vgl.][s.\,v.~\textit{Hartmann von Aue:
\tit{Iwein}}]{hsc}.%
%
	\footnote{Transkriptionen der einzelnen Textzeugen des \tit{Iwein} werden
		im Rahmen von \citetitle{iwdigital} verfügbar gemacht, siehe unter
		\citeurl{iwdigital}. Zum \tit{Parzival} siehe das Berner
		\tit{Parzival}-Projekt unter \citeurl{parzivalprojekt}.%
	}

Phänomene im Rahmen von Spaltungskonstruktionen wie zum Beispiel der
Bedeutungs\-unterschied zwischen der Kontaktstellung und der Distanzstellung
von Quantoren sind eine interessante Fragestellung in Hinsicht auf ihre
formal\-syntaktische Modellierung \autocite[siehe
z.\,B.][]{pittner1995,merchant1996,fanselowcavar2002,nolda2007,shen2019}. In
der vorliegenden Auswertung deutete sich an, dass im auch
Mittelhochdeutschen\il{Mittelhochdeutsch} ein grammatischer Unterschied
vorliegt, der sich in der anscheinend höheren Affinität zu semantischer
Kongruenz beim distanten Quantor äußert. Denkbar wäre eine Folgestudie, die
explizit \q{problematische} \isi{Controller}\is{Ausnahme} in diesem Kontext
betrachtet, zum Beispiel Lexical Hybrids wie \norm{wīp} `Frau (\NeutF)', und
sie mit \q{unproblematischen} Substantiven vergleicht. Bezüglich des
Bindungsverhaltens\is{Bindung} von Floating Quantifiers wäre genereller auch
der Vergleich mit Reflexivpronomina und prädikativen
Adjektiven\is{Adjektiv!prädikativ} interessant, da auch sie sich im Mittelfeld
befinden und mit Subjekten koindiziert sind. Weiterhin könnte es lohnend sein,
die Überlegungen von \citet{spector2009} zur Spaltung von Topik- und
Subjektfunktion in Konstruktionen mit Floating Quantifier näher zu untersuchen,
auch was die Modellierung des Objektbezugs durch den Quantor betrifft.

Darüber hinaus kam zur Sprache, dass sich im Plural der starken
\isi{Adjektivdeklination} zum Neuhochdeutschen\il{Neuhochdeutsch} hin das
Flexiv \norm{-e} auf alle Genera ausbreitet
\autocite[vgl.][191--192]{reichmannwegera1993}. Bereits \citet{askedal1973}
stellt dies in den Kontext der Numerusprofilierung. \citet{dammelgillmann2014}
behandeln diesbezüglich die \isi{Diachronie} des morphologischen Systems der
Substantive, allerdings sind auch Adjektive\is{Adjektiv!attributiv} als
Modifikatoren von Substantiven Teil des Nominalkomplexes. Zu untersuchen wäre
also, wie sich in ihrem Ansatz die Ausbreitung der maskulin-femininen
Flexionsendung in den größeren Kontext des Umbaus der Nominal\-flexion einfügt.

\is{Desiderat|)}

Vieles könnte im Rahmen dieser Untersuchung noch gesagt und präzisiert werden,
doch möchte ich an dieser Stelle einen vorläufigen Punkt setzen und mit Ishmael
schließen: \foreignblockquote{english}[{Herman Melville: Moby-Dick;
\cite[159]{melville:mobydick}}]{It was stated at the outset, that this system
would not be here, and at once, perfected. You cannot but plainly see that I
have kept my word.
% But I now leave my cetological System standing thus unfinished, even as the
% great Cathedral of Cologne was left, with the crane still standing upon the
% top of the uncompleted tower. For small erections may be finished by their
% first architects; grand ones, true ones, ever leave the copestone to
% posterity.
\textelp{}
God keep me from ever completing anything. This whole book is but a
draught---nay, but the draught of a draught}. % (Schluss von Kapitel 32)
