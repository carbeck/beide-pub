\addchap{\lsAbbreviationsTitle}

\section{Allgemeines}

\begin{description}[
	align=left,
	font=\normalfont,
	leftmargin=*,
	nosep,
	widest={BKMS},
]
\item[ahd.]			althochdeutsch
\item[BKMS]			Bosnisch-Kroatisch-Mazedonisch-Serbisch
\item[Bz.]			Bezirk (Österreich)
\item[Dépt.]		Département (Frankreich)
\item[HPSG]			Head-driven Phrase Structure Grammar
						\autocite{pollardsag1994}
\item[Kr.]			Kreis (Deutschland)
\item[Kt.]			Kanton (Schweiz)
\item[LFG]			Lexical-functional Grammar
						\autocites{kaplanbresnan1982}{bresnanetal2016}
\item[md.]			mitteldeutsch
\item[mhd.]			mittelhochdeutsch
\item[nd.]			niederdeutsch
\item[obd.]			oberdeutsch
\item[r]			recto (Vorderseite)
\item[t]			Spur (\fw{trace})
\item[v]			verso (Rückseite)
\item[XML]			Extended Markup Language
\item[\xhead{X}]	Phrasenkopf vom Typ \emph{X}
\item[\xbar{X}]		Projektion von \xhead{X}
\item[XP]			Konstituente vom Typ \emph{X}
\end{description}

%%%%%%%%%%%%%%%%%%%%%%%%%%%%%%%%%%%%%%%%%%%%%%%%%%%%%%%%%%%%%%%%%%%%%%%%%%%%%%%%

\section{Grammatische Annotation}

Die grammatische Annotation von Beispielen richtet sich nach den
\citetitle{lgr} \autocite{lgr}. Darüber hinaus werden die folgenden Abkürzungen
verwendet:\\

\begin{description}[
	align=left,
	font=\normalfont\scshape,
	leftmargin=*,
	nosep,
	widest={\textsc{prontype}},
]
\item[*\dots]			inakzeptabel, unbelegt
\item[\tsup{?}\dots]	zweifelhaft
\item[$\uparrow$]		darüberliegend
\item[$\downarrow$]		darunterliegend
\item[$\in$]			Element von
\item[$\cap$]			Schnittmenge
\item[$\lor$]			oder
\item[\req]				beschränkt auf
\item[\SM]				männliches Geschlecht
\item[\SF]				weibliches Geschlecht
\item[\SMF]				kombiniert männliches und weibliches Geschlecht
\item[\SX]				unbekanntes Geschlecht
\item[\SA]				unspezifisch
\item[\SI]				unbelebt

\item[adj]		Adjunkt
\item[anim]		Belebtheit
\item[case]		Kasus
\item[cl]		Nominalklasse
\item[concord]	grammatischer Konkord
\item[conj]		Konjunktion
\item[deg]		Grad
\item[df]		Diskursfunktion
\item[gend]		Genus
\item[gf]		grammatische Funktion
\item[hon]		Höflichkeitsform
\item[index]	Index der anaphorischen Referenz
\item[num]		Numerus
\item[pers]		Person
\item[poss]		Possessor
\item[preconj]	Präkonjunktion
\item[pred]		Prädikator
\item[prontype]	Pronomentyp
\item[quant]	Quantor
\item[sex]		Sexus
\item[spec]		Spezifikator
\item[st]		stark
\item[subt]		Subjekt
\item[tense]	Tempus
\item[wk]		schwach
\end{description}

%%%%%%%%%%%%%%%%%%%%%%%%%%%%%%%%%%%%%%%%%%%%%%%%%%%%%%%%%%%%%%%%%%%%%%%%%%%%%%%%

\section{Forschungsliteratur}

\begin{description}[
	align=left,
	font=\itshape,
	leftmargin=*,
	nosep,
	widest={CAO 1--5, R},
]
\item[CAO 1--5, R] \tit{Corpus der altdeutschen Originalurkunden bis zum Jahr
						1300}, Bd.~1--5, Regesten
						\nosh\autocite{cao1,cao2,cao3,cao4,cao5,caor}
\item[DRW~7]		\tit{Deutsches Rechtswörterbuch}, Bd.~7 \nosh\autocite{drw7}
\item[HSC]			\tit{Handschriftencensus} \nosh\autocite{hsc}
\item[KC]			\tit{Kaiserchronik}
					\autocites{schroeder1895}%
						{nellmann1983}%
						% [s.\,v.~\tit{Kaiserchronik}]{hsc}
\item[ReA]			\citetitle{ddd} \autocite{ddd}
\item[ReM]			\citetitle{rem} \autocite{rem}
\item[WMU 1--3] \tit{Wörterbuch der mittelhochdeutschen Urkundensprache auf der
					Grundlage des Corpus der altdeutschen Originalurkunden bis
					zum Jahr 1300}, Bd. 1--3 \nosh\autocite{wmu1,wmu2,wmu3}
\item[ZfdA]			Zeitschrift für deutsches Altertum und deutsche Literatur
\end{description}

%%%%%%%%%%%%%%%%%%%%%%%%%%%%%%%%%%%%%%%%%%%%%%%%%%%%%%%%%%%%%%%%%%%%%%%%%%%%%%%%

\section{Handschriften der \tit{Kaiserchronik}}
\label{sec:hssverzkc}

Die folgenden Handschriften und Fragmente der \tit{Kaiserchronik} (\KC) werden
mit ihren Siglen im Text genannt. Weiterführende Informationen zur Handschrift
sowie Links zu Digitalisaten können dem \tit{HSC} unter der jeweiligen
Handschriften-ID entnommen werden.%
% %
% 	\footnote{Siehe z.\,B.~\url{https://handschriftencensus.de/1432} für A1.}
% %
\\

\begin{description}[
	align=left,
	font=\normalfont,
	leftmargin=*,
	nosep,
	widest={a14},
]
\item[A1]	Vorau, Archiv des Augustiner-Chorherrenstifts, Ms 276
				\autocite[1432]{hsc}
\item[a11]	Nürnberg, Germanisches Nationalmuseum, Hs. 22067
				\autocite[1189]{hsc}
\item[a14]	Straßburg, Bibliothèque nationale et universitaire, ms. 2215
				\autocite[1828]{hsc}
\item[B1]	Wien, Österreichische Nationalbibliothek, Cod. 2779
				\autocite[2693]{hsc}
\item[b1]	Basel, Universitätsbibliothek, Cod. N I 3, Nr. 89
				\autocite[1158]{hsc}
\item[C1]	Wien, Österreichische Nationalbibliothek, Cod. 2685
				\autocite[2013]{hsc}
\item[H]	Heidelberg, Universitätsbibliothek, Cod. pal. germ. 361
				\autocite[1181]{hsc}
\item[K]	Karlsruhe, Badische Landesbibliothek, Cod. Aug. 52
				\autocite[8470]{hsc}
\item[M]	München, Bayerische Staatsbibliothek, Cgm 37
				\autocite[2119]{hsc}
\item[P]	Prag, Národní knihovna České republiky, Cod. XIII.G.43
				\autocite[1168]{hsc}
\item[T]	München, Bayerische Staatsbibliothek, Cgm 965
				\autocite[8472]{hsc}
\item[VB]	Wien, Österreichische Nationalbibliothek, Cod. 2693
				\autocite[1215]{hsc}
\item[VC]	Wien, Österreichische Nationalbibliothek, Cod. 12487
				\autocite[3394]{hsc}
\item[W]	Wolfenbüttel, Herzog August Bibliothek, Cod. Guelf. 15.2 Aug. 2º
				\autocite[6668]{hsc}
\item[Z]	Leutkirch, Fürstliche Waldburg zu Zeil und Trauch\-burg\-sches
				Gesamt\-archiv, ZAMs 30
				\autocite[8471]{hsc}
\end{description}
