\addchap{\lsAbbreviationsTitle}

\section*{Allgemeines}

\begin{description}[
	align=left,
	font=\normalfont,
	leftmargin=*,
	noitemsep,
	widest={HPSG},
]
\item[Bl.]			Blatt
\item[Bz.]			Bezirk (Österreich)
\item[Dépt.]		Département (Frankreich)
\item[HPSG]			Head-driven Phrase Structure Grammar
						% \autocite{pollardsag1994}
\item[Kl.]			Kloster
\item[Kr.]			Kreis (Deutschland)
\item[Kt.]			Kanton (Schweiz)
\item[LFG]			Lexical-functional Grammar
						% \autocites{kaplanbresnan1982}{bresnanetal2016}
\item[r]			recto (Vorderseite)
\item[v]			verso (Rückseite)
\item[WGS]			World Geodetic System
\item[XML]			Extended Markup Language
\end{description}


%%%%%%%%%%%%%%%%%%%%%%%%%%%%%%%%%%%%%%%%%%%%%%%%%%%%%%%%%%%%%%%%%%%%%%%%%%%%%%%%

\section*{Sprachvarietäten}

\begin{description}[
	align=left,
	font=\normalfont,
	leftmargin=*,
	noitemsep,
	widest={BKMS},
]
\item[ahd.]			althochdeutsch
\item[alem.]		alemannisch
\item[BKMS]			Bosnisch-Kroatisch-Mazedonisch-Serbisch
\item[idg.]			indogermanisch
% \item[md.]			mitteldeutsch
\item[mhd.]			mittelhochdeutsch
% \item[nd.]			niederdeutsch
\item[nhd.]			neuhochdeutsch
\item[obd.]			oberdeutsch
\item[oobd.]		ostoberdeutsch
\item[PGmc]			Proto-Germanic
\item[PIE]			Proto-Indo-European
\end{description}

%%%%%%%%%%%%%%%%%%%%%%%%%%%%%%%%%%%%%%%%%%%%%%%%%%%%%%%%%%%%%%%%%%%%%%%%%%%%%%%%


\section*{Grammatische Annotation}\largerpage

Die grammatische Annotation von Beispielen richtet sich nach den
\citetitle{lgr} \autocite{lgr}. Darüber hinaus werden die folgenden Abkürzungen
verwendet:

\begin{description}[
	align=left,
	font=\normalfont\scshape,
	leftmargin=*,
	noitemsep,
	widest={\textsc{prontype}},
]
\item[*\dots]			inakzeptabel, unbelegt
\item[\tsup{?}\dots]	zweifelhaft
\item[$\uparrow$, $\downarrow$]	darüber-, darunterliegend
\item[$\in$]			Element von
\item[$\cap$]           Schnittmenge
\item[$\lor$]			oder
\item[\req]				beschränkt auf
\item[\SM, \SF, \SMF]	männliches, weibliches, gemischtes Geschlecht
\item[\SX]				unbekanntes Geschlecht
\item[\SA]				unspezifisch
\item[\SI]				unbelebt

\item[adj]		Adjunkt
\item[anim]		Belebtheit
\item[case]		Kasus
\item[cl]		Nominalklasse
\item[concord]	grammatischer Concord
\item[conj]		Konjunktion
\item[deg]		Steigerungsgrad
\item[df]		Diskursfunktion
\item[gend]		Genus
\item[gf]		grammatische Funktion
\item[hon]		Höflichkeitsform
\item[index]	Index der anaphorischen Referenz
\item[lsk]		linke Satzklammer
\item[mf]		Mittelfeld
\item[num]		Numerus
\item[pers]		Person
\item[poss]		Possessor
\item[preconj]	Präkonjunktion
\item[pred]		Prädikator
\item[prontype]	Pronomentyp
\item[quant]	Quantor
\item[rsk]		rechte Satzklammer
\item[sex]		Sexus
\item[spec]		Spezifikator
\item[st]		stark
\item[subj]		Subjekt
\item[\normalfont t]	Spur (\fw{trace})
\item[tense]	Tempus
\item[vf]		Vorfeld
\item[wk]		schwach

\item[\normalfont \xhead{X}, \xbar{X}, XP]
	Kopf, Projektion und Phrase vom Typ X
\end{description}

%%%%%%%%%%%%%%%%%%%%%%%%%%%%%%%%%%%%%%%%%%%%%%%%%%%%%%%%%%%%%%%%%%%%%%%%%%%%%%%%

\section*{Handschriften der \tit{Kaiserchronik}}
\label{sec:hssverzkc}

Die folgenden Handschriften und Fragmente der \tit{Kaiserchronik} werden mit
ihren Siglen im Text genannt. Diese richten sich nach \citetitle{kcdigital}
(\cite{kcdigital}; \citeurl{kcdigital}).

\begin{description}[
	align=left,
	font=\normalfont,
	leftmargin=*,
	noitemsep,
	widest={a14},
]
\item[A1]	Vorau, Archiv des Augustiner-Chorherrenstifts, Ms~276
				% \autocite[1432]{hsc}
\item[a11]	Nürnberg, Germanisches Nationalmuseum, Hs.~22067
				% \autocite[1189]{hsc}
\item[a14]	Straßburg, Bibliothèque nationale et universitaire, ms.~2215
				% \autocite[1828]{hsc}
\item[B1]	Wien, Österreichische Nationalbibliothek, Cod.~2779
				% \autocite[2693]{hsc}
\item[b1]	Basel, Universitätsbibliothek, Cod. N I 3, Nr.~89
				% \autocite[1158]{hsc}
\item[C1]	Wien, Österreichische Nationalbibliothek, Cod.~2685
				% \autocite[2013]{hsc}
\item[H]	Heidelberg, Universitätsbibliothek, Cod.~Pal.~germ.~361
				% \autocite[1181]{hsc}
\item[K]	Karlsruhe, Badische Landesbibliothek, Cod.~Aug.~52
				% \autocite[8470]{hsc}
\item[M]	München, Bayerische Staatsbibliothek, Cgm~37
				% \autocite[2119]{hsc}
\item[P]	Prag, Národní knihovna České republiky, Cod.~XIII~G~43
				% \autocite[1168]{hsc}
\item[T]	München, Bayerische Staatsbibliothek, Cgm~965
				% \autocite[8472]{hsc}
\item[VB]	Wien, Österreichische Nationalbibliothek, Cod.~2693
				% \autocite[1215]{hsc}
\item[VC]	Wien, Österreichische Nationalbibliothek, Cod.~12487
				% \autocite[3394]{hsc}
\item[W]	Wolfenbüttel, Herzog August Bibliothek, Cod.~Guelf.~15.2~Aug.~2º
				% \autocite[6668]{hsc}
\item[Z]	Leutkirch, Fürstliche Waldburg zu Zeil und Trauch\-burg\-sches
				Gesamt\-archiv, ZAMs~30
				% \autocite[8471]{hsc}
\end{description}

%%%%%%%%%%%%%%%%%%%%%%%%%%%%%%%%%%%%%%%%%%%%%%%%%%%%%%%%%%%%%%%%%%%%%%%%%%%%%%%%

\section*{Forschungsliteratur}

\begin{description}[
	align=left,
	font=\itshape,
	leftmargin=*,
	noitemsep,
	widest={CAO 1--5, R},
]
\item[CAO 1--5, R]	\tit{Corpus der altdeutschen Originalurkunden bis zum Jahr
						1300}, Bd.~1--5, Regesten
						\nosh\autocite{cao1,cao2,cao3,cao4,cao5,caor}
\item[DRW~7]		\tit{Deutsches Rechtswörterbuch}, Bd.~7 \nosh\autocite{drw7}
\item[HSC]			\tit{Handschriftencensus} \nosh\autocite{hsc}
\item[KC]			\tit{Kaiserchronik}
						\autocites{schroeder1895}%
							{kcdigital}%
							% [s.\,v.~\tit{Kaiserchronik}]{hsc}
\item[ReA]			\citetitle*{ddd} \autocite{ddd}
\item[ReM]			\citetitle*{rem} \autocite{rem}
\item[WMU 1--3]     \tit{Wörterbuch der mittelhochdeutschen Urkundensprache auf
						der Grundlage des Corpus der altdeutschen
						Originalurkunden bis zum Jahr 1300}, Bd. 1--3
						\nosh\autocite{wmu1,wmu2,wmu3}
\end{description}
