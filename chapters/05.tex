\chapter{Methodik der Datenerhebung}
\label{ch:methoden}
\is{Stichprobe|(}

\section%
	{Datenerhebung aus dem \tit{Corpus der altdeutschen Originalurkunden}}
\label{sec:miningcao}

Die Daten für die Teilauswertung zum \tit{Corpus der altdeutschen
Originalurkunden} (\CAO) wurden aus der in
\textcites[207]{beckerschallert2021}[155--158]{beckerschallert2022b}
beschriebenen \isi{Datenbank} exzerpiert. Grundlage dafür bildet die
digitalisierte Fassung des \CAO{}
\autocites{cao-online}[vgl.~dazu][]{gniffkerapp2005} inklusive der von
\citeauthor{beckerschallert2022b} vorgenommenen Zuordnung\is{Annotation} von
Ortskoordinaten. Das \CAO{} umfasst insgesamt etwa zwei Millionen Wortformen
inklusive Mehrfachausfertigungen und nicht-deutschsprachiger
Urkunden\is{Urkunde}. Pro Urkunde liegen damit durchschnittlich 453 Wortformen
vor, die \isi{Standardabweichung} beträgt 500 Wortformen.

Wie in \sectref{sec:materialcao} bezüglich der Text\-treue des \CAO{}
beschrieben, gehe ich davon aus, dass die in der Edition gebotenen
Transkriptionen\is{Transkription} für die Zwecke dieser Untersuchung
hinreichend fehlerfrei sind. Bei der Belegsammlung wurden nur solche
Urkunden\is{Urkunde} berücksichtigt, die eine eindeutige Jahresangabe besitzen
und die laut der Zuordung im Schreibortverzeichnis von \citet{cao-online} einem
einzigen Ausstellungsort zugeordnet sind. Bei der näheren Beschäftigung mit dem
Material hat sich gezeigt, dass der Zeitfaktor weniger ausschlaggebend ist, da
ohnehin die meisten Textstücke im \CAO{} aus der Zeit zwischen 1280 und 1299
stammen (vgl.~\tabref{tab:urkstat}).

Bezüglich des Orts\is{Distribution!geografische} sollen aber nur
Ausstellungs\-kontexte mit möglichst kleinräumigem Bezug erfasst werden, um
Urkundentexte\is{Urkunde} mit überregionaler Geltung und gegebenenfalls damit
einhergehender Vermeidung von Regionalismen nach Möglichkeit auszuschließen.
Implizit lehnen sich diese Kriterien an die in \citet[41--42]{ganslmayer2012}
diskutierten an, wobei der Faktor \q{Text} in Bezug auf
Parallelausfertigungen\is{Paralleltext} und zeitferne Abschriften ignoriert
wurde. Wenn durch die in \citet[155--158]{beckerschallert2022b} geschilderten
Ausschlusskriterien lediglich etwa 50\,\% des Gesamttextbestands zur Verfügung
stehen, dürften die 8,9\,\% der deutschsprachigen Urkunden, die
\citeauthor{ganslmayer2012} aufgrund des Faktors \q{Text} ausschließt, kaum ins
Gewicht fallen (neben 4.289 deutschsprachigen Urkunden weist
\cite[41]{ganslmayer2012} insgesamt 35 lateinische\il{Lateinisch} und 293
\il{Mittelniederländisch}mittelniederländische Urkunden aus). Des Weiteren
wurde auf eine Bestimmung des Schreibdialekts\is{Schreibdialekt} jeder
einzelnen ausgewerteten Urkunde im Abgleich mit ihrem angegebenen
Ausstellungsort verzichtet.

Die Suche in der Datenbank geschah mit Hilfe eines regulären
Ausdrucks\is{regulärer Ausdruck} \autocite[dazu
z.\,B.][33--37]{perkuhnetal2012}, der auf Basis der Liste der im
\CAO{} belegten Formen des Lemmas \norm{bėide} formuliert wurde \autocites(mit
allen Deklinationsformen insgesamt ca.~2.050 Belege)[vgl.][166--168]{wmu1}.
\tabref{tab:beidespelcao} listet die belegten grafischen Varianten pro Position
im Wort entsprechend diesem Suchausdruck tabellarisch auf.%
%
	\footnote{Die einzige im \tit{\WMU{}} verzeichnete Form mit
		\lit{-t-} ist \lit{beyte} \autocite[166]{wmu1}: \lit{durch beyte deſ
		vor~ge\-nan\-ten Hugeſ von Luzelſteyn vn̄ Henriches von Fleckenſteyn}
		`durch beide, des vorgenannten Hugo von Lützelstein und Heinrichs von
		Fleckenstein' \autocites(Nr.~N~674, Straßburg, 1294)[484,18]{cao5}.
		Diese Urkunde war nicht Teil der Auswertung. Formen wie
		\lit{bete} oder \lit{pet} stehen ansonsten für \norm{bęte} `Bitte,
		Gebet' oder stellen Formen von \norm{biten} `bitten' dar.}
%
Bei der Suche nach Zeichenketten wurden die Belegstellen nicht automatisch nach
dem syntaktischen Funktionskontext des jeweiligen Belegs
(determinierender\is{Determinierer} Quantor, \isi{Konjunktion}) geschieden.
Diese Zuordnung\is{Annotation} geschah manuell in einem zweiten Schritt.

\begin{table}
\centering
\caption{Im \WMU{} belegte Schreibweisen von mittelhochdeutsch \norm{bėide/-iu}
	pro Position im Wort}
\begin{tabular}{l l l l}
\lsptoprule

\mc{3}{c}{Stamm}
	& \mc{1}{c}{Flexion}
	\\

\cmidrule(r){1-3}
\cmidrule(l){4-4}

\begin{minipage}{1em}
	b\\
	p
\end{minipage}
	& \begin{minipage}{.25\linewidth}
		ai,
		ay,
		aî,
		aͤ,
		aͥ,
		ee,
		ei,
		ey,
		eî,
		eͤ,
		ie,
		iæ,
		âi,
		æi,
		êi,
		e,
		æ,
		é,
		ê
	\end{minipage}
	& \begin{minipage}{1em}
			d\\
			(t)
	\end{minipage}
	& \begin{minipage}{.25\linewidth}
			eiw,
			iuͦ,
			ivͤ,
			iͤv,
			eu,
			ev,
			ew,
			iu,
			iv,
			iû,
			iͮ,
			uͥ,
			vͥ,
			îv,
			e,
			i,
			j,
			u,
			v,
			ú,
			û,
			Ø
	\end{minipage}
	\\
\lspbottomrule
\end{tabular}
\label{tab:beidespelcao}
\end{table}

\phantomsection
\label{phsec:caohiatus}

Insgesamt wurden 401 Belege zu \norm{bėide}-Targets\is{Target} erfasst, von
denen 244 auf Kontexte mit Quantor entfielen, 157 auf Kontexte mit Konjunktion;
55 Belegstellen wurden aus verschiedenen Gründen ausgeschlossen
(vgl.~\tabref{tab:ausgewcao} und Anhang~\ref{sec:urkliste}). Von den 4.289 im
\CAO{} enthaltenen Urkunden\is{Urkunde} und 1.021 zuordenbaren
Ausstellungsorten sind in der Stichprobe 291 Urkunden aus 114 Orten
vertreten\is{Distribution!geografische}, von denen 256 Urkunden aus 102 Orten
für die Auswertung verwendbar waren.

\begin{table}
\centering
\caption{Zahl der gesammelten und ausgewerteten Belegstellen}
\begin{tabular}{l r r r}
\lsptoprule

%
	& Gesammelt
	& Ausgewertet
	& in \%
	\\

\midrule

% Zahlen retrospektiv überprüft am 01.08.23 an
% data/2022-05-19_beide_cao_controller-target.ods

Quantor
	& 244 % gesamt
	& 219 % ausgewertet
	& 89,8 % in %
	\\

Konjunktion
	& 157 % gesamt
	& 127 % ausgewertet
	& 80,9 % in %
	\\

\midrule

Summe
	& 401 % gesamt
	& 346 % ausgewertet
	& 86,3 % in %
	\\

\lspbottomrule
\end{tabular}
\label{tab:ausgewcao}
\end{table}

Zu allen Kongruenztargets\is{Target} wurden die \isi{Controller} erfasst -- im
Fall von Targets in unmittelbarer Abhängigkeit von pronominalen Controllern
diese sowie die letzte Nennung der Referenten, auf die sie
verweisen\is{Anapher} (\q{\isi{Erstcontroller}}). Alle Controller und Targets
wurden anschließend nach ihren formalen\is{Merkmale!grammatische} und
semantischen Personenmerkmalen\is{Personenmerkmal}\is{Merkmale!semantische}
annotiert. Außerdem wurde jeweils die Wortart festgehalten sowie die
Flexionsform des Targets mit deren Flexionstyp (\norm{-e}, -Ø,
\norm{-iu}) und ob gegebenenfalls Schwa-\isi{Apokope} vorliegt. Auch der
lineare Abstand\is{Distanz!lineare} der betreffenden aufeinander bezogenen
Wortformen und die \isi{Domäne} der Kongruenzbeziehung\is{Kongruenzrelation}
(gleiche Phrase, gleicher Teilsatz, anderer Satz) wurden registriert.

Targets\is{Target}, bei denen in Urkunden vom gleichen Ort nach kursorischer
Durchsicht kein Unterschied in der starken \isi{Adjektivdeklination} zwischen
\norm{-e} und \norm{-iu} im Nom./Akk.~Pl.\ ausgemacht werden konnte, wurden
ausgesondert (vgl.~\sectref{sec:adjdeclcao}). Targets, bei denen der Vokal des
Flexionssuffixes im \isi{Hiatus} steht und der daher zumindest theoretisch
elidierbar ist \autocites[vgl.][90--91]{askedal1973}[191]{gjelsten1980} wurden
gesondert markiert, wenn auch die Urkundentexte nicht an ein metrisches Schema
gebunden sind.

%%%%%%%%%%%%%%%%%%%%%%%%%%%%%%%%%%%%%%%%%%%%%%%%%%%%%%%%%%%%%%%%%%%%%%%%%%%%%%%

\section{Datenerhebung aus der \tit{Kaiserchronik}}

Die Texte der \tit{Kaiserchronik} (\KC) liegen dank \citetitle{kcdigital}
\autocite{kcdigital} im XML-Format vor. Dieses kann problemlos in Klartext
zurückgewandelt werden, indem Annotationstags\is{Annotation} verworfen oder
serialisiert werden.%
%
	\footnote{Ich danke Magnus Breder Birkenes (Oslo), der diesen
	Arbeitsschritt bereits vollzogen und mir die Klartexte sämtlicher
	Textzeugen zur Verfügung gestellt hat.}
%
Dies ermöglicht wie auch beim \CAO{}-Material eine Textsuche mit Hilfe von
regulären Ausdrücken\is{regulärer Ausdruck}. Die Vorgehensweise gestaltet sich
dennoch leicht anders, da für die \KC{} kein lemmatisiertes Verzeichnis von
grafischen Varianten wie mit dem \WMU{} für die Urkunden\is{Urkunde} vorliegt,
sodass kein maßgeschneiderter Suchausdruck aufgrund bereits indizierter
Schreibvarianten destilliert werden kann.

Stattdessen ist es notwendig, sich an die vertretenen Schreibweisen durch
begründete Annahmen heranzutasten. Aus der Arbeit am \CAO{} ist zu schließen,
dass das Konsonantengerüst des Wortstamms regelmäßig die Form \textit{b/p--d}
hat; auch die Lemmaliste zum \tit{Mittelhochdeutschen Wörterbuch}
\autocite[s.\,v.~\fw{beide}]{mwb1} enthält diesbezüglich keine anderweitige
Evidenz.%
%
	\footnote{Siehe
		\url{http://www.mhdwb-online.de/konkordanz.php?lid=13638000&seite=1}%
		% (28.12.2021)
		.}
%
Die grafische Variation unter den 2.559 Belegen, die zum gegenwärtigen
Zeitpunkt dort aufgelistet werden, bemisst sich auf \lit{b, p} -- \lit{ai, ay,
ei, ey, eí, eǐ, éi, êi, é, ê} -- \lit{d} -- \lit{iͤv, eu, ev, iu, iv, uͦ, íu,
iͮ, a, e, i, u, è, ú, û,} Ø. In den ausgewerteten Handschriften der \KC{} gibt
es keine Anzeichen für Schreibvarianten von \norm{bėide} mit \lit{-t-}, auch
nicht in Kontexten mit \isi{Apokope}. Formen ähnlich \norm{bėide} mit \lit{-t-}
oder \lit{-t} sind den Lemmata \norm{bęte} `Bitte, Gebet', \norm{bęten/biten}
`bitten, beten', \norm{bėtte} `Bett', \norm{bėiten} `zögern, warten' und
\norm{bieten} `(an)bieten' zuzuordnen.

Für die gegenwärtige Auswertung kommen nur die Flexionsformen vom Typ \norm{-e}
und \norm{-iu} in Frage, daher wurden bei der Belegsammlung Formen, bei denen
in der Zeichenkette nach dem Wortstamm \norm{m}, \norm{n}, \norm{r}, \norm{s}
oder \norm{z} enthalten sind, nicht berücksichtigt;
Falschpositive\is{Falschpositiv} wurden manuell überprüft und eliminiert.
\tabref{tab:beidespelkc} gibt eine Übersicht über die für das \KC{}-Material
ermittelten grafischen Varianten der Wortformen \norm{bėide} und \norm{bėidiu}
pro Position im Wort. Eine hypothetische Form \fw{*weid-} mit
bairischer\il{Bairisch} Umgekehrtschreibung von \norm{b} und \norm{w}
\autocite[153]{paul2007} konnte nicht belegt werden.

% \b(?:
%   b(?:
%     aedev|æidiv|a[yÿ]de|ejde|eyde?|êide|aid(?:
%       e[uv]?|[ií]v|u̍
%     )?|aíd(?:
%       [eí]v|u̍|e
%     )|eid(?:
%       e[uv]|iv|í[uv]|[ev]
%     )?|evdi[uv]|eíd(?:
%       í[uv]|e
%     )|ed(?:
%       [ií][uv]|e
%     )?|æd(?:
%       [ei]v|e
%     )
%   )
%   |p(?:
%     aideu|aẏ?dív|a[yÿ]de|ede?|aid(?:
%       i[uv]|ív|e
%     )?|aíd(?:
%       [ií]v|e
%     )|eid(?:
%       eu?|[ií]v
%     )?|eyd(?:
%       iv|e
%     )
%   )
% )\b

\begin{table}
\centering
\caption{In der \KC{} belegte Schreibweisen von
	mittelhochdeutsch \norm{bėide/-iu} pro Position im Wort}
\begin{tabular}{l l l l}
\lsptoprule

\mc{3}{c}{Stamm}
	& \mc{1}{c}{Flexion}
	\\

\cmidrule(r){1-3}
\cmidrule(l){4-4}

\begin{minipage}{1em}
	b,\\
	p
\end{minipage}
	& \begin{minipage}{.2\linewidth}
		ae,
		ai,
		ay,
		aí,
		aÿ,
		aẏ,
		ei,
		ej,
		ev,
		ey,
		eí,
		æi,
		êi,
		a,
		e,
		æ
	\end{minipage}
	& d
	& \begin{minipage}{.2\linewidth}
			eu,
			ev,
			iu,
			iv,
			íu,
			ív,
			u̍,
			e,
			Ø
	\end{minipage}
	\\
\lspbottomrule
\end{tabular}
\label{tab:beidespelkc}
\end{table}

Die Auswahl der Textzeugen beschränkt sich auf die drei
Leithandschriften\is{Leithandschrift} der neuen \KC{}-Edition von
\citeauthor{chincaetal2019b} (Siglen A1, B1, C1) sowie die Vergleichs- (H, VB)
und Parallelhandschriften\is{Paralleltext}, insofern sie umfassende Teile des
Texts enthalten (M, P, K, Z).%
%
	\footnote{Im Folgenden werden die einzelnen Textzeugen der \KC{} der
	Kürze halber mit Siglen bezeichnet. Diese richten sich nach
	\citetitle{kcdigital} \autocite{kcdigital}.}
%
Ausgenommen wurden die \isi{Vergleichshandschrift} VC sowie die
Parallelhandschrift W: bei Vorarbeiten hatte sich VC bezüglich ihres
Sprachstands als problematisch erwiesen, da der Text im
bairisch-österreichischen\il{Bairisch} \isi{Schreibdialekt} (mit leichtem
mitteldeutschen\il{Mitteldeutsch} Einschlag)
\blockcquote[73]{wolf:kckat}{deutlichen Änderungen im Laufe des
Schreibprozesses} unterliegt. W stellt eine Mischredaktion aus A und C dar und
enthält außerdem die \tit{Prosa\-kaiserchronik}\is{Prosa}
\autocite[48--54]{weis2022}, ist also ebenfalls ein sprachlich heterogener
Text. Fragmente wurden bei der Auswertung generell nicht berücksichtigt, da sie
im Durchschnitt einen einzigen Beleg für \norm{bėide} pro Textzeuge liefern,
jedoch in 14 von 38 Fällen keinen einzigen.

Neben einzelnen Episoden mit weiblichen Protagonisten -- zum Beispiel Veronica,
Lucretia oder Crescentia (\KC:~V.~729--838, 4335--4772, 11518--12808;
\cite[vgl.][94--96, 161--169, 292--314]{schroeder1895}) -- handelt die \KC{} dem
Prolog nach vornehmlich

\blockquote[{\KC:~V.~19--20; \cite[79]{schroeder1895}}]{
	\norm{von den bâbesen unt von den chunigen,\\
	baidiu guoten unt ubelen}

	`von den Päpsten und von den Königen,\\
	sowohl guten als auch schlechten'
}

Aufgrund der thematischen Ausrichtung geht es also vornehmlich um Männer, Paare
aus Mann und Frau kommen extrem selten vor. Die \KC{} ist trotz allem eine
Untersuchung wert, da es sich bei ihr um einen der prestige- und
einflussreichsten Texte des deutschsprachigen Mittelalters handelt
\autocite[93]{wolf2008}, durch dessen digitale Erschließung mit zeichengenauer
\isi{Transkription} \autocite{kcdigital} der Sprachwissenschaft ein
umfangreiches Textkorpus\is{Korpus} zur Verfügung steht. Die große Menge der
Textzeugen auch innerhalb der einzelnen Rezensionen macht die \KC{} attraktiv
als Parallelkorpus\is{Paralleltext} \autocite{cysouwwaelchli2007}. Alle 53
derzeit über \citetitle{kcdigital} verfügbaren Textstücke aus 47 verschiedenen
Handschriften umfassen etwa 995.832 Wortformen, wobei sich ein Mittelwert pro
Handschrift von 21.649 Wortformen ergibt; die \isi{Standardabweichung} liegt
bei 34.652 Wortformen.

Von 59 gesammelten Belegen mit kombiniertem gemischtgeschlechtlichen Bezug
liegen lediglich sechs attributiv\is{Attribut} gegenüber 53 mit \norm{bėide}
als Teil der korrelativen \isi{Konjunktion} \norm{bėide \dots\ unde} `sowohl
\dots\ als auch' vor. Dass nur sehr wenige
attributive Belege in der Stichprobe enthalten sind, macht den Vergleich
zu den Urkunden\is{Urkunde} des \CAO{} besonders interessant, da dort
umgekehrte Mengenverhältnisse vorliegen: Im \CAO{} stehen 76 Belege mit
kombiniertem gemischtgeschlechtlichen Bezug zur Verfügung, von denen 70
attributiv sind und lediglich sechs Teil der korrelativen Konjunktion. Die
Belegsammlung zur \KC{} lässt sich so als Gegengewicht zu derjenigen des \CAO{}
verstehen, nicht nur, was die Textgattung betrifft. Die \tabref{tab:beidevar}
schlüsselt die Zahl der gesammelten Belege nach Handschrift mit ihrer
jeweiligen Sigle, Funktion und Wortform auf.

\begin{table}
\centering
\caption{Vorkommen der Formen \norm{bėide} und \norm{bėidiu} in den
exzerpierten Handschriften nach Funktion}
% Diese Tabelle enthält *alle* exzerpierten Belege, ungefiltert!
\begin{tabular}[t]{
	l
	r r
	r r
	r
	r
}
\lsptoprule

\mr[c]{2}{*}{Hs.}
	& \mc{2}{c}{Quantor}
	& \mc{2}{c}{Konjunktion}
	& \mr[c]{2}{*}{Summe}
	& \mr[c]{2}{*}{Wortformen}
	\\

\cmidrule(rl){2-3}
\cmidrule(rl){4-5}

%
	& \textit{bėid(e)}
	& \textit{bėidiu}
	& \textit{bėid(e)}
	& \textit{bėidiu}
	\\

\midrule

A1
	& 24
	& 
	& 16
	& 21
	& 61
	& 90.068
	\\

M
	& 26
	& 
	& 
	& 40
	& 66
	& 84.658
	\\

H
	& 26
	& 
	& 37
	& 
	& 63
	& 86.573
	\\

\midrule

B1
	& 26
	&  6
	&  2
	& 28
	& 62
	& 79.850
	\\

P
	& 17
	& 
	& 16
	& 
	& 33
	& 28.717
	\\

VB
	& 26
	&  1
	& 25
	& 24
	& 76
	& 76.000
	\\

\midrule

C1
	& 13
	&  2
	& 
	& 34
	& 49
	& 87.791
	\\

K
	& 15
	&  2
	&  1
	& 32
	& 50
	& 89.888
	\\

Z
	& 17
	& 
	& 31
	& 
	& 48
	& 90.939
	\\

\midrule

Summe
	&     190
	&      11
	&     128
	&     179
	&     508
	& 714.484
	\\

\lspbottomrule
\end{tabular}
\label{tab:beidevar}
\end{table}

Trotz des Vorbehalts\is{Vorbehalt} zur zahlenmäßigen Verteilung der Belege auf
die unterschiedlichen syntaktischen Kontexte\is{Distribution!syntaktische} ist
gerade der Vergleich der bairischen\il{Bairisch} Handschrift B1 zum
Urkunden\-material\is{Urkunde} interessant, da B1 die einzige der neun
untersuchten Handschriften ist, in der im Kontext des kombinierten
gemischtgeschlechtlichen Bezugs beim Quantor \norm{bėide} überhaupt beide
Formen, \norm{bėide} und \norm{bėidiu}, auftreten.

In der Untersuchung des \CAO{} wird sich zeigen, dass in
bairischen\il{Bairisch} Urkunden\is{Urkunde} -- anders als in den meisten
\KC{}-Handschriften -- durchaus Variation zwischen diesen beiden Formen
herrscht. Abgesehen von \norm{bėide} in seiner Funktion als Quantor kommt
Variation in der Funktion als \isi{Konjunktion} bei den Handschriften A1 und VB
vor. Auch VB ist in Bezug auf das Urkundenmaterial insofern bemerkenswert, als
\citet[224]{schneider1987} die Handschrift unter der Rubrik zum letzten Viertel
des 13.~Jahrhunderts aufführt; \citet[65]{wolf:kckat} schätzt sie auf
\blockquote{um 1290/1300}. Die Handschrift dürfte also etwa
zeitgleich mit dem Großteil der Urkunden des \CAO{} entstanden sein
(vgl.~\tabref{tab:urkstat}).

Generell stellt sich bei der Auswertung der \KC{}-Belege die Frage nach deren
Zählweise. Anders als beim \CAO{}, das nur wenige Parallelausfertigungen
enthält \autocite[vgl.][326--328]{ganslmayeretal2003}, verteilen sich 456 von
insgesamt 508 aus der \KC{} exzerpierte Targets\is{Target} auf 110
identifizierte Parallelstellen. Dies bedeutet, dass 89,7\,\% der Gesamtzahl der
Belege mindestens einem anderen Parallelbeleg zugeordnet werden können, wobei
im Durchschnitt vier Belege ein Set bilden, neben 52 Einzelbelegen. Da auch
zwischen Parallelstellen mitunter Variation herrscht, würde eine
Zusammenfassung der zugehörigen Belege die Auszählung verzerren. Jeden Beleg
einzeln zu zählen erscheint daher als die bessere Strategie. Die eher geringe
Belegmenge pro Kontext erlaubt es in vielen Fällen, Einzelbelege und ihre
Parallelstellen anzugeben und gemeinsam zu diskutieren.

\subsection{Ausgeschlossene Belege}
\label{subsec:ausgeschlossene_kc}

In der \KC{} gibt es einzelne Stellen, die sich bei der Auswertung als
unverständlich oder fehlerhaft erwiesen und sich der morphologischen
Annotierung entzogen, oder aber in ungeeigneten Kontexten stehen und daher
ebenfalls nicht verwertbar sind. Darüber hinaus wurden die beiden
mitteldeutschen\il{Mitteldeutsch} Handschriften H und P sowie die
schwäbische\il{Schwäbisch} Handschrift Z nicht regulär in die Auswertung
einbezogen, da dort regelmäßig keine Variation zwischen \norm{bėide} und
\norm{bėidiu} vorliegt \autocite[vgl. auch][183]{ksw2}.

Der Fall eines vermuteten Abschreibfehlers wird durch den Beleg in
\REF{ex:kcexcl1} vertreten. Der Editionstext nach \citet{schroeder1895}
wird zum Vergleich in \REF{ex:kcexcl1_schroeder} angegeben. Die Korrektheit der
\isi{Transkription} der Textstelle wurde am Digitalisat\is{Digitalisat} der
Handschrift überprüft und bestätigt\is{Validierung}.

\begin{exe}
\ex \begin{xlist}
	\ex \label{ex:kcexcl1}
		\gll von den pæpſten vnd von chûnigen \\
			von den Päpsten und von Königen \\
	\sn \gll Peide und frûmigen \\
			beide und tüchtigen \\
		\trans `von den Päpsten und von Königen, sowohl als auch tüchtigen
			\textins{sic}'
			(%
				B1: 2va,20--21%
			)

	\ex \label{ex:kcexcl1_schroeder}
		\gll von den bâbesen unt von den chunigen, \\
			von den Päpsten und von den Königen \\
	\sn \gll baidiu guoten unt ubelen, \\
			beide guten und bösen \\
		\trans `von den Päpsten und von den Königen, sowohl guten als auch
			bösen'
			(%
				\KC: V.~19--20;
				\cite[79]{schroeder1895}%
			)
\end{xlist}
\end{exe}

Auch die älteste vollständige Handschrift (A1) berichtet, die Chronik handele
von den Päpsten und von den Königen \lit{bædiv gvͦten vn̄ ubelen} `sowohl guten
als auch bösen' (A1:~1ra,18; \KC:~V.~20; vgl.~\cite[79]{schroeder1895}),
insofern wird \norm{bėide} hier nicht als nachgestellter Quantor zu werten
sein. Die B-\allowbreak{}Pa\-ral\-lel\-hand\-schrift P enthält hier die
Variante \lit{bejde boſen un̄ frumege} `sowohl bösen als auch
tüchtige\textins{n}' (P:~1ra,10) und in der B-\isi{Vergleichshandschrift} VB
lautet die Zeile \lit{Boͤſen vnd frvͤmigen} `bösen und tüchtigen', allerdings
ohne einleitendes \norm{bėide} (VB:~1ra,18). Am ehesten wird der Beleg in
\REF{ex:kcexcl1} also zur gleichen Gruppe wie seine
Parallelstellen\is{Paralleltext} passen: \norm{bėide} als korrelative
\isi{Konjunktion} mit zwei koordinierten Adjektiven\is{Adjektiv!attributiv}
(vgl. \sectref{subsec:beidkoordtarg}), von denen das erste ausfällt.

Mit Bezug auf V.~19--20 verweist \citet[26, Fußnote 45]{weis2022} mit
\citet[55, Fußnote 87]{dickhutbielsky2015} außerdem auf eine Kontroverse um
deren grammatische Bezüge. Ich stimme mit \citet[239]{haupt2019} überein, V.~20
\blockquote{als Klammer nur auf die zwei zunächst folgenden Wörter
(\textit{guoten unt ubelen}), im Sinne von \enquote{sowohl als auch}} bezogen
zu lesen. Ob sich das Attribut\is{Attribut} \lit{baidiu guoten unt ubelen}
`sowohl guten als auch bösen' nur auf die Könige oder auch auf die Päpste
bezieht, lässt sich nicht eindeutig\is{Ambiguität} beurteilen. Belege, die den
Rückbezug der \isi{Konjunktion} \lit{bėide} auf zwei Konjunkte nahelegen,
ließen sich im exzerpierten Material keine finden. Da keine zwei Könige
explizit genannt werden, erscheint eine Interpretation von \lit{baidiu} als
nachgestelltem Quantor mit Bezug auf \lit{chunigen} `Königen' unplausibel,
zumal in diesem Fall die Form des Quantors (Nom./Akk.~Pl. N.) nicht kongruent
mit dem Präpositionalobjekt wäre (Dat.~Pl. M.).

In einem anderen Fall bezieht sich in dem in \REF{ex:chindpaid_1} zitierten
Beleg \lit{chínd} `Kinder' auf die Knaben Faustinus und Faustus (M:~10rb,5--6;
\KC:~V.~1239--1240; vgl.~\cite[104]{schroeder1895}). Der Beleg ist unflektiert.
Die angegebenen Parallelstellen\is{Paralleltext} enthalten jeweils \lit{der
kinde baider} `der beiden Kinder'. Bei Voranstellung des schwach flektierten
Adjektivs\is{Adjektivdeklination} wäre unabhängig vom Genus\is{Genuskongruenz}
regulär mit \norm{bėiden} zu rechnen, vergleiche \REF{ex:chindpaid_2}.

\begin{exe}
\ex \label{ex:chindpaid}
\begin{xlist}
	\ex \label{ex:chindpaid_1}
		\gll Frowe der trovm ergat dir niht zelaíd. \\
			Frau der Traum ergeht dir nicht zuleide \\
	\sn \gll Nv {vnder wínt} dich der chínd paid. \\
			nun nimm.an dich der Kind[\textsc{gen.pl.\NeutM}]
			beide[\textsc{gen.pl.\NeutM}] \\
		\trans `Frau, der Traum soll dich nicht belasten. Nun nimm dich der
			beiden Kinder an.'
		% → GEN.PL.M/F/N ansonsten stark -er, schwach -en.
			(%
				M:~11ra,17; vgl.
				A1:~6rb,22--23;
				H:~8ra,12--13;
				\KC:~V.~1354;
				\cite[106]{schroeder1895}%  1073x
			)

	\ex \label{ex:chindpaid_2}
		\gll Gewan er {dreizich tvſent}. \\
			Gewann er dreißigtausend \\
	\sn \gll Aller guten chnehte. \\
			aller gut-\textsc{gen.pl.\MascM.wk} Knechte \\
		\trans `gewann er \textins{=~Severus} für sich dreißigtausend
			von allen guten Soldaten.'
			(%
				M:~52vb,20--21; vgl.
				A1:~30rb,14--15;
				H:~41rb,36--37;
				B1:~19vc,23--24;
				VB:~33vb,20--21;
				% VC:~24vc,23--24;
				K:~42ra,24--25;
				Z:~137va,1--2;
				\KC:~V.~6967--6968;
				\cite[209]{schroeder1895}%
			)
\end{xlist}
\end{exe}

Vergleichbare Belege mit Nachstellung des Adjektivs\is{Adjektiv!attributiv} in
einer determinierten NP\is{Determiniererphrase} liegen in der
Stichprobe lediglich vier vor. Davon enthält der einzige Beleg im
Akkusativ \REF{ex:maccadj_1} unflektiertes \lit{gut} `gut', gegenüber
flektiertem \lit{guten} `guten' bei Voranstellung, wie exemplarisch in
\REF{ex:maccadj_2} gezeigt. Die Handschrift A1 enthält an dieser Stelle
% [31\vb, 29]
tatsächlich \lit{manigen helt guͦten} `manch guten Helden'.

\begin{exe}
\ex \label{ex:maccadj}
	\begin{xlist}
	\ex \label{ex:maccadj_1}
		\gll Di furten manigẽ helt gut. \\
			die führten viel-\textsc{acc.sg.\MascM.st} Held
			gut[\textsc{acc.sg.\MascM}] \\
	\sn \gll Si chomen dar mít eínẽ mvt. \\
			sie kamen dahin mit vereintem Gesinnung \\
	\sn \gll Jn di ſtat zerome. \\
			in die Stadt zu=Rom \\
		\trans `\textelp{} die führten viele gute Helden an. Sie kamen einmütig
			hin zu der Stadt Rom.'
			(%
				M:~55vb,20--24; vgl.
				A1:~31vb,27--31;
				H:~43vb,15--19;
				C1:~38va,18--21;
				% VC:~26rb,23--27; 
				K:~44rb,28--32;
				Z:~145ra,18--21;
				\KC:~V.~7348--7350;
				\cite[216]{schroeder1895}%
			)

	\ex \label{ex:maccadj_2}
		\gll Do vragt er di geſvnden \\
			da fragte er die Leute \\
	\sn \gll Wo er iheſum moht vínden. \\
			wo er Jesus könnte finden \\
	\sn \gll Den vil guten arzat \\
			den viel gut-\textsc{acc.sg.\MascM.wk} Arzt \\
		\trans `Da fragte er die Leute, wo er Jesus finden
			könnte, den sehr guten Arzt.'
			(%
				M:~6rb,31; vgl.
				A1:~3vb,26--28;
				H:~4rb,28--30;
				B1:~4rb,35--36;
				P:~7ra,9--10;
				C1:~4ra,24--27;
				Z:~12va,12--15;
				\KC:~V.~723--725;
				\cite[94]{schroeder1895}%
			)
		\\
	\end{xlist}
\end{exe}

Auffällig an \REF{ex:maccadj_1} ist, dass \lit{gut(en)} `gut(en)' auf
\norm{mvt} `Gesinnung, Stimmung, Geist' gereimt wird. Es ist möglich, dass an
dieser Stelle die Endung aus Reimzwang wegfällt. Andererseits liegt in der
Adjektivstichprobe auch in anderen untersuchten Handschriften im
Kontext der Nachstellung\is{Abfolge} eine Tendenz zur \isi{Endungslosigkeit}
vor. Dies wird bestätigt\is{Validierung} durch die Beobachtung von
\citet[241]{ksw2}, die allgemein eine über das 13.~Jahrhundert hin zunehmende
Neigung zum Weglassen der Flexion beim nachgestellten
Adjektiv\is{Adjektiv!attributiv} im Reim\is{Versende} feststellen. Die
Endungslosigkeit von \lit{paid} in \REF{ex:chindpaid_1} wird vermutlich
ebenfalls in diesem Zusammenhang stehen.

\phantomsection
\label{phsec:kchiatus}
Entsprechend der von \citet{gjelsten1980} genannten methodischen Kritikpunkte
an \posscite{askedal1974} Studie wurden außerdem solche Belege nicht in die
Auswertung einbezogen, die vor Vokal oder in Reimposition\is{Versende} stehen.
\citet{askedal1973} merkt bezüglich eines ähnlichen Falls in seiner Auswertung
der \citetitle{maroldschroeder1969}-Edition von \citet{maroldschroeder1969}
bereits selbst an, dass denkbar sei, dass die Flexionsendung \norm{-iu} in
Hiatuspositionen\is{Hiatus} bei der Rezitation apokopiert\is{Apokope} werde und
möglicherweise hyperkorrekt\is{Übergeneralisierung} dafür \orth{e} geschrieben
werde \autocite[90--91]{askedal1973}.

Des Weiteren führt er mit Bezug auf \citet[662--663]{grimm1870} die
Reimposition\is{Versende} als für eine morphosyntaktische Analyse problematisch
an, insofern \blockcquote[89]{askedal1973}{die Flexionsendung \norm{-iu} des
Nom.\ Sing.\ Fem.\ und Nom.\ Akk.\ Plur.\ Neutr.\ in der
mittelhochdeutschen\il{Mittelhochdeutsch}\ Epik niemals in der Reimposition am
Versende vorkommt}. In der Belegsammlung zur \KC{} finden sich insgesamt 133
Belege von 510 in Neutralisierungspositionen; \REF{ex:neutralpos} und
\REF{ex:neutralpos2} geben Beispiele aus beiden Gruppen. Gegenbeispiele, in
denen \norm{bėidiu} in der Reimposition vorkommt, sind auch im hier
exzerpierten Material nicht vorhanden, allerdings liegen auch keine Belege vor,
in denen andernfalls regulär mit \norm{-iu} zu rechnen wäre. Belege mit
\norm{iu} vor Vokal gibt es immerhin elf, davon bis auf einen
\REF{ex:neutralpos2} nur bei \norm{bėide} als \isi{Konjunktion}.

\begin{exe}
\ex \label{ex:neutralpos}
	\gll Den gewalt habten ſi beide \\
		den Macht hatten \textsc{3pl\subM.nom} beide-\textsc{nom.pl\subM.st} \\
\sn \gll Daz enſchol v̂ns nímmer werden leide \\
		das \textsc{neg}=wird uns nie werden Leid \\
	\trans `Diese Macht hatten sie beide. Das wird uns nie Leid werden.'
		(%
			B1:~13ra,41--42; vgl.
			A1:~18rb,3;
			H:~24vb,35;
			M:~32ra,3;
			VB:~21ra,31;
			P:~37ra,26;
			C1:~23rb,5;
			K:~25rb,39;
			Z:~83ra,5;
			\KC:~V.~4261--4264;
			\cite[159]{schroeder1895}%
		)

\ex \label{ex:neutralpos2}
	\begin{xlist}
	\ex \label{ex:neutralpos2_1}
		\gll der berch der haizet gargan. \\
			der Berg der heißt Gargān \\
	\sn \gll da worden ſi baide oͮffe erſlagen. \\
			da wurden \textsc{3pl\subM.nom} beide-\textsc{nom.pl.\MascM.st} auf
				erschlagen \\
		\trans `Der Berg heißt Gargān, darauf wurden sie beide erschlagen.'
			(%
				A1:~33rb,28--30; vgl.
				H:~46ra,12;
				M:~58vb,15;
				C1:~40va,15;
				K:~46va,30;
				Z:~152va,5;
				\KC:~V.~7704--7705;
				\cite[222]{schroeder1895}%
			)

	\ex \label{ex:neutralpos2_2}
		\gll Rome vnd Lateran \\
			Rom und Lateran \\
	\sn \gll Wurden ím baidu̍ vndertan \\
			wurden ihm beide-\textsc{nom.pl.\NeutI.st} untertan \\
		\trans `Rom und Lateran wurden ihm beide untertan.'
			(%
				K:~68vb,13--14; vgl.
				A1:~49vb,15--16;
				H:~69ra,29--30;
				M:~87ra,32--33;
				B1:~31vb,18--19;
				VB:~82va,45--46;
				C1:~60ra,14--15;
				% VC:~39vc,25--26;
				Z:~229ra,6--7;
				\KC:~11416--11417;
				\cite[290]{schroeder1895}%
			)
		\\
	\end{xlist}
\end{exe}

Mit Ausnahme der Handschrift K wurden keine \KC{}-Text\-zeugen mit
alemannischem\il{Alemannisch} \isi{Schreibdialekt} in die Analyse aufgenommen.
Der Vollständigkeit halber seien in \REF{ex:beidkcalem} die wenigen Belege zu
\norm{bėide} aus solchen Fragmenten aufgeführt, die im
\tit{Handschriftencensus} \nosh\autocite{hsc} als spezifisch
alemannisch\il{Alemannisch} ausgewiesen werden \autocite[vgl.][4, 44,
54]{wolf:kckat}.

\begin{exe}
\ex \label{ex:beidkcalem}
	\begin{xlist}
	\ex \gll bediv rome vnde Lateran \\ % 2013
		     beide Rom[\textsc{acc.sg.\NeutI}] und
		     Lateran[\textsc{acc.sg.\MascI}] \\
		\trans `sowohl Rom als auch der Lateran'
			(%
				a11:~2vb,17; vgl.
				\KC:~V.~5953;
				\cite[191]{schroeder1895}%
			)

	\ex \gll die iuncherren beide \\ % 1091
		     die Junker-\textsc{nom.pl} beide-\textsc{nom.pl.\MascM} \\
		\trans `die beiden jungen Herren'
			(%
				a14:~1vb,14; vgl.
				\KC:~V.~1417;
				\cite[107]{schroeder1895}%
			)

	\ex \gll Beidiv die ſtat \textins{u}n̄ daz lan\textins{t} \\ % Ø
		     beide die Stadt[\textsc{acc.sg.\FemI}] und das
		     Land[\textsc{acc.sg.\NeutI}] \\
		\trans `sowohl die Stadt als auch das Land'
			(%
				b1:~1rb,4%
			)

	\ex \gll Beidiv wip un̄ man \\ % 2004
		     beide Frau[\textsc{nom.sg.\NeutF}] und
		     Mann[\textsc{nom.sg.\MascM}] \\
		\trans `Frau wie Mann'
			(%
				b1:~2rb,29; vgl.
				\KC:~V.~619;
				\cite[92]{schroeder1895}%
			)
\end{xlist}
\end{exe}
\is{Stichprobe|)}
