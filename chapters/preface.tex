\addchap{\lsPrefaceTitle}

Die vorliegende Arbeit stellt eine überarbeitete Fassung meiner Dissertation
mit dem Titel \tit{Kongruenzphänomene im Mittelhocheutschen: Gender Resolution
bei mittelhochdeutsch \emph{beide}. Eine Analyse im Rahmen der
Lexical-Functional Grammar} dar, die ich, Carsten Becker aus Lörrach, am
19.05.2022 am Fachbereich Germanistik und Kunstwissenschaften der
Philipps-Universität Marburg (Hochschulkennziffer:~1180) eingereicht und am
20.09.2022 disputiert habe. Begutachtet wur\-de die Arbeit von Prof.~Dr.~Jürg
Fleischer und Prof.~Dr.~Jürgen Wolf.

Größere Änderungen zwischen der eingereichten und der publizierten Fassung
haben sich im Abschnitt zu Genus, Sexus und Belebtheit (\sectref{sec:gendsex})
ergeben. Dieser wurde um die Rezeption von Fachliteratur zu diesen drei
Themengebieten erweitert, anstatt lediglich die Situation des Genusgebrauchs in
den ausgewerteten Quellen und das Vorgehen bei der Annotation in der
vorliegenden Untersuchung kurz zu beschreiben. \chapref{ch:theorie} zu
theoretischen Grundlagen (ursprünglich Kapitel~4) wurde zugunsten eines
stringenteren Aufbaus vorgezogen.

Darüber hinaus wurde der Text des Einleitungskapitels (\chapref{ch:einleitung})
leicht erweitert, um die Motivation der Arbeit und des verwendeten
Theorieansatzes sowie bewusste Einschränkungen hinsichtlich der verwendeten
Materialien expliziter herauszustellen. Ein Missverständnis zu
\posscite{askedal1973} Markiertheitsbegriff in \sectref{phsec:markiertheit}
wurde ausgeräumt und die Passage umformuliert. In den Aus\-wertungs\-kapi\-teln
\ref{ch:adjflex} bis \ref{ch:kcanalyse} sind ausführliche Zitatsammlungen
zugunsten einzelner, illustrativer Beispiele gekürzt worden. Außerdem wurde die
Dokumentation der Stich\-probe zur Grafie des Schwa\-vokals in alemannischen
Urkunden (\sectref{subsec:ausgesurk}) nachgetragen. Formale sowie kleine
inhaltliche Fehler wurden stillschweigend korrigiert, Sei\-ten\-zahlen in
aktualisierten Auflagen angepasst und bibliografische Angaben zu mittlerweile
erschienener Fachliteratur ergänzt.

Mit meiner Dissertation knüpfe ich thematisch an meine Masterarbeit an
\autocite{becker2016}, in der ich mich bereits mit der Flexion von Adjektiven
im Mittelhochdeutschen anhand des \tit{Corpus der altdeutschen Originalurkunden
bis zum Jahr 1300} (\CAO) und der B-Handschrift des \tit{Iwein} Hartmanns von
Aue (Gießen, Universitätsbibl., Hs~97) beschäftigt habe. Die Idee zum Thema der
Dissertation erwuchs aus der gewinnbringenden Lektüre von \citet{corbett2006}
zu Kongruenz\-phänomenen und der Feststellung, dass die \tit{Mittelhochdeutsche
Grammatik} von \citet{paul2007} gerade zu interessanten morpho\-syntaktischen
Details häufig nur sehr pauschale Angaben macht. Vor dem Hintergrund
sprach\-historischer, dialekt\-geografischer Studien zum \CAO{}
\autocite{beckerschallert2021,beckerschallert2022a, beckerschallert2022b}
erschien es reizvoll, sich dezidiert einem Phänomen zu widmen, das zwar bekannt,
aber nicht allzu häufig belegt ist, um gerade auch die Ergiebigkeit und
Aussagekraft des Urkunden\-materials zu einem grammatischen Randfall zu testen.

Ein herzliches Dankeschön gilt allen, die in den letzten sieben Jahren mit
ihrem Zuspruch und ihrer Unterstützung dazu beigetragen haben, dass die Arbeit
an diesem Buch ein gutes Ende gefunden hat. An erster Stelle sind das meine
Betreuer, Jürg Fleischer und Jürgen Wolf. Von euch durfte ich lernen,
philologisches und linguistisches Know-how gewinnbringend miteinander zu
verknüpfen. Danke, dass ihr mir den Freiraum gegeben habt, bei dieser Arbeit
alle Register meiner Interessen zu ziehen, stets ein offenes Ohr für Fragen
hattet und nie verlegen wart, euch Zeit zur Korrektur und Besprechung von
Entwürfen zu nehmen.

Oliver Schallert danke ich für den Funken, der letztlich die Idee zu dieser
Untersuchung gezündet hat. Danke auch für deine Gastfreundschaft, sei es in
Marburg oder in Augsburg, sowie für dein freundschaftliches Mentoring seit es
mich als studentische Hilfskraft in der Deutschen Philologie des Mittelalters
vor gut zehn Jahren zwecks Kooperation zwischen unseren beiden Abteilungen in
die Sprachgeschichte verschlagen hat. Dank gilt darüber hinaus Daniel David
Weis, Lea Schäfer und Hanna Fischer sowie den Community-Proofreader*innen von
Language Science Press. Auch von euren und Ihren Vorschlägen, Korrekturen,
Nachfragen und Anmerkungen durfte ich profitieren. Ungenauigkeiten und Fehler,
die jetzt noch im Text enthalten sind, liegen allein in meiner Verantwortung.
Daniel David Weis und Magnus Breder Birkenes möchte ich außerdem herzlich
danken für viele gute Gespräche und freundschaftliche Ermutigungen,
durchzuhalten, ganz besonders während zäher Zeiten.
\bigskip

\noindent%
Carsten Becker\hfill Berlin im Dezember 2023
