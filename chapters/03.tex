\chapter{Forschungsüberblick}
\label{ch:forschungsueberblick}

\section%
	{\norm{Bėide} als Quantor -- \norm{X unde Y \dots\ (si) bėide/bėidiu}}
\label{sec:ovwbeidequant}

Bei dem Kongruenzphänomen, das in dieser Studie anhand des Quantors
\norm{bėide} `beide (\textsc{nom+acc.pl.m+f.st})' beziehungsweise \norm{bėidiu}
`beide (\textsc{nom+acc.pl.n.st})' in mittelhochdeutschen\il{Mittelhochdeutsch}
Quellen untersucht wird, handelt sich um eine bekannte Eigenheit der
germanischen\il{Germanisch} Sprachen. Diese zeigt sich insbesondere in den
älteren Sprachstufen, ist darüber hinaus aber noch im modernen
Isländischen\il{Isländisch} \autocites[283]{corbett1991}[569]{wechsler2009} und
Färöischen\il{Färöisch} \autocite[225--226]{thrainsson2004} erhalten geblieben.
In \REF{ex:germbeide} werden zur Illustration einige Beispiele aus
historischen\is{Diachronie} Quellen präsentiert.%
%
	\footnote{Die Stellen sind \citet[12]{askedal1973} entnommen, vgl.~auch
		\citet{hock2008,hock2009}.}

\begin{exe}
\ex \label{ex:germbeide}
	\begin{xlist}
	\ex \label{ex:germbeide_1}
		\langinfo%
			{Gotisch}
			{}
			{\cite[nach][]{projectwulfila2004}}
			\\
		% Digi: https://www.alvin-portal.org/alvin/imageViewer.jsf?dsId=ATTACHMENT-0147&pid=alvin-record:60279
		\gll Iosef \textelp{} miþ Mariin \textelp{} warþ þan,
				miþþanei þo wesun jainar \textelp{} \\
			Josef[\textsc{nom.sg.\MascM}] {} mit Maria-\textsc{dat.sg.\FemF} {}
				wurde dann während \textsc{dem.nom.pl.\NeutMF} waren dort {} \\
		\trans `Josef \textelp{} mit Maria \textelp{} Während diese dort
			waren, wurde dann \textelp{}'
			(%
				\iai{Wulfila}, \tit{Bibel}: Lk~2,4--6;
				vgl. Uppsala, Universitätsbibl., MS~DG~1: 124r,1--9%
			)

	\ex \label{ex:germbeide_2}
		\langinfo%
			{Altwestnordisch}
			{}
			{\cite[nach][73]{neckelkuhn1962}}\\
		% Digi: https://handrit.is/manuscript/view/is/GKS04-2365/31?iabr=on#page/11v
		\gll Gerðr qvað: \textelp{} \\
			Gerðr[\textsc{nom.sg.\FemF}] sprach {} \\
		\gll né við Freyr, meðan occart fiǫr lifir, \\
		% ne viþ freẏ meꝥ occart fior lif͛
			noch \textsc{1du\subMF.nom} Freyr[\textsc{nom.sg.\MascM}] während
				\textsc{1du.gen-nom.sg.\NeutMF} Körper lebt \\
		\gll byggiom bæði saman. \\
		% byɢiõ bęði ſaman.
			wohnen-\textsc{1pl.ind.prs} beide-\textsc{nom.pl.\NeutMF.st} zusammen \\
		\trans `Gerðr sprach: \textelp{} noch Freyr und ich, so
			lange wir leben, beide zusammen wohnen.'%
		%
			\footnote{Für die Glossierung danke ich Svenja Walkenhorst
				(Marburg). \citet[53]{terry1990} übersetzt frei:
				\foreignblockquote{english}{Frey will never
				enjoy my favor as long as we're both alive}.}
		%
			(%
				\tit{Edda}: For Skírnis 20,4--5;
				vgl. Reykjavík, Stofnun Árna Magnússonar, GKS 2365 4to: 22,18--19%
			)

	\ex \label{ex:germbeide_3}
		\langinfo%
			{Altenglisch}
			{}
			{\cite[nach][27]{krapp1931}}\\
		% Digi: https://digital.bodleian.ox.ac.uk/objects/d5e3a9fc-abaa-4649-ae48-be207ce8da15/
		% \gll ⁊ ƿit hér baru ſtandað. \\
		\gll and wit her baru standað \\
			und \textsc{1du\subMF.nom} hier nackt-\textsc{nom.pl.\NeutMF.st}
				stehen \\
		\trans `und wir beide \textins{=~Adam (\MascM) und Eva (\FemF)}
			hier nackt stehen.'
			(%
				\iai{Cædmon}, \tit{Genesis}: V.~811;
				vgl. Oxford, Bodleian Lib., Cod.~Junius 11: 38,9--10%
			)			

	\ex \label{ex:germbeide_4}
		\langinfo%
			{Altsächsisch}
			{}
			{\cite[nach][35]{sievers1878}}\\
		\gll Giuuitun im tho thiu godun tuue \\
			heiligten ihm da \textsc{dem.nom.pl.\NeutMF} gut-\textsc{nom.pl.wk}
			zwei[\textsc{\NeutMF}] \\
		\gll Ioseph endi Maria \\
			Josef[\textsc{nom.sg.\MascM}] und Maria[\textsc{nom.sg.\FemF}] \\
		\gll bediu fon Bethleem \\
			beide-\textsc{nom.pl.\NeutMF.st} von Bethlehem \\
		\trans `Da beteten ihn diese guten zwei an, Josef und Maria, die beiden
			von Bethlehem.'
			(%
				\tit{Heliand}: V.~458--459;
				vgl. München, Bayerische Staatsbibl., Cgm~25: 7v,5--6%
				% [\cite[8827]{hsc}]
			)

	\ex \label{ex:germbeide_5}
		% Althochdeutsch (Wien, ÖNB, Cod. 2687: 15v,20):
		% \gll Vuárun ſẹ\textsup{iu} béthịu góte filu drúdịu \\
		\langinfo%
			{Althochdeutsch}
			{}
			{\cite[nach][15v]{kleiberhellgardt2004}}\\
		\gll Vuárun ſiu béthịu góte fılu drúdịu \\
			Waren \textsc{3pl.nom.pl.\NeutMF} beide-\textsc{nom.pl.\NeutMF.st}
				Gott viel lieb-\textsc{nom.pl.\NeutMF.st} \\
		\trans `Sie beide \textins{=~Zacharias (\textsc{\MascM}) und
			Elisabeth (\textsc{\FemF})} waren Gott sehr lieb.'
			(%
				\iai{Otfrid von Weißenburg}, \tit{Evangelienbuch}: I.4,5;
				vgl. Wien, Österreichische Nationalbibl., Cod.~2687: 15v,20%
				% [\cite[6494]{hsc}]
			)
		\\
	\end{xlist}
\end{exe}

In allen Fällen liegen Paare aus Mann und Frau vor; sich auf diese gemeinsam
beziehende Wortformen weisen jeweils das Neutrum auf. Auch wenn in
\REF{ex:germbeide_1} und \REF{ex:germbeide_3} mit \lit{þo} `diese
(\textsc{pl.\NeutMF})' beziehungsweise \lit{baru} `nackt
(\textsc{pl.\NeutMF})' keine Formen von \fw{beide} betroffen sind,
illustrieren sie durch ihre neutrale Kongruenz dennoch das Phänomen.

Mit Fokus auf die älteren Sprachstufen des Deutschen beobachtet dazu bereits
\citeauthor{grimm1848} in der \citetitle{grimm1848}:
\blockcquote[978]{grimm1848}{Da unserm adjectiv und, ausser dem
persönlichen, dem übrigen pronomen die dualform mangelt, so verdient hier
erwogen zu werden, dass unsre syntax mit zwei subjecten verschiednes
geschlechts das adj.\ im pl.~neutr.\ verbindet}. In der \tit{Deutschen
Grammatik}\il{Neuhochdeutsch} wird dieser Gedanke vertieft:
\blockcquote[311--312]{grimm1890}{\textins*{S}ollen adjectiva oder pronomina
auf ein männliches und weibliches subst.\ \emph{zugleich} bezogen werden, so
stehen sie im \emph{neutro}; jene subst.\ mögen vom natürlichen oder bloß
grammatischen geschlecht sein}. An späterer Stelle argumentiert
\citeauthor{grimm1898}, dass für die Kombination von Maskulinum und Femininum
\blockcquote[329]{grimm1898}{der uralte grundsatz \textins{gilt}, daß ein auf
beide zugleich bezügliches pron.\ adj.\ und partic.\ in den \emph{pl.\ des
\mbox{neutr.}} zu stehn kommt, und gerade vorzugsweise bei personen}. Das
Neutrum trete darüber hinaus auch in Abhängigkeit von der Kombination eines
Singulars Maskulinum oder Femininum mit einem Singular Neutrum auf
\autocite[331]{grimm1898}.

Speziell auf das Mittelhochdeutsche\il{Mittelhochdeutsch} bezogen fällt die
Regelangabe bei \citeauthor{paul2007} gewohnt knapp und impressionistisch aus:
\blockcquote[384]{paul2007}{In der Regel steht ein Pron.\ oder Adj.\ im
Pl.~Neutr., wenn es zusammenfassend auf mehrere Subst.\ mit unterschiedlichem
Genus bezogen ist \textelp{}. Es kann aber auch der Pl.~Mask.\ in solchen
Fällen gebraucht werden}. Ähnlich kurz wie \citet{paul2007} formuliert
\citet[39]{behaghel1928}: \blockquote{Das feste Glied besteht aus einer Summe
von Gliedern verschiedenen Geschlechts; es erscheint von alters her das
Attribut\is{Attribut}, das Prädikat und das aufnehmende Pronomen im Neutrum
Pluralis. \textelp{} Doch wird diese aus der Vorzeit überkommene formale
Regelung nicht selten durch die Rücksicht auf das natürliche Geschlecht
durchkreuzt}.

\citet{behaghel1928} scheint also anzunehmen, dass das Neutrum als
Ausgleichsform aufgrund eines formalen, morphosyntaktisch bedingten Mechanismus
erscheine, während die neuere Forschung etwa von \citet{wechslerzlatic2003} und
\citet{wechsler2009} davon ausgeht, dass Gender Resolution ein in der Semantik
begründetes Phänomen darstellt, wie in \chapref{ch:diskussion} zu zeigen sein
wird. Darüber hinaus vermerken auch \citet[188]{dal2014} bloß, dass
\blockquote{\textins*{w}enn mehrere Substantive von verschiedenem Geschlecht
durch Konjunktion verbunden sind, \textelp{} darauf bezogene
Attribute\is{Attribut} und hinweisende Pronomina ursprünglich im Neutr.\ Plur.\
\textins{stehen}}.

Ausführliche Kritik an den Formulierungen von \citet{grimm1890,grimm1898} und
\citet{behaghel1928} sowie von Dal\nocite{dal2014} und Paul\nocite{paul2007},
die auch in den jeweils aktuellen Auflagen
\autocite{dal2014,paul2007} unverändert geblieben sind, übt bereits
\citet[11--15, 195--213]{askedal1973}. Dieser beschäftigt sich mit der
diachronen\is{Diachronie} Entwicklung dieses Phänomens vom
Althochdeutschen\il{Althochdeutsch} bis zur \q{Blütezeit} der
mittelhochdeutschen\il{Mittelhochdeutsch} Epik um 1200
\autocites[vgl.][317]{schneidermohr2001}[dazu auch][3--29]{johnson1999}. Die
Essenz seiner Kritik ist, dass bei den genannten Darstellungen

\begin{itemize}
	\item nicht klar zwischen natürlichem und grammatischem Geschlecht (also
		semantischen und formalen Kriterien) unterschieden wird;
	\item die Regelangaben die diachrone\is{Diachronie} Ebene außer Acht
		lassen;
	\item insbesondere auf die Verbindung von Personen (Animata\is{Animata})
		eingegangen wird, kaum aber auf die von Dingen und Abstrakta
		(Inanimata);
	\item sich die Beschreibung auf die bloße Feststellung von Regel und
		Ausnahme\is{Ausnahme} beschränkt und dabei \isi{Belebtheit} kaum eine
		oder keine Rolle spielt;
	\item als Begründung für die Kongruenzregel entweder nur \q{formale} (das
		heißt morphologische?) Aspekte oder in der Tradition der
		Junggrammatiker die reine Lautentwicklung angeführt werden.
\end{itemize}

\citet{askedal1973} geht es bei seiner Studie nicht nur um
\norm{bėidiu}, sondern um den größeren Kontext des Verlusts der Genusopposition
zwischen Maskulinum-Femininum und Neutrum im Plural
\autocite[169--177]{askedal1973}. Er untersucht dafür hauptsächlich das
Pronominalsystem der fünf Texte: \tit{Tatian}, Otfrids von Weißenburg\ia{Otfrid
von Weißenburg} \tit{Evangelienbuch}, die \tit{Altdeutsche Genesis}, Wolframs
von Eschenbach\ia{Wolfram von Eschenbach}
\tit{Parzival} und Gottfrieds von Straßburg\ia{Gottfried von Straßburg}
\tit{Tristan}. Herausfordernd für eine quantitative Untersuchung ist, dass in
allen Texten trotz ihrer Länge jeweils nur wenige relevante Belege vorliegen
\autocites[187]{askedal1973}[118]{fleischerschallert2011}. Wichtig an
\citeauthor{askedal1973}s Arbeit ist der Versuch, \fw{Gender Resolution} -- die
Auflösung konfligierender Genusmerkmale in der Kongruenz\-morphologie
\autocites(siehe \sectref{sec:gendres}){corbett1983} -- als komplexen
morphosyntaktischen Vorgang aufzufassen, nämlich als eine syntaktische
Operation, die mit semantischen und formalen Merkmalen operiert und in der
Morphologie ihren Ausdruck findet. Leider lässt \citet{askedal1973} eine
übersichtliche Darstellung seiner Daten und Ergebnisse vermissen, sodass ein
direkter Vergleich zwischen seinen einzelnen Teilauswertungen schwierig ist;
eine kurze Zusammenfassung der wichtigsten Erkenntnisse seiner Arbeit bieten
\citet[118--119]{fleischerschallert2011}.

Interessant für die vorliegende Studie ist, dass der \tit{Parzival} und der
\tit{Tristan} bezüglich ihres zeitlichen Überlieferungsschwerpunkts in den
Rahmen der hier untersuchten Texte fallen. \citet[1378]{bumke1999} zufolge kann
als Entstehungszeit für den \tit{Parzival} Wolframs von Eschenbach etwa die
Zeit zwischen 1200 und 1210 angesetzt werden. Die ältesten erhaltenen
Textzeugen werden auf die erste Hälfte des 13.~Jahrhunderts, die jüngsten auf
die Zeit um 1500 datiert
\autocites[1381]{bumke1999}[vgl.~auch][s.\,v.~\textit{Wolfram von Eschenbach:
\tit{Parzival}}]{hsc}. Die Entstehungszeit des \tit{Tristan}-Romans Gottfrieds
von Straßburg\ia{Gottfried von Straßburg} gibt \citet[155]{kuhn1982} mit der
Zeit zwischen 1200 und 1220 an. Er ist uns in Handschriften aus dem frühen
13.~Jahrhundert bis zum letzten Viertel des 15.~Jahrhunderts überliefert
\autocite[vgl.][s.\,v.~\textit{Gottfried von Straßburg: \tit{Tristan}}]{hsc}.

Die Belegverteilung für den \tit{Parzival} und den \tit{Tristan} bei
\citet{askedal1973} wird in \tabref{tab:askbeide} aufgeführt, insofern die
Daten aus dem Fließtext herausgelesen werden konnten. Kontexte mit der
Flexionsendung in Hiatuspositionen und am Zeilenende wurden dabei jeweils nicht
mitgezählt. Das heißt, \citet[89--91]{askedal1973} argumentiert mit Verweis auf
\citet[662--663]{grimm1870}, dass in mittelhochdeutschen\il{Mittelhochdeutsch}
Verstexten im Endreim nie \norm{-iu} zu beobachten ist, stattdessen immer
\norm{-e} gesetzt wird. Des Weiteren mahnt er zur Vorsicht bei Kontexten, in
denen in einem metrisch gebundenen Text eine flektierte Wortform auf Vokal
endet und die nachfolgende mit einem Vokal beginnt. Um einen metrisch glatten
Vers zu produzieren, kann der unbetonte Endvokal der ersten Wortform beim
Vortrag ausfallen, wodurch die Möglichkeit besteht, dass der grammatische
Unterschied zwischen \norm{-e} und \norm{-iu} durch \isi{Apokope} des
adjektivischen Flexionsmorphems\is{Adjektiv} zur Vermeidung eines Hiatus
aufgehoben wird.%
%
	\footnote{Wohl vor dem Hintergrund der
		mittelhochdeutschen\il{Mittelhochdeutsch} Schwa-\isi{Apokope}
		\autocites{lindgren1953}[109--111]{paul2007} spekuliert
		\citet[91]{askedal1973}, \textquote{daß das Graphem \orth{e} für jeden
		tilgbaren Vokal gesetzt werden darf, d.\,h., der Verwendungsbereich des
		\orth{e} wird auch auf getilgtes [y] erweitert.} \citet[27, 109--111,
		203]{paul2007} sowie eine Suche nach dem Stichwort \q{Hiatus} in der
		digital aufbereiteten Version der Grammatik geben jedoch keinen Hinweis
		auf ein derartiges Phänomen; \citet[244]{ksw2} vermerken bezüglich der
		Adjektivflexion\is{Adjektivdeklination} allenfalls unflektierte Formen
		zur Hiatusvermeidung.}
		
\begin{sidewaystable}
\caption%
	{Belegverteilung von \norm{bėide} und \norm{bėidiu} bei Wolframs\ia{Wolfram
	 	von Eschenbach} \tit{Parzival} und Gottfrieds\ia{Gottfried von
	 	Straßburg} \tit{Tristan} in \citet{askedal1973}}
\begin{threeparttable}
\begin{tabular}{
	l l
	c
	r r
	c
	r r
	c
	r
}
\lsptoprule
\mr{2}{*}[-.5ex]{Bezug auf}
	& \mr{2}{*}[-.5ex]{Bezugsart}
	& % --
	& \mc{2}{c}{\tit{Parzival}}
	& % --
	& \mc{2}{c}{\tit{Tristan}}
	& % --
	& \mr{2}{*}[-.5ex]{Summe}
	\\

\cmidrule{4-5}
\cmidrule{7-8}

%
	& %
	& % --
	& \norm{bėide}
	& \norm{bėidiu}
	& % --
	& \norm{bėide}
	& \norm{bėidiu}
	& % --
	& %
	\\

\midrule

\mr[t]{3}{*}{\makecell[tl]{versch.\ Genera\\ (belebt)}}
	& direkt
	& % --
	& %
	& 2\tnote{a}
	& % --
	& 1\tnote{b}
	& 1\tnote{a}
	& % --
	& 4
	\\

%
	& indirekt (\norm{si} etc.)
	& % --
	& 1
	& 10
	& % --
	& 7
	& 6
	& % --
	& 24
	\\

%
	& indirekt (\norm{diu})
	& % --
	& %
	& 1
	& % --
	& 1
	& %
	& % --
	& 2
	\\

\midrule

Mann + Pferd
	& direkt
	& % --
	& %
	& 1\tnote{a}
	& % --
	& 1\tnote{a}
	& %
	& % --
	& 2
	\\

\midrule

\mr[t]{2}{*}{\makecell[tl]{gl.\ Genus\\ (unbelebt)}}
	& direkt
	& % --
	& %
	& 3\tnote{c}
	& % --
	& %
	& %
	& % --
	& 3
	\\

%
	& indirekt (\norm{diu})
	& % --
	& %
	& 1
	& % --
	& %
	& %
	& % --
	& 1
	\\

\midrule

\mr[t]{4}{*}{\makecell[tl]{versch.\ Genera\\ (unbelebt)}}
	& direkt
	& % --
	& %
	& 1
	& % --
	& %
	& %
	& % --
	& 1
	\\

%
	& indirekt (\norm{si} etc.)
	& % --
	& 1
	& 1
	& % --
	& %
	& 3
	& % --
	& 5
	\\

%
	& indirekt (\norm{die})
	& % --
	& %
	& 1
	& % --
	& %
	& %
	& % --
	& 1
	\\

%
	& indirekt (\norm{diu})
	& % --
	& 1
	& 2
	& % --
	& 1
	& 2
	& % --
	& 6
	\\

\midrule

Summe
	& %
	& % --
	& 3
	& 23
	& % --
	& 11
	& 12
	& % --
	& 49
	\\

\lspbottomrule	
\end{tabular}
\begin{tablenotes}[para]
\footnotesize
	\item [a] Distanzstellung
	\item [b] pronominal-anaphorisch\is{Anapher}
	\item [c] pronominal-kataphorisch
\end{tablenotes}
\end{threeparttable}
\label{tab:askbeide}
\end{sidewaystable}

\citet{askedal1973} unterscheidet in seiner Beleganalyse nicht strikt nach
bestimmten Kombinationen von Sexus oder Genus, sondern lediglich zwischen der
Kombination von gleichen und verschiedenen Genera sowie belebtem\is{Animata}
(oder menschlichem) und unbelebtem Bezug. In der Spalte \textit{Bezugsart}
bezieht sich \q{direkt} auf den unmittelbaren Bezug von \norm{bėide} auf zwei
nominale Größen und \q{indirekt} auf den mittelbaren Bezug über ein Pronomen.
\norm{Si} steht dabei stellvertretend für genusindifferente Personalpronomina
gemäß \posscite[97]{askedal1973} Klassifikation, wozu nicht nur \norm{si} `sie
(\textsc{pl})' und \norm{wir} `wir', sondern auch \norm{uns} `uns' und
\norm{iuch} `euch' gehören. In der pronominalen Verwendung treten \norm{bėide}
und \norm{bėidiu} anaphorisch\is{Anapher} oder kataphorisch auf, also
rückverweisend auf zwei im Kontext bereits etablierte \REF{ex:askedal73pr_1}
beziehungsweise vorausweisend auf zwei zu etablierende Referenten
\REF{ex:askedal73pr_2}.

\begin{exe}
\ex \label{ex:askedal73pr}
	\begin{xlist}
	\ex \label{ex:askedal73pr_1}
		\gll da riwalin da {blanſcheflur ·} \\
			da Riwalīn[\textsc{nom.sg.\MascM}] da
				Blanscheflūr[\textsc{nom.sg.\FemF}] \\
	\sn \gll da beide da lealamûr \\
			da beide-\textsc{nom.pl.m+f\subMF.st} da \fw{leal amur} \\
		\trans `da Riwalīn, da Blanscheflūr -- da beide, da \fw{leal amur}'
			(%
				\iai{Gottfried von Straßburg}: \tit{Tristan}, V.~1359--1360
				nach München, Bayerische Staatsbibl., Cgm~51: 9v,28--29;
				% [\cite[1286]{hsc}],
				vgl.~\cite[22]{maroldschroeder1969}%
				% (= S. 22 → in ⁵2004: S. 25)
			)

	\ex \label{ex:askedal73pr_2}
		\gll nv rvͦche helt mir beidiv ſagen. \\
			nun geruhe Held mir beide-\textsc{acc.pl.\NeutI.st} sagen \\
	\sn \gll dinen namen vnt dinen art. \\
			dein-\textsc{acc.sg.\MascI.st} Name[\textsc{\MascI}]-\textsc{obl}
			und dein-\textsc{acc.sg.\MascI.st}
			Herkunft[\textsc{acc.sg.\MascI}] \\
		\trans `Nun geruhe Held, mir beides zu sagen, deinen Namen und
			deine Herkunft.'\footnotemark{}
			(%
				\iai{Wolfram von Eschenbach}, \tit{Parzival}: 745,18--19
				nach St.~Gallen, Stiftsbibl., Cod.~Sang.~857: 265a,33--34;
				% [\cite[1211]{hsc}],
				vgl.~\cite[749]{knechtschirok2003}%
			)
	\end{xlist}
\end{exe}
%
	\footnotetext{Das Substantiv \norm{art}, u.\,a. `Herkunft, Abstammung,
		Gattung', ist sowohl als Maskulinum als auch als Femininum belegt
		\autocite[s.\,v.~\textit{art}]{mwb1}. Ausweislich der Flexion des
		vorausgehenden Possessivbegleiters handelt es sich an der zitierten
		Stelle um ein Maskulinum. \citet[749]{knechtschirok2003}
		übersetzen sinngemäß \textquote{Nun sei so freundlich, Held, und sag
		mir deinen Namen und woher du kommst.}%
		}

Mit Bezug auf das zu untersuchende Phänomen lässt sich aus
\citeauthor{askedal1973}s Besprechung seiner Belege entnehmen, dass gerade im
Mittelhochdeutschen\il{Mittelhochdeutsch} nach 1200 eine Ausweitung der
maskulin-femininen Form bei kombinierter gemischt\-geschlechtlicher Referenz in
die morphologische Domäne der Modifikatoren vordringt und in Konkurrenz zum
Neutrum tritt. So steht im \tit{Parzival} beim direkten, das heißt
attributiven\is{Attribut} Bezug auf die Kombination von Referenten mit
unterschiedlichem Geschlecht, Mensch und Nutztier sowie unbelebten Referenten
vom gleichen Genus in allen sechs Fällen die neutrale Form \norm{bėidiu}. Und
auch beim indirekten Bezug von \norm{bėide} auf belebte\is{Animata} Referenten
von verschiedenem Geschlecht mittels Pronomina steht hauptsächlich \norm{si
bėidiu}. Beim Bezug auf unbelebte Referenten mit verschiedenem Genus kommen
aber nahezu alle Kombinationsmöglichkeiten von \norm{si/die/diu bėide/bėidiu}
nur einmal vor, sodass keine klare Tendenz zu erkennen ist. Beim indirekten
Bezug auf unbelebte Referenten mit gleichem Genus steht neutral
\norm{diu bėidiu} \autocites[145--148,
158--161]{askedal1973}[nach][]{lachmannhartl1952}.

Im \tit{Tristan} steht beim direkten belebten\is{Animata} Bezug auf
unterschiedliche Geschlechter und beim Bezug auf Mensch und Reittier in den
meisten Fällen die maskulin-feminine Form \norm{bėide}. Beim indirekten
belebten\is{Animata} Bezug auf unterschiedliche Geschlechter kommen dagegen
\norm{si bėide} und \norm{si bėidiu} nahezu gleich häufig vor. Beim unbelebten
Bezug auf unterschiedliche Genera steht andererseits hauptsächlich neutral
\norm{si/diu bėidiu} \autocites[95--99,
126--128]{askedal1973}[nach][]{maroldschroeder1969}.

Zusammenfassend sieht \citeauthor{askedal1973} den Grund für
Ausgleichserscheinungen in einer \q{Markiertheitshierarchie} der Genera im
Deutschen\il{Neuhochdeutsch} \autocite[241--247]{askedal1973}, aus der sich
seiner Auffassung nach ergibt, dass beim kombinierten Bezug auf Referenten mit
verschiedenem Geschlecht \blockcquote[253]{askedal1973}{auf dasjenige Merkmal
zurückgegriffen werden \textins{muß}, das weder [+\,Mask] noch [+\,Fem] ist,
nämlich [+\,Neutr], das zwar sexuell bezogen bleibt, aber die natürlichen
Geschlechts\-unterschiede innerhalb der zu pronominalisierenden Konfiguration
neutralisiert}. \citet[173--177]{askedal1973} argumentiert basierend auf
\citeauthor{greenberg1966}s Universalie 36 weiter,%
%
	\footnote{Diese besagt, dass die Kategorie Genus die Existenz der Kategorie
		Numerus voraussetzt:
		\foreignblockcquote{english}[112]{greenberg1966}{If a language has
		the category of gender, it always has the category of number}.%
	}
%
dass Pluralität dem Genus als Flexionskategorie semantisch übergeordnet sei,
was den Verlust der Genusdistinktion im Plural befördere. Die Ausweitung der
maskulin-femininen Form in der adjektivischen
Plural\-flexion\is{Adjektivdeklination} stelle darüber hinaus eine
Zwischenstufe zur vollständigen Beseitigung der Genusflexion im Plural dar.

Bezüglich der \q{Markiertheit} der Genera im grammatischen System des
Deutschen\il{Neuhochdeutsch} stützt sich \citet{askedal1973} hauptsächlich auf
Überlegungen von \citet{jakobson1932} und \citet{bierwisch1967}. Er vertritt
die Ansicht, dass das Maskulinum eine \q{unmarkierte} Form darstellt,
\textcquote[241]{askedal1973}{weil es weder eine Spezifikation weiblichen
Geschlechts noch eine der \q{Asexualität} beinhaltet}. Damit setzt er mit
direktem Verweis auf \citeauthor{jakobson1932}s Untersuchung zum
Russischen\il{Russisch} auch für das Deutsche die Merkmalskombination
[\textsc{±\,f, ±\,n}] zur paradigmatisch-struktu\-rellen Definition der Genera
an. Dass das Maskulinum in der deutschen Grammatik grundlegender ist als das
Femininum macht er daran fest, dass \textcquote[242]{askedal1973}{\textins{b}ei
verallgemeinerndem Bezug \textelp{} Mask.\ obligatorisch \textins{ist}},
abgesehen von der Beobachtung, dass gerade bei Anaphora daneben auch Plural
Neutrum mit Bezug auf belebte\is{Animata} Paare auftritt. \citet{askedal1973}
verwendet den Terminus \q{unmarkiert} im Grunde also synonym zu dem, was
\citet[205--218]{corbett1991} und \citet{wechsler2009} als \q{\isi{Default}}
bezeichnen.

Die Überlegungen \citeauthor{askedal1973}s zum semantischen Informationsgehalt
und zum Ausgleich von Genusmerkmalen spiegeln sich ansatzweise bei
\citet{corbett1991} wider. Dieser argumentiert positivistisch und
sprachökonomisch anhand von Beispielen aus verschiedenen europäischen Sprachen,
dass nicht allein paradigmatische oder semantische \q{Unmarkiertheit} den
Ausschlag zur Wahl eines bestimmten Genus als Ergebnis von Gender Resolution
gebe \autocite[290--293]{corbett1991}, sondern dieses \q{Resolutionsgenus} in
jedem Fall eine semantische Motivation benötige und die Plural\-kategorie
möglichst klar kennzeichnen sollte \autocite[293--299]{corbett1991}. Am
Beispiel des Isländischen\il{Isländisch}, das ein dreigliedriges Genussystem
mit dem Neutrum als Resolutionsgenus besitzt \REF{ex:icelgendres}, zeigt er,
dass grammatisch neutrale Bezeichnungen in Bezug auf Menschen\-gruppen oder im
Geschlecht nicht festgelegte Personen in der Sprache vorkommen, was das Neutrum
als Möglichkeit zum Ausgleich von Genusmerkmalen legiti\-miere.

\begin{exe}
\ex \label{ex:icelgendres}
	\langinfo%
		{Isländisch}%
		{}
		{\cites[nach][283]{corbett1991}[569]{wechsler2009}}\\
	\gll Drengurinn og telpan eru þreytt. \\
		Junge[\textsc{m.sg}] und Mädchen[\textsc{f.sg}] sind
		müde[\textsc{n.pl}] \\
	\trans `Der Junge und das Mädchen sind müde.'
\end{exe}

Darüber hinaus sei zumindest für den Plural Maskulinum und Femininum gegeben,
dass die jeweiligen Flexions\-endungen eindeutig die Pluralkategorie markieren,
während sich der Plural Neutrum mit dem Singular Femininum überschneide
\autocite[298--299]{corbett1991}. \tabref{tab:faerislmhdadj} gibt die jeweiligen
Formen für das Färöische\il{Färöisch} \autocite[100--101]{thrainsson2004},
Isländische\il{Isländisch} \autocite[84--90]{kress1982} sowie für das
Mittelhochdeutsche\il{Mittelhochdeutsch} in seiner
oberdeutschen\il{Oberdeutsch} Ausprägung \autocites[182]{ksw2} zum Vergleich
an.

\begin{table}
\centering
\caption{Flexionsendungen starker Adjektive\is{Adjektivdeklination} im
		Nom./Akk.~Sg.~F. und Pl.~M./F./N. des Färöischen\il{Färöisch},
		Isländischen\il{Isländisch} und
		Mittelhochdeutschen\il{Mittelhochdeutsch}}
\begin{tabular}{
	l l
	c c c c
}
\lsptoprule

\mr{2}{*}[-.5ex]{Sprache}
	& \mr{2}{*}[-.5ex]{Kasus}
	& \textsc{sg}
	& \mc{3}{c}{\textsc{pl}}
	\\

\cmidrule(rl){3-3}
\cmidrule(l){4-6}

%
	& %
	& \textsc{f}
	& \textsc{n}
	& \textsc{m}
	& \textsc{f}
\\

\midrule

Färöisch
	& \textsc{nom}
	& \cellcolor{black!50}{-Ø}
	& \cellcolor{black!50}{-Ø}
	& -ir
	& \cellcolor{black!67}{\color{white}{-ar}}
	\\

%
	& \textsc{acc}
	& -a
	& \cellcolor{black!50}{-Ø}
	& \cellcolor{black!67}{\color{white}{-ar}}
	& \cellcolor{black!67}{\color{white}{-ar}}
	\\

\midrule

Isländisch\il{Isländisch}
	& \textsc{nom}
	& \cellcolor{black!50}{-Ø}
	& \cellcolor{black!50}{-Ø}
	& -ir
	& \cellcolor{black!67}{\color{white}{-ar}}
	\\

%
	& \textsc{acc}
	& \cellcolor{black!33}{-a}
	& \cellcolor{black!50}{-Ø}
	& \cellcolor{black!33}{-a}
	& \cellcolor{black!67}{\color{white}{-ar}}
	\\

\midrule

Mittelhochdeutsch\il{Mittelhochdeutsch}
	& \textsc{nom}
	& \cellcolor{black!50}{-iu}
	& \cellcolor{black!50}{-iu}
	& \cellcolor{black!33}{-e}
	& \cellcolor{black!33}{-e}
	\\

(oberdeutsch\il{Oberdeutsch})
	& \textsc{acc}
	& \cellcolor{black!33}{-e}
	& \cellcolor{black!50}{-iu}
	& \cellcolor{black!33}{-e}
	& \cellcolor{black!33}{-e}
	\\

\lspbottomrule
\end{tabular}
\label{tab:faerislmhdadj}
\end{table}

Trotz seines abweichenden\is{Ausnahme} Formenbestandes trifft Ähnliches auch
auf das Mittelhochdeutsche\il{Mittelhochdeutsch} zu, insofern die Endung
\norm{-e} der starken \isi{Adjektivdeklination} im Nominativ nur im Plural zu
finden ist und sich daneben der Plural Neutrum \mbox{(\norm{-iu})} mit dem
Singular Femininum überschneidet, wie im Isländischen\il{Isländisch} und
Färöischen\il{Färöisch}. Nimmt man den Akkusativ hinzu, fällt auch der
Nom.~Pl.~M./F.\ (\norm{-e}) mit dem Akk.~Pl.~M./F.\ und dem Akk.~Sg.~F.\
zusammen (vgl.~auch \tabref{tab:ahd_stradj} zum Althochdeutschen). Daneben
besitzt auch das Mittelhochdeutsche\il{Mittelhochdeutsch} neutrale Substantive
mit persönlichem Bezug wie
\norm{kint} `Kind',
\norm{gesinde} `Gefolge, Dienerschaft',
\norm{hėr} `Heer, Schar', % Menge, Volk},
\norm{hīwische} `Geschlecht, Familie', % Hausgesinde}
oder \norm{volc} `Leute, Volk'. %, Schar}.

Wenngleich \citet{askedal1973} zugute zu halten ist, dass er den
Variantenapparat der von ihm verwendeten Editionen in seine Auswertung
miteinbezogen hat, besteht zumindest tendenziell die Frage, inwiefern
morphologische Variation durch die Normalisierungs\-praxis älterer Editionen in
ihrem Anspruch, den Archetyp eines Texts zu rekonstruieren, verdeckt wird.

%%%%%%%%%%%%%%%%%%%%%%%%%%%%%%%%%%%%%%%%%%%%%%%%%%%%%%%%%%%%%%%%%%%%%%%%%%%%%%%

\section%
	{\norm{Bėide} als Konjunktion -- \norm{bėide X unde Y}}
\label{sec:ovwbeideconj}

Bei dem mittelhochdeutschen\il{Mittelhochdeutsch} Ausdruck \norm{bėide \dots\
unde} `sowohl \dots\ als auch' handelt es sich um eine korrelative
Konstruktion, insofern die zwei Teile zusammen eine feste Einheit bilden. Das
einleitende \norm{bėide} bedingt das Auftreten von \norm{unde}, genauso wie bei
seiner modernen Entsprechung \fw{sowohl} nicht ohne \fw{als auch} stehen kann.
\citet[367]{dalrymple2001} bezeichnet Lexeme wie \norm{bėide} in diesem
Zusammenhang als \textit{preconjunctions} (\textsc{preconj}), doch ist mit
\citet[419]{johannessen2005} anzumerken, dass der Status und die Bezeichnung
von derlei Funktionswörtern umstritten ist. Sie spricht selbst von Korrelativen
(\textit{correlatives}) als Gattung funktionaler Adverbien\is{Adverb} und fasst
darunter neben \norm{both} `beide' zum Beispiel auch \mbox{\fw{either}}
`entweder' und \norm{neither} `weder'. In Abgrenzung zur Funktion von
\norm{bėide} als Quantor wird im Folgenden der Einfachheit halber von seiner
anderen Rolle als Konjunktion die Rede sein.

\citet[425--428]{johannessen2005} konkretisiert des Weiteren, dass sich diese
Adverbien\is{Adverb} ähnlich wie Fokuspartikeln verhalten. Zu dieser kleinen
Gruppe von funktionalen Adverbien\is{Adverb} zählen Ausdrücke wie \fw{allein},
\fw{nur} oder \fw{auch}, die eine ganze Reihe von Funktionen ausüben können,
darunter, dass sie Alternativen einführen oder anzeigen und innerhalb ihres
Skopus quantifikatorische Bedeutung haben \autocite[vgl.][1--4,
15]{koenig1991}. In Beispiel \REF{ex:focpart1_1} beschränkt \fw{nur} die
Aussage dahingehend, dass unter den Alternativen \fw{Anna} und \fw{Christian}
allein die erstere zutrifft. Die Betonung auf der syntaktischen Einheit, über
die die Fokuspartikel Skopus hat, ist dabei typisch
\autocite[10--14]{koenig1991}. Ähnlich verhält es sich bei \REF{ex:focpart2_1},
wo \fw{entweder} explizit macht, dass nur eines der beiden Konjunkte in seinem
Skopus, \fw{Saft} und \fw{Limo}, zur Auswahl steht. Auch in diesem Kontext
tragen die Konjunkte jeweils eine Betonung. Die korrespondierenden Fragen in
\REF{ex:focpart1_2} und \REF{ex:focpart2_2} verdeutlichen, dass die vom
Adverb\is{Adverb} eingeleitete NP den Fokus des jeweiligen Satzes darstellt.

\begin{exe}
\protectedex{
\ex\label{ex:focpart_1}
\begin{xlist}
	\ex \label{ex:focpart1_1}
		Nur \emph{Anna} kommt zu Besuch, Christian bleibt zu
		Hause.
	\ex \label{ex:focpart1_2}
		Wer kommt zu Besuch? --- \emph{Anna.}
\end{xlist}

\ex \begin{xlist}
	\ex \label{ex:focpart2_1}
		Du kannst entweder \emph{Saft} oder \emph{Limo} trinken.
	\ex \label{ex:focpart2_2}
		Was möchtest du trinken? --- \emph{Saft/Limo.}
\end{xlist}
}
\end{exe}

Im Allgemeinen wird davon ausgegangen, dass sich die
mittelhochdeutsche\il{Mittelhochdeutsch} Konstruktion \norm{bėide \dots\ unde}
`sowohl \dots\ als auch' aus einer appositiven\is{Apposition} Struktur
entwickelt hat, wie sie auch im modernen Deutschen\il{Neuhochdeutsch} möglich
ist: \fw{beide, \emph{X} und \emph{Y}} \autocite[vgl.][626--627 und die
dortigen Referenzen]{ksw2}. Eine Recherche im \citetitle{ddd} (\tit{ReA};
\nosh\cite{ddd}) nach der Konstruktion \norm{bėide \dots\ joh} `sowohl \dots\
als auch' \autocite[vgl.][49]{schuetzeichel2012} hat die Belegtypen in
(\ref{ex:beidejohahd_1}--\ref{ex:beidejohahd_3}) für das
(Spät-)Althochdeutsche\il{Althochdeutsch} (11.~Jh.) ergeben. \citet{braune2018}
und \citet{schrodt2004} bieten keine Beispiele dies\-bezüglich.

%%%%%%%%%%%%%%%%%%%%%%%%%%%%%%%%%%%%%%%%

In \REF{ex:beidejohahd_1} dient \lit{péidíu} `beide' der Ankündigung, dass die
Kombination von zwei Konjunkten folgt. Es ist in diesem Zusammenhang noch als
kataphorisches Pronomen aufzufassen. Die eigentliche Konjunktionskonstruktion
besteht aus \norm{joh \dots\ joh} `sowohl \dots\ als auch'
\autocite[vgl.][169]{schuetzeichel2012}. Aufgrund der Pluralform wurde
angenommen, dass \lit{péidíu} sich hier auf die prädikativen
Adjektive\is{Adjektiv!prädikativ} \lit{míchel} `groß' und \lit{lúzzel} `klein'
bezieht und daher im Plural Neutrum steht.

\begin{exe}
\ex \label{ex:beidejohahd_1}
	\langinfo%
		{Althochdeutsch}%
		{}%
		{\cite[nach][57]{king1972}}\\
\gll táz péidíu íst ióh míchel ióh lúzzel \\
	\textsc{rel.nom.sg.\NeutI} beide-\textsc{nom.pl.\NeutI} ist und
	groß[\textsc{nom.sg.\NeutI}] und klein[\textsc{nom.sg.\NeutI}] \\
\trans `das beides ist, sowohl klein als auch groß.'
	(%
		\iai{Notker~III.\ von St.~Gallen}, \tit{Boethius: Categoriae}: 2,15%
	)
\end{exe}

In \REF{ex:beidejohahd_2} suggeriert der Punctus (\lit{·}) als Sprechpause
ebenfalls eine kataphorische Interpretation, zumal \lit{pêide} `beide' hier
ebenfalls im Genus mit den Konjunkten übereinstimmt.

\begin{exe}
\ex \label{ex:beidejohahd_2}
	\langinfo%
		{Althochdeutsch}%
		{}%
		{\cite[nach][35]{tax1979}}\\
	\gll Trúhten besuôchet pêide · guôten ioh úbelen \\
		Herr befragt beide-\textsc{acc.pl.\MascA.st} {}
			Gut-\textsc{acc.sg.\MascA.wk} und
			Böse-\textsc{acc.sg.\MascA.wk} \\
	\trans `Der Herr befragt beide, den Guten und den Bösen.'
		(%
			\iai{Notker~III.\ von St.~Gallen}, \tit{Psalter}: 10,6%
			% = Vulgata Ps LXX 10,6 ~ Luther Ps 11,5
		)
\end{exe}

Die Formulierung ist aber ähnlich wie die in \REF{ex:beidejohahd_3}, bei der
\lit{béidíu} `beide' als Katapher oder als Konjunktion aufgefasst
werden kann.

\begin{exe}
\ex \label{ex:beidejohahd_3}
	\langinfo%
		{Althochdeutsch}%
		{}%
		{\cite[nach][6]{king1972}}\\
	\gll Tíu múgen sîn béidíu propria ióh appellatiua \\
		\textsc{dem.nom.pl.\NeutI} können sein beide-\textsc{nom.pl.\NeutI}
			\fw{proprius}-\textsc{nom.pl.\NeutI} und
			\fw{appellātīvus}-\textsc{nom.pl.\NeutI} \\
	\trans `Die können sowohl Propria als auch Appellativa sein.'
		(%
			\iai{Notker~III.\ von St.~Gallen}, \tit{Boethius: Categoriae}: 1,3%
		)
\end{exe}

Des Weiteren ist an \REF{ex:beideintiahd_3} mit \lit{unte} statt \lit{joh}
`und' auffällig, dass \lit{pediu} `beide' schon mit Präpositionalphrasen steht
(\lit{in demo lihnamen} `im Körper', \lit{in demo muôte} `im Geist'), die als
solche keine Personenmerkmale definieren, mit denen \lit{pediu} kongruieren
könnte.

\begin{exe}
\protectedex{%
\ex \label{ex:beideintiahd_3}
	\langinfo%
		{Althochdeutsch}
		{}
		{\cite[nach][171]{steinmeyer1916}}\\
	\gll pediu in demo lihnamen unte in demo muôte \\
		beide in \textsc{def.dat.sg.\MascI} Körper-\textsc{dat.sg.\MascI} und in
			\textsc{def.dat.sg.\MascI} Geist-\textsc{dat.sg.\MascI} \\
	\trans `sowohl im Körper als auch im Geist'
		(%
			\tit{Predigtsammlung~B}: 3,25--26%
		)%
}
\end{exe}

Für die mittelhochdeutsche\il{Mittelhochdeutsch} Periode liegt mit
\citet{askedal1974} ein Aufsatz vor, der sich mit der
mittelhochdeutschen\il{Mittelhochdeutsch} Entsprechung der Konstruktion
\fw{sowohl \dots\ als auch} beschäftigt. Entgegen dem, was zum Beispiel bereits
\citet[433--434]{behaghel1923} feststellt -- nämlich, dass schon im
Althochdeutschen\il{Althochdeutsch} eine Erstarrung von
\norm{bėide} in diesem Kontext eintritt --, vertritt \citeauthor{askedal1974}
die These, dass sich Konstruktionen dieses Typs noch in Gottfrieds
\tit{Tristan}\nocite{maroldschroeder1969} und Wolframs
\tit{Parzival}\nocite{lachmannhartl1952} \textquote{als recht feste Gefüge
manifestieren, die immer noch nahe an der appositiven\is{Apposition}
Ausgangsstruktur sind}, dementsprechend also
\blockcquote[37]{askedal1974}{überwiegende Einhaltung von Kongruenzregeln, wo
dies möglich ist}, vorherrsche. Er betrachtet dabei nur Kontexte wie in
\REF{ex:mhdbeideunde2}, in denen Substantive koordiniert werden, während die
Konstruktion im Mittelhochdeutschen\il{Mittelhochdeutsch} auch mit anderen
Konjunkten als Substantiven und Nominalphrasen auftritt.

\begin{exe}
\ex\label{ex:mhdbeideunde2}
	\gll Hi bi warint beide ritire vn̄ burgere \\
		Hier bei waren beide Ritter-\textsc{nom.pl.\MascM} und
			Bürger-\textsc{nom.pl.\MascA} \\
	\trans %
		`Anwesend waren sowohl Ritter als auch Bürger' (alternativ: `Anwesend
		waren beide, Ritter und Bürger'; \cites(Nr.~N~321, Rosheim,
		Dépt.~Bas-Rhin, 1286)[245,29]{cao5})
\end{exe}

\citet[187]{gjelsten1980} räumt in ihrer Replik auf \posscite{askedal1974}
Aufsatz ein, dass die Struktur von Belegen wie denen in \REF{ex:mhdbeideunde2}
nicht ganz eindeutig\is{Ambiguität} sei, da man sie sowohl als \fw{beide,
\emph{X} und \emph{Y}} und als \fw{sowohl \emph{X} als auch \emph{Y}} lesen
könne. Allerdings widerspricht sie \citeauthor{askedal1974}s Behauptung mit
scharfen Worten. So sei \blockcquote[196]{gjelsten1980}{der kühne Versuch
unternommen worden, die komplementäre Verteilung von \fw{beide} und \fw{beidiu}
anhand von einem Material nachzuweisen, in dem die Belegmasse derart homogen
ist, daß sie einem nicht einmal den Gedanken eingeben kann, daß sich die
syntaktischen Umgebungen, in denen die Form \fw{beide} auftritt, durch die
beschriebenen Merkmale von denjenigen unterscheiden, in denen die Form
\fw{beidiu} erscheint.}

Die Variation in der Form der Konjunktion betreffend stellen \citet[628]{ksw2}
fest, dass \blockquote{\textins*{f}ür das Auftreten von \norm{bėide} neben
regelhaftem \norm{bėidiu} im Obd.\ \textelp{} keine durchgängige
Bedingung\is{Bedingung} erkennbar \textins{ist}}, und verweisen ihrerseits auf
\citet{gjelsten1980}, jedoch nicht ganz kritiklos: Sie schließe aus ihrer
Nachauswertung von \textquote{über 1000 bewertbare\textins*{n}
Belege\textins*{n}} aus \blockcquote[198]{gjelsten1980}{über dreißig Werken},
dass \norm{bėide} gegenüber \norm{bėidiu} überwiege, was sich aufgrund der
Korpusbefunde der Autoren nicht bewahrheitet habe. Sie mutmaßen, dass
\citeauthor{gjelsten1980} bei ihrer Textauswahl diachrone\is{Diachronie} und
geografische\is{Sprachgeografie} Aspekte außer Acht gelassen haben könnte.
Überprüfbar ist diese Vermutung nicht, weil sie keine Auskunft über ihre
Textauswahl gibt.

Fälle von \norm{bėiden} `beiden (\textsc{dat.pl})' oder \norm{bėider} `beider
(\textsc{gen.pl})' mit appositiver\is{Apposition} Konjunktion konnten in einer
Stichprobe anhand der automatischen Annotation\is{Annotation} des \tit{Corpus
der altdeutschen Originalurkunden} (\CAO) nach \citet{schmid2019} nicht
gefunden werden. Auch eine manuelle Durchsicht der hier verwendeten
\tit{Kaiserchronik}-Handschriften (\KC) ergab keine Treffer. \citet[626]{ksw2}
nennen zumindest die zwei Beispiele in \REF{ex:gendatconj} aus literarischen
Texten, die dort \textquote{noch} vorkommen.

\begin{exe}
\ex \label{ex:gendatconj}
	\begin{xlist}
	\ex \label{ex:gendatconj_1}
		\gll er lov̂ffet nv nachet beîder. \\
			er läuft nun nackt beide-\textsc{gen.pl.st} \\
	\sn \gll der ſinne vn̄ der cleîder. \\
			der Verstand-\textsc{gen.pl} und der
				Kleid-\textsc{gen.pl} \\
		\trans `Er läuft nun umher nackt an beiden, dem
			Verstand und den Kleidern.'
			(%
				\iai{Hartmann von Aue}, \tit{Iwein}: V.~3359--3360 nach
				Gießen, Universitätsbibl., Hs~97: 65v,2--3;
				% [\cite[1102]{hsc}];
				\cite[vgl.][500]{mertens2004}%
			)

	\protectedex{%
	\ex \label{ex:gendatconj_2}
		\gll er miſſete gern ir beider. \\
			er entbehrte gern ihr beide-\textsc{gen.pl.st} \\
	\sn \gll der boſten vnt der beſten. \\
			der Geringsten[\textsc{gen.pl}?] und der
			Bester[\textsc{gen.pl}?] \\
		\trans `Er entbehrte gern ihr beider, der Geringsten
			und der Besten.'\footnotemark{}
			(%
				\iai{Wolfram von Eschenbach}, \tit{Parzival}: 375,6--7 nach
				St.~Gallen, Stiftsbibl., Cod.~Sang.~857: 108b,6--7;
				% [\cite[1211]{hsc}];
				\cite[vgl.][379]{knechtschirok2003}%
			)%
	}
	\end{xlist}
\end{exe}
%
	\footnotetext{Die Formen \lit{der boſten} und \lit{der besten} sind
		ambig\is{Ambiguität} bezüglich ihrer Flexion, da \norm{dęr \dots-en}
		sowohl für den Gen.~Sg.~F.\ als auch für den Gen.~Pl.\ aller Genera
		steht \autocite[182, 433]{ksw2}. \citet[379]{knechtschirok2003}
		übersetzen diese Stelle dem Kontext nach frei mit \textquote{Das Beste
		gab er hin mit leichter Hand wie das Geringste}. Denkbar wäre also die
		Ellipse eines Substantivs wie \norm{dinc} `Ding, Sache
		(\textsc{pl.\NeutI})'.%
	}

Wenn alles so klar erscheint, wieso lohnt sich dann trotzdem ein Blick auf die
Konstruktion? Hier kommt das Argument der geografischen\is{Sprachgeografie}
Verteilung zum Tragen. \citet[627]{ksw2} geben in einer Tabelle die räumliche
und zeitliche Verteilung von \norm{bėidiu} an. Zugleich enthält ihr Korpus nur
verhältnismäßig wenige Urkunden aus dem 13.\ und 14.\ Jahrhundert. Durch den
Vergleich der Urkundenbelege mit der
\tit{Mittelhochdeutschen\il{Mittelhochdeutsch} Grammatik} lässt sich die
Zuverlässigkeit des \CAO{} vor dem Hintergrund der Überlieferung überprüfen,
zumal sich Urkunden besonders durch ihre kleinräumige Auflösung hervortun.
Umgekehrt lassen sich geografische\is{Sprachgeografie} Angaben der Grammatik
zumindest für das späte 13.~Jahrhundert mit dem \CAO{} anhand seiner relativ
klar lokalisierbaren Daten verifizieren. Die \KC{} ist andererseits aufgrund
ihres langen Tradierungs\-zeitraums reizvoll, um auch die
diachrone\is{Diachronie} Perspektive zumindest im Ansatz zu untersuchen --
aufgrund ihrer regionalen Verbreitung hauptsächlich im bairischen\il{Bairisch}
Sprachgebiet.
